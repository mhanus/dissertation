
% Thesis Abstract -----------------------------------------------------

\begin{alwayssingle} \pagestyle{empty}
  \begin{center}
  \vspace*{1.5cm}
  {\Large \bfseries  Abstract}
  \end{center}
  \vspace{0.5cm}
\begin{quote}

The subject of this work is computational modeling of neutron transport relevant to economical and safe operation of
nuclear facilities. The general mathematical model of neutron transport is provided by the linear Boltzmann's transport
equation\comment{This first-order partial differential equation with integral terms describes the dependence of neutron
field on six phase-space variables (position, direction, energy)} and the thesis begins with its precise mathematical
formulation and presentation of known conditions for its well-posedness.

\comment{High-dimensionality and complicated structure of the equation for real-world problems preclude analytical
solution and present serious difficulties when a computational solution is attempted. Several dimension-reduction
techniques thus need to be employed along with appropriate numerical solution schemes. This is the subject of the}
In the following part, we study approximation methods for the transport equation, starting with the classical
discretization of energetic dependence and followed by the review of two most widely used methods for approximating 
directional dependence (the $\SN$ and $\PN$ methods). While these methods are usually presented independently of each 
other, we show that they can be put into a single framework of Hilbert space projection techniques. This fact is then
used in conjunction with the results of the first part to rigorously prove rotational invariance of the $\PN$
equations and to analyze convergence of the basic iterative scheme for solving the $\SN$ equations. This part of the
thesis is concluded by the description of a finite element method for the final discretization of spatial dependence and a discussion of the solution of the resulting system of
algebraic equations.

The main new results are contained in the following two chapters focusing on the simplified $\PN$ approximation, which
is a computationally more convenient albeit not as mathematically well-founded variant of the $\PN$ approximation. We
prove well-posedness of the weak form of the $\SPN[3-7]$ equations and present a new way of deriving the equations from
an alternative set to the $\PN$ equations, obtained from special linear combination of
spherical harmonics.% (the so-called Maxwell-Cartesian spherical harmonics, hence the abbreviation $\MCPN$ for this
% approximation), u%tilizing its tensor structure. Algebraic manipulations allowing to obtain the $\SPN[3]$-equivalent equations from the
%$\MCPN[3]$ set in an interior of a homogeneous region are explicitly given in \sref{sec:mcp3_red}.

The final part of the thesis contains numerical examples of the $\SN$ and $hp$-adaptive $\SPN$ calculations using a
neutronics framework that has been implemented by the author to the $hp$-adaptive finite element library Hermes2D. The $\SPN[1]$
(or diffusion) model also serves as a basis of a real-world reactor calculation suite co-developed by the author for the
purposes of ``Project TA01020352 -- Increasing utilization of nuclear fuel through optimization of an inner fuel cycle and
calculation of neutron-physics characteristics of nuclear reactor cores''. An example benchmark used to test the code
concludes the thesis.
 

\vspace*{1cm}
%\begin{quote}
{\large \bfseries  Keywords:}

The neutron diffusion approximation, whereby the NTE is reduced to a second-order elliptic PDE (or, when energy
dependence is taken into account implicitly, a weakly coupled non-symmetric system of second-order PDEs with
positive-definite symmetric part -- the so-called \textit{multigroup neutron diffusion approximation}), also forms the

\end{quote}

\clearpage
\selectlanguage{czech}
  \begin{center}
  \vspace*{1.5cm}
  {\Large \bfseries  Abstrakt}
  \end{center}
  \vspace{0.5cm}
\begin{quote}

Práce se zabývá matematickým a numerickým modelováním transportu neutronů, se zaměřením na výpočty neutronových
charakteristik jaderných reaktorů. Obecný matematický model transportu neutronů je reprezentován lineární Boltzmannovou
transportní rovnicí. Práce začíná její přesnou matematickou formulaci a přehledem\linebreak výsledků týkajících se její
řešitelnosti ve druhé kapitole. Následující kapitoly jsou zaměřeny na přibližné metody řešení této rovnice.

Po stručném popisu klasické diskretizace energetické závislosti je hlavní část třetí kapitoly věnována aproximaci
směrové závislosti pomocí dvou stěžejních metod -- metody diskrétních ordinát ($\SN$) a metody sférických harmonických
funkcí ($\PN$). Zatímco obvykle jsou tyto metody formulovány nezávisle, v práci je ukázáno, jak je lze obě popsat pomocí
jednotného rámce jako projekci na podprostor Hilbertova prostoru funkcí definovaných na sféře. Této skutečnosti je
posléze využito při důkazu rotační invariantnosti $\PN$ rovnic a při konvergenční analýze základní iterační
metody pro řešení $\SN$ soustavy.
Třetí kapitola je zakončena popisem aplikace metody konečných prvků na finální diskretizaci prostorové závislosti.

Hlavní nové výsledky této práce se týkají metody zjednodušených sférických harmonických funkcí ($\SPN$), jež představuje
výpočetně efektivní aproximaci metody $\PN$. Ve čtvrté kapitole je standardním \linebreak způsobem odvozena slabá
formulace $\SPN$ rovnic a dokázána její korektnost pro $N = 3,5,7$. V páté kapitole je pak odvozena
nová soustava parciálních diferenciálních rovnic odpovídající $\PN$ aproximaci ($\MCPN$ aproximace). Na příkladu $\MCPN[3]$ aproximace je ukázáno, jak lze využít
tenzorovou strukturu těchto rovnic k transformaci na soustavu ekvivalentní s $\SPN[3]$ aproximací.

V šesté kapitole je popsána implementace $\SN$ a $\SPN$ aproximací do knihovny Hermes2D a na několika
příkladech ukázány základní vlastnosti těchto aproximací.  Speciální pozornost je věnována implementaci nespojité
Galerkinovy metody (pro $\SN$ aproximaci)%, umožňující využít speciální způsob asemblace soustav PDR v knihovně
% Hermes2D, v níž je každá neznámá funkce aproximována na vlastním konečně-prvkovém prostoru. 
a modifikaci standardního indikátoru chyby pro
$hp$-adaptivitu v Hermes2D pro $\SPN$ aproximaci. Práce je ukončena ukázkou řešení standardního 3D benchmarku pomocí
mnohagrupového difúzního kódu, který autor na základě zkušeností s vývojem neutronických modulů v knihovně Hermes2D
vyvinul pro účely projektu ``TA01020352 -- Zvýšení využití jaderného paliva pomocí optimalizace vnitřního palivového
cyklu a výpočtu neutronově-fyzikálních charakt. aktivních zón jaderných reaktorů''.



  \vspace*{1cm}
{\large \bfseries  Klíčová slova:}

%\begin{quote}

The neutron diffusion approximation, whereby the NTE is reduced to a second-order elliptic PDE (or, when energy
dependence is taken into account implicitly, a weakly coupled non-symmetric system of second-order PDEs with
positive-definite symmetric part -- the so-called \textit{multigroup neutron diffusion approximation}), also forms the


\end{quote}

\end{alwayssingle}\selectlanguage{english}


% ---------------------------------------------------------------------- 
