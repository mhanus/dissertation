\chapter{Mathematical model of neutron transport}\label{sec:bte}

\ifpdf
    \graphicspath{{2_BTE/figures/PNG/}{2_BTE/figures/PDF/}{2_BTE/figures/}}
\else
    \graphicspath{{2_BTE/figures/EPS/}{2_BTE/figures/}}
\fi

\section{Neutronic phase space} \label{sec:phase}
Continuing with the notation introduced in \ref{sec:bte-intro}, let us define the neutronic phase space $$
  X := \{(\br,\bomega,E):\ \br\in V\subset\R[3], \bomega\in \Sphere, E \in [\Emin,\Emax]\}
$$ together with its outflow and inflow boundary subsets, respectively:
$$
  \pX[\pm] := \bigl\{ (\br,\bomega,E) \in \pV \times \Sphere \times [E_m, E_M], \mbox{ s.t. } \bomega\cdot\bn(\br)
  \gtrless 0 \bigr\}
$$ We will assume that $V$ is a convex domain with piecewise smooth boundary $\pV$, oriented by its outward normal field
$\bn(\br)$. The direction vector $\bomega = \bv/v$ has the following Cartesian coordinates:
\begin{equation*}
	\bomega = \left[\begin{array}{c}
		\Omega_x \\
		\Omega_y \\
		\Omega_z
	\end{array}\right] = \left[\begin{array}{c}
		\sint\cosp \\
		\sint\sinp \\
		\cost
	\end{array}\right].
\end{equation*}
The product measure
\begin{equation}\label{eq:measure}
  \d{x} = \d{\mu(X)} = \d{\mu(V\times\Sphere\times [\Emin,\Emax])} = \d{\br}\d{\bomega}\d{E}
\end{equation}
is used when integrating over $X$, where $\d{\bomega}$ is defined as the solid angle subtended at the center of
 $\Sphere$ by the angular differential element:
$$
	\d{\bomega} = \sint \d{\polar}\d{\azimuthal}
$$
while the boundary measure
$$
\db = \abs{\bomega\cdot\bn}\d{\gamma}\d{\bomega}\d{E},\quad \d{\gamma} = \d{\mu(\pV)},
$$
is used when integrating over $\pX[\pm]$.

\section{Steady state neutron transport problem}\label{sec:definitions}

Equilibrium distribution of neutrons within the phase space, originating from a time-invariable volumetric source
$\src(\br,\bomega,E)$ within $V$, is given by the time-independent version of eq. \eqref{eq1}:
\begin{equation}\label{eq:bte1}
\begin{multlined}
  \bomega\cdot\nabla\psi(\br,\bomega,E) + \sigma_t(\br,E)\psi(\br,\bomega,E) =\\[.25em]
   = \intE[']{\Emin}{\Emax}{
      \intA[']{\kappa(\br,\bomega\cdot\bomega',E\sla E')\psi(\br,\bomega',E')}
    } + q(\br,\bomega,E).
 \end{multlined}
\end{equation}
together with specified distribution at the inflow boundary. The two most often used inflow boundary conditions are:
\begin{itemize}
	\item incoming angular neutron flux
	\begin{equation}\label{eq:bte2}
	  \psi\vert_{\pX[-]} = \psi_{\text{in}}
	\end{equation}
	($\psi_{\text{in}}\equiv 0$ corresponds to vacuum in $\R[3]\setminus \overline V$, which is a common
	 assumption in nuclear reactor modeling),
	
	\item specular reflection at boundary
	\begin{equation}\label{eq:bte3}
  	\psi(\br,\bomega,E) = \psi(\br, \bomega_R, E),\quad (\br,\bomega,E)\in \pX[-],\ \ \bomega_R = \bomega - 2 \bn 
  	(\bomega \cdot \bn) 
  \end{equation}
  (this condition is used to model planes of symmetry).
\end{itemize}
For more general boundary conditions used in conjunction with eq. \eqref{eq:bte1}, see e.g. \cite{Sanchez3}.

A physically plausible solution of BTE should be moreover non-negative throughout $V$ and continuous along any direction
$\bomega$, i.e. $\psi(\br + s\bomega,\bomega,E)$ is a continuous function of $s$ for any $\br$, $\bomega$, $E$. Note
that $\psi(\br + s\bomega',\bomega,E)$ \textsl{may} be discontinuous when $\bomega' \neq \bomega$.

The kernel of the integral operator on the right-hand side is commonly split as follows:
$$
  \kappa(\br,\bomega\cdot\bomega',E\sla E') = \sigma_s(\br,\bomega\cdot\bomega',E\sla E') +
  \frac{\chi(E)\nu(\br,E')\sigma_f(\br,E')}{4\pi}
$$ where $\sigma_s$ is called \textit{double-differential macroscopic cross-section for scattering} and $\sigma_f$ the
\textit{macroscopic cross-section for fission}. We note in the latter that fission is an isotropic process (thus
$\sigma_f$ is the average value over $\Sphere$), by which a neutron with energy $E'$ releases (on average) $\nu(\br,E')$
neutrons with new energy distribution $\chi(E)$. The new energy distribution is normalized so that $$
  \intE[']{\Emin}{\Emax}{\chi(E')\sigma_f(\br,E)} = \sigma_f(\br,E).
$$ New neutrons can also be released into the system as a result of inelastic scattering but this contribution is
usually neglected in applications of interest to us.

The \textit{total macroscopic cross-section}, $\sigma_t$, characterizing the probability that neutron with energy $E$
undergoes a collision of any type with nuclei at $\br$, can be decomposed as $$
  \sigma_t(\br,E) = \sigma_a(\br,E) + \sigma_s(\br,E) = \sigma_c(\br,E) + \sigma_f(\br,E) + \sigma_s(\br,E),
$$ where
\begin{itemize}
	\item $\sigma_a(\br, E)$ $\ldots$ absorption cross section,
	\item $\sigma_c(\br, E)$ $\ldots$ non-productive capture cross section,
	\item $\displaystyle\sigma_s(\br, E) = \intE[']{\Emin}{\Emax}{\intA[']{\sigma_s(\br,\bomega'\cdot\bomega, 
	E\shortrightarrow E')}}$.
\end{itemize}
A natural assumption is that the macroscopic cross-sections are bounded measurable functions
\footnote{under the product measure \eqref{eq:measure}, resp. under the product measure $\d{\mu(V\times\Sphere^2\times
 [\Emin,\Emax]^2)}$ for the double-differential cross-section}, piecewise continuous throughout $V$. 

From the solution of eq. \eqref{eq:bte1}, one can derive the following integral quantities
\begin{itemize}
  \item \textit{scalar neutron flux density}
  \begin{equation}%\label{eq:scalar_flux}
    \phi(\br, E) = \intA{\psi(\br,\bomega,E)},
  \end{equation}
  \item \textit{net neutron current density}
	\begin{equation}%\label{eq:bJ}
		\bJ(\br, E)	= \intA{\bomega\psi(\br,\bomega,E)},
	\end{equation}
\end{itemize}
so that integrating $\phi(\br, E)$ over a given volume or $\bJ(\br,E)\cdot\bn(\br)$ over a given surface, respectively,
gives the total number of neutrons with energy $E$ within that volume or crossing that surface in the direction of 
$\bn$, respectively. Also,
\begin{equation}\label{eq:rr}
  \sigma_x(\br, E) \phi(\br, E)
\end{equation}
represents the \textit{reaction rate} (per unit time) of given type. All these quantities produce directly observable 
effects and may be experimentally measured by various detector mechanisms, which is the reason why they are more 
important in practical calculations than the actual solution of the BTE (i.e. the angular neutron flux). Note that 
for the total volumetric scalar flux to be finite (as is physically expected), we should look for the solution $\psi$ 
of the BTE in the space $\Lp[1](X)$ with boundary values in $\Lp[1](\pX[\pm])$ (equipped with the respective measures
defined in Sec. \ref{sec:phase}), which is therefore usually taken as the natural function space setting 
(\cite{DautrayLions}; we will return to this topic in Sec. \ref{sec:solvability}).

Apart from the ``fixed source'' problem defined above, the other important problem in neutron transport requires the
determination of material composition (i.e. the values of $\sigma_x$) for a given reactor core geometry (or vice versa)
which ensures a steady neutron distribution (that means -- steady power generation) with no additional neutron sources
than the fission. This is called a ``criticality problem'' and entails finding a suitable parameter with which the
homogeneous version of eq. \eqref{eq:bte1} (i.e. with $q\equiv 0$ and b.c. \eqref{eq:bte3} or \eqref{eq:bte2} with
$\psi_{\text{in}} \equiv 0$) has a non-trivial non-negative solution. The free parameter is commonly placed in front of
the fission part and plays the role of an eigenvalue of eq. \eqref{eq:bte1} -- it is well known that the desired
solution is the unique eigenfunction associated with the simple eigenvalue of smallest modulus (see e.g. \cite[Chap.
XXI, \S3]{DautrayLions} or \cite{Sanchez3}). We will not, however, cover the critical eigenvalue problem here in any
more detail.

\subsection{Solvability}\label{sec:solvability}
Let $$
  \Hp(X) = \left\{\psi\in\Lp(X, \d{x}), \bomega\cdot\nabla\psi\in\Lp(X, \d{x}), \psi\vert_{\pX[-]}\in\Lp(\pX[-],
  \db)\right\}
$$ and
\begin{equation*}
  \begin{gathered}
    A\psi(\br,\bomega,E) = \bomega\cdot\nabla\psi(\br,\bomega,E),\\%[.85em]
    \Sigma_t\psi(\br,\bomega,E) = \sigma_t(\br,E)\psi(\br,\bomega,E),\\%[.75em]
    K\psi(\br,\bomega,E) = \intE[']{\Emin}{\Emax}{
            \intA[']{\kappa(\br,\bomega\cdot\bomega',E\sla E')\psi(\br,\bomega',E')}
          }.
  \end{gathered}
\end{equation*}
We shall call $A$, $\Sigma_t$, $K$ and $T = A + \Sigma_t - K$ the \textit{advection}, \textit{collision}, 
\textit{transfer} and \textit{transport} operator, respectively. Note that the operator $\Sigma_t - K$ is self-adjoint
 if and only if the kernel $\kappa$ of $K$ is symmetric in $E$ and $E'$; with the exception of the mono-energetic case, 
 this is generally not true (as already mentioned in the introduction). We also note that for piecewise smooth $\pV$ 
 and $1\leq p < \infty$, the traces $\psi\vert_{\pX[\pm]}\in\Lp(\pX[\pm], \db)$ are well defined for functions 
$$
  \psi\in \{\varphi\mid \varphi\in\Lp(X, \d{x}), \bomega\cdot\nabla\varphi\in\Lp(X, \d{x})\}
$$
and admit a continuous lifting (\cite[Chap. XXI]{DautrayLions}, \cite{Boulanouar1}). This lifting allows us to convert 
the problem with non-homogeneous boundary conditions \eqref{eq:bte2} to that with homogeneous ones, so we will deal only 
with the latter in this section. Existence results for the reflective or more general boundary conditions require 
special trace theorems, see \cite[Chap. XXI, Appendix of \S2]{DautrayLions}.


We will now follow the general idea of \cite[Chap. XXI, \S 2, Proposition 5]{DautrayLions} to prove the following

\begin{theorem}\label{thm1}
Assume that
\begin{enumerate}[label=(\alph*)]
	\item $\sigma_t \in \Lp[\infty](X)$, $\sigma_t \geq \mst > 0$ a.e. in $V\times[\Emin,\Emax]$,
	\item $\kappa \geq 0$ a.e. in $V\times\Sphere^2\times[\Emin,\Emax]^2$,
	\item $\displaystyle\esssup_{X} c(\br,\bomega,E) < 1$ where
	  \begin{equation}\label{eq:c}
	    c(\br,\bomega,E) := \frac{1}{\sigma_t(\br,E)}\int_{\Emin}^{\Emax}\int_{\Sphere} \kappa(\br,\bomega\cdot\bomega',
	    E\shortrightarrow E')\,\d{E'}\d{\bomega'}.
	  \end{equation}
\end{enumerate}
Then the fixed source, steady state neutron transport problem with vacuum boundary conditions 
\begin{equation*}
  \left\{
  \begin{aligned}
     &T\psi(\br,\bomega,E) = q(\br,\bomega,E),\\
     &\Dom{T} = \{\psi\in \Hp[1](X),\ \psi\vert_{\pX[-]} = 0\},
  \end{aligned}
  \right.
\end{equation*}
has a unique solution $\psi(\br,\bomega,E)\in\Dom{T}\subset\Lp[1](X)$ for any $q\in \Lp[1](X)$.
\end{theorem}
\begin{proof}
  The assumptions are equivalent to 
  \begin{enumerate}
    \item[(c')] $\displaystyle\qquad\qquad 0 \leq \intE[']{\Emin}{\Emax}{\intA[']{\kappa(\br,\bomega\cdot\bomega',E\shortrightarrow E')}} \leq \sigma_t(\br,E) - \mst$
  \end{enumerate}
  for a. e. $(\br,\bomega,E) \in X$ which implies that\footnote{the sharp brackets denote the duality pairing between 
  $\Lp[1](X)$ and $\Lp[\infty](X)$ via $\psi^\ast = \sgn \psi$}
  $$
    \begin{aligned}
    \langle \psi^\ast, K\psi \rangle &= \intX{\sgn \psi(\br,\bomega,E)\intE[']{\Emin}{\Emax}{\intA[']{\kappa(\br,\bomega\cdot\bomega',E\shortrightarrow E')\psi(\br,\bomega',E')}}}\\
    & \leq 
    \intE[']{\Emin}{\Emax}{\intA[']{\psi(\br,\bomega',E') \bigl[ \sigma_t(\br,E') - \mst \bigr]}}\\ &= \intX{\psi(\br,\bomega,E) \bigl[ \sigma_t(\br,E) - \mst \bigr]}.
    \end{aligned}
  $$
  Next,  
  $$
  \begin{aligned}
    \langle \psi^\ast, (\Sigma_t - \mst\Id)\psi\rangle &= \intX{\left[\sigma_t(\br,E) - \mst\right]\psi(\br,\bomega,E)\sgn \psi(\br,\bomega,E)} \\
    & = \intX{\left[\sigma_t(\br,E) - \mst\right]\abs{\psi(\br,\bomega,E)}} \geq \intX{\left[\sigma_t(\br,E) - \mst\right]\psi(\br,\bomega,E)}
  \end{aligned}
  $$
  and hence in summary
  $$
    \langle \psi^\ast, (\Sigma_t - K - \mst\Id)\psi\rangle \geq 0.
  $$
  Similarly
  $$
    \langle \psi^\ast, A\psi\rangle = \intX{(\bomega\cdot\nabla\psi)\sgn\psi} = \intX{\bomega\cdot\nabla\abs{\psi}}
    = \int_{\pX[+]}(\bomega\cdot\bn)\abs{\psi}\,\db \geq 0.
  $$
  using the Green's formula and vanishing inflow boundary conditions to obtain the last equality.
  Both operators are therefore m-accretive on $\Lp[1](X)$ (see \cite[Chap. XVII A]{DautrayLions}) and since the operator 
  $\Sigma_t - K - \mst\Id$ is bounded, also the whole operator 
  $$
    A + \Sigma_t - K - \mst\Id = T - \mst\Id
  $$ 
  is m-accretive on $\Lp[1](X)$. It then follows that 
  for every $\lambda > 0$, $\Id + \lambda (T - \mst\Id)$ is bijective from $D(T)$ onto $\Lp[1](X)$ and the conclusion 
  of the theorem is obtained for $\lambda = 1/\mst$.
\end{proof}
The value $c$ in Thm. \ref{thm1} has the physical meaning of the average number of neutrons emitted (with any velocity) 
after a neutron with velocity $(\bomega, E)$ collides with a nucleus at point $\br\in V$. Condition (c) thus expresses 
the requirement that the system be \textit{subcritical} in order for a steady solution in presence of external sources 
to be achieved\footnote{Note that $c$ slightly above $1$ may be allowed in practice due to the leakage of neutrons out 
of the system boundaries, which is not included in def. \eqref{eq:c}.}. Dautray and Lions in 
\cite[\S2, Chap. XXI]{DautrayLions} also outline the proof based on the same ideas in general $\Lp(X)$ spaces for 
$1 < p < \infty$, where in particular the case $p = 2$ is important\footnote{even though it does not lead to a direct 
physical interpretation as the $\Lp[1](X)$ setting} as the Hilbert space structure of $\Lp[2](X)$ allows to use richer 
set of mathematical tools to formulate practical solution methods (like those based on Fourier transform or the finite 
element methods). However, we need to add to assumptions (a-c) another assumption
\begin{enumerate}
  \item[(d)] ~~~~~~~~~~~~$\displaystyle 0 \leq \intE[']{\Emin}{\Emax}{\intA[']{\kappa(\br,\bomega\cdot\bomega',E\sla E')}} \leq \sigma_t(\br,E) - \mst$
\end{enumerate}
which does not have as plausible physical explanation as the assumption (c) (it could be interpreted as that beam of 
neutrons with any given velocity $(\bomega, E)$ is always attenuated by the collision with nuclei at any given point 
$\br\in V$ \footnote{i.e. the absorption rate dominates the rate at which new neutrons are introduced into the beam 
after collisions induced by all possible neutrons impinging on that point}). In Proposition 6, Dautray and Lions also 
state the existence of unique solution in $\Lp[\infty](X)$ for $q\in \Lp[\infty](X)$; the proof in this case is based on
 the inversion of the transport operator along characteristics\footnote{The integral form of the BTE, mentioned already 
 in the introduction, is obtained in this way.} and uses condition (d) instead of (c). Note that the approach based on 
 the characteristic form of the BTE has also been used by Sanchez in \cite{Sanchez3}, who proved the existence result in
  the cross-section weighted space $\Lp[1]_{\sigma_t}(X)$, using general boundary conditions. This seems to be an 
  alternative physically natural functional setting due to the definition of the reaction rate, eq. \eqref{eq:rr}; 
  moreover, assumption (a) may be relaxed by allowing $\sigma_t = 0$ in arbitrarily large regions 
  (the \textit{void regions}).

\subsection{Multigroup approximation}\label{sec:MG}
The above results hold also for the multigroup form of the transport equation (\cite[Chap. XXI, \S1]{DautrayLions}),
which is obtained by dividing the interval of neutron energies as follows\footnote{Note that the energy intervals
(groups) are numbered in a descending order, i.e. a group with larger index contains lower energies than a group with
lesser index.}:
$$
  [\Emin,\Emax] = [E^{G-{1/2}},E^{G+{1/2}}]\cup \ldots \cup [E^{g-1/2}, E^{g+1/2}] \cup \ldots \cup [E^{1/2},E^{3/2}],
$$ integrating \eqrefs{eq:bte1}{eq:bte3} over each energy group $[E^{g-1/2}, E^{g+1/2}]$ and defining appropriate
group-integrated quantities corresponding to the average group energy\footnote{A counting measure on the set
$\{E^G,\ldots,E^1\}$ is then used in the proofs instead of the continuous Lebesgue measure $\d{E}$ in the energy
integrals.}\footnote{We will always assume that the group parameters appearing in the equations (like macroscopic
cross-sections) are given and only remark that their actual calculation is an essential and highly non-trivial task.}
$E^g$. This conventional procedure leads to the following set of $G$ coupled neutron transport equations
\begin{equation}\label{eq:mg}
  \begin{gathered}
	\left\{
	  \begin{aligned}
      &T^\text{G}\{\psi^g(\br,\bomega)\} = \{q^g(\br,\bomega)\},\\
      &\Dom{T^\text{G}} = \bigl\{\{\psi^g\}\in \bigl[\Hp[1](X)\bigr]^G,\ \psi^g\vert_{\pX[-]} = 0,\ g = 1,\ldots,G\bigr\}
    \end{aligned}
  \right.\\[.2em]
    T^\text{G}\{\psi^g(\br,\bomega)\} = \left\{\left(A + \Sigma_r^g\right)\psi^g(\br,\bomega) - \summ{g'=1,g'\neq g}{G} K^{gg'}\psi^{g'}(\br,\bomega)\right\}\\
    \Sigma_r^g = \sigma_t^g(\br) - \intA[']{\kappa^{g\sla g}(\br,\bomega\cdot\bomega')},\quad K^{gg'} = \intA[']{\kappa^{g\sla g'}(\br,\bomega\cdot\bomega')}.
  \end{gathered}
\end{equation}
Note that it is customary to move the \textit{self-scattering} (diagonal) part of the transfer operator to the collision 
operator. Since the reactions in which energetic distribution of both the incoming and outgoing neutrons lies within the 
same group are included in both $\sigma_t$ and $\kappa$ (see Sec. \ref{sec:definitions}), this transformation makes 
$\Sigma_r^g\psi^g$ represent the actual rate of neutron removal from the group, while $K^{gg'}\psi^{g'}$ the rate of 
neutron addition to that group.

We conclude this chapter by recalling the traditional way of iterative solution of the multigroup system, known as 
\textit{source iteration}. Mathematically, it is a Gauss-Seidel iteration for the operator equation \eqref{eq:mg}:
$$
  \left(A + \Sigma_r^g - \summ{g'\leq g-1}{} K^{gg'}\right)\psi^g_{i+1} = \summ{g' \geq g+1}{} K^{gg'}\psi^{g'}_{i} + q^g,\quad g = 1,\ldots,G,\ \ i = 0,1,\ldots
$$
Now a mono-energetic transport problem in group $g$ has to be solved in each iteration, and only the advection operator 
$A$ spoils the self-adjointness of the formulation. 
%The following chapter is devoted to a particular self-adjoint approximation of the mono-energetic problem.


