\ifpdf
	\graphicspath{{4/pic/PNG/}{4/pic/PDF/}{4/pic/}}
\else
	\graphicspath{{4/pic/EPS/}{4/pic/}}
\fi

\chapter{The $\MCPN$ approximation}\label{chap:mcpn}

In this chapter, we start constructing an approximation to the monoenergetic neutron transport equation $T\psi - q = 0$
of the form
\begin{equation}\label{eq:MC-bte}
    \bomega\cdot\nabla\psi(\br,\bomega) + \sigma_t(\br)\psi(\br,\bomega) - \intA[']{\left[\sigma_s(\br,\bomega\cdot\bomega') + 
    \frac{\nu(\br)\sigma_f(\br)}{4\pi}\right]\psi(\br,\bomega')} - q(\br,\bomega) = 0.
\end{equation}
We will focus on the angular approximation here and assume that $\psi(\cdot,\bomega)\in\Lp[2](\Sphere)$
% in the discussion below (even though the final derivation in Sec. \ref{sec:mcpn} actually requires only $\Lp[1](\Sphere)$ integrability)
. 
Let us start by recalling the well established spherical harmonics method.

\section{$\PN$ approximation}\label{sec:SPH}
Being a square integrable function on $\Lp[2](\Sphere)$, $\psi(\cdot,\bomega)$ may be represented by a generalized
Fourier series expansion in terms of a complete orthonormal basis of this space. The $\PN$ method uses the basis of
tesseral spherical harmonic functions $\{\Y{n}{m}\}$ of degree $n\geq 0$ and order $-n \leq m \leq n$ (shortly
\textit{tesseral harmonics}). A finite method is obtained by the classical Galerkin approximation of the BTE in a
subspace $\Lp[2]_N(\Sphere)\subset\Lp[2](\Sphere)$ spanned by $\{\Y{n}{m}\}$,  $n \leq N$ \footnote{For subtle reasons
that could be found in the literature (\cite[Sec. 10.3.2]{Davison}, \cite[Sec. 9.6]{Stacey1}), only odd-$N$ $\PN$
approximations are usually used.} -- i.e. it consists of the expansion
\begin{equation}\label{eq:exp}
  \psi(\br,\bomega) \approx \suma[n]{0}{N}\suma[m]{-n}{n}\angmom{n}{m}(\br)\Y{n}{m}(\bomega)
\end{equation}
and orthogonal projection of the residual $T\psi - q$ onto $\Lp[2]_N(\Sphere)$ (similarly, the exact boundary conditions 
are projected onto a subspace of $\Lp[2](\pX[-])$ spanned by the complete set of odd-degree tesseral harmonics; 
the resulting conditions are known as the \textit{Marshak conditions}\footnote{We only make a remark that there is 
another form of approximate boundary conditions known as the Mark conditions. The relative merit of one over the other 
is not completely resolved so both are widely used. We choose the former as they are consistent with the Galerkin 
derivation of the $\PN$ equations.}).

The choice of the tesseral harmonics basis facilitates the realization of the above procedure. Besides the
orthonormality property, they obey a three-term recurrence rule that is used advantageously in expanding the advection
part
% (however, a closure relation such as that partial derivatives of $\angmom{N+1}{m}(\br)$ vanish for all relevant $m$
% becomes necessary)
and they also naturally appear in the \textit{addition theorem} for Legendre polynomials $P_n$:
\begin{equation}\label{eq:addition}
  P_n(\bomega\cdot\bomega') = \frac{4\pi}{2n+1}\suma[m]{-n}{n}\Y{n}{m}(\bomega)\Y{n}{m}(\bomega').
\end{equation}
This equality is used for simplifying the scattering part of the transfer operator, after its expansion in terms of 
Legendre polynomials (which form a complete orthonormal set in $\Lp[2]([-1,1])$):
\begin{equation}\label{eq:exp2}
  \sigma_s(\br,\bomega\cdot\bomega') \approx \suma[k]{0}{K}
      \frac{2k+1}{4\pi}\sigma_{sk}(\br)P_k(\bomega'\cdot\bomega)
\end{equation}
(where $K \leq N$ is usually called \textit{degree of scattering anisotropy}).
%where
%$\Sigma_{sk}(\br) = 2\pi \muint[_0]{\P{k}(\mu_0)\Sigma_s(\br, \mu_0)}$
%are the coefficients of the expansion of the scattering cross-section 
% ROTATIONAL INVARIANCE OF SUBSP.
Also, we directly obtain the important physical quantities as $\phi = \angmom{0}{0}$, $\mathbf{J} = \bigl[\angmom{1}{0},\angmom{1}{1},\angmom{1}{-1}\bigr]^T$.

\section{Other spherical harmonic approximations}
This standard procedure is described in various books on reactor physics and used in several computer codes
(\cite{MARC,Capilla,vanCriekingen1}). It results however in a complicated set of strongly coupled partial differential
equations without any clear structure that could be used for its simplification. This motivates the search for an
alternative set of expansion functions for \eqref{eq:exp} that would share with $\Y{n}{m}$ the important properties like
those mentioned above but have a more regular structure.

\subsection{Surface and solid spherical harmonics}
One possible such set arises by studying the linear combinations of tesseral harmonics of fixed degree. We will simplify
the notation by setting $\mu = \cost$, replace $\bomega$ with $(\mu, \azimuthal)$ where necessary and also use the
following relations between the velocity and direction vectors in Cartesian coordinates:
$$
  \bv = [v_x, v_y, v_z]^T = v\bomega,\quad \bomega = [\Omega_x, \Omega_y, \Omega_z]^T = \frac{\bv}{v}.
$$

\begin{definition}\label{defn:SSH}
  A general linear combination of the $2n + 1$ tesseral harmonics of degree $n$ is called a \textit{surface spherical 
  harmonic of degree $n$} and can be written as
  \begin{equation}\label{eq:exp3}
    Y_n(\mu,\azimuthal) = A_0 P_n(\mu) + \suma[m]{1}{n}\left[ A_m \cos(m\azimuthal)\P{n}{m}(\mu) + B_m \sin(m\azimuthal)\P{n}{m}(\mu)\right]
  \end{equation}
  where $\P{n}{m}$ are the \textit{associated Legendre polynomials} \cite[Chap. VI]{Byerly}. 
\end{definition}
Note that there is a one-to-one relationship between the coefficients in eq. \eqref{eq:exp} for fixed $n$ and the 
coefficients in \eqref{eq:exp3}.

Multiplying by $v^n$, we obtain the (regular) \textit{solid spherical harmonic}\footnote{Regular solid spherical
harmonics are one class of solutions of the Laplace equation $\lap Y = 0$ in spherical coordinates
$(v,\azimuthal,\polar)$ which vanish as $v\to 0$. The other are the irregular solid spherical harmonics, which have
singularity of the form $v^{-n-1}$ at the origin (\cite[Chap. VI]{Byerly}).}. Paper \cite{Ackroyd1} appears to be the
first where solid spherical harmonics were practically used to solve the neutron transport equation. Under the
assumption of isotropic scattering ($\sigma_{sk} = 0$ for $k \geq 1$ in \eqref{eq:exp2}) he derived for a homogeneous
region a set of coupled diffusion equations (called SH\PN) without any other requirement (unlike the classical
derivation of the $\SPN$ equations, which required certain assumptions about dimensionality or material properties, cf.
\cite{Larsen1}, \cite{Pomraning1}), together with heuristic boundary and interface conditions. As was shown in the
paper, the SH\PN~ equations reduced by simple substitutions to the set of $\SPN$ equations originally formulated by
Gelbard and it is interesting to note that this actually showed that the latter are within an isotropically scattering
homogeneous medium equivalent to the full solid harmonics expansion -- the same result for surface spherical harmonics
has been independently proven in \cite{Coppa1} and revisited recently (\cite{Coppa2,McClarren1}). Unfortunately, the
general treatment in \cite{Ackroyd1} is very technically involved and in author's opinion quite difficult to follow --
this may be the reason why the idea has not been picked up and possible research directions outlined in the paper's
conclusion not pursued.

\subsection{Cartesian tensors}\label{sec:tensors}
In order to develop a conceptually simpler and arguably also more useful approach, we need to recall some basic facts
about Cartesian tensors. We will use the convention that Greek subscripts represent axes of a Cartesian coordinate
system with unit vectors $\be_x$, $\be_y$, $\be_z$. We will also use the usual Einstein's summation convention which
implies summation over any index which appears twice in an indexed expression, for instance $$
  \AAA{\alpha\beta\gamma}\BBB{\beta\gamma} =
  \suma[\beta]{1}{3}\suma[\gamma]{1}{3}\AAA{\alpha\beta\gamma}\BBB{\beta\gamma} = \CCC{\alpha}.
$$

\begin{definition}
  An $n$-dimensional array $\AA$ of $3^n$ components $\AAA{\alpha_1\ldots\alpha_n}$ is called \textit{Cartesian tensor 
  of rank $n$} if it transforms as
  \begin{equation}\label{eq:tenstran}
    A^{(n)'}_{\alpha_1\ldots\alpha_n} = g_{\alpha_1\beta_1}\cdots g_{\alpha_n\beta_n}\AAA{\beta_1\ldots\beta_n}
  \end{equation}
  under the change of coordinate system $Oxyz \to Ox'y'z'$ by the action of an orthogonal matrix $\mathbf{G} = [g_{\alpha\beta}]$:
  $$
    \be_\alpha' = g_{\alpha\beta}\be_\beta.
  $$
\end{definition}
As we will only use the Cartesian tensors, we will henceforth omit the word Cartesian. Word ``tensor'' will also be used
 for a general \textit{tensor field}, components of which are functions -- like $\AA(\br)$. The transformation has then 
 the following form:
$$
  A^{(n)'}_{\alpha_1\ldots\alpha_n}(\br) = \AAA{\alpha_1\ldots\alpha_n}(\mathbf{G}^{-1}\br) = g_{\alpha_1\beta_1}\cdots g_{\alpha_n\beta_n}\AAA{\beta_1\ldots\beta_n}(\br).
$$
We will denote by $\eye$ the identity rank-2 tensor (matrix) and by $ \mathbb{O}^{(n)}$ the zero rank-n tensor.

Addition and subtraction of two tensors of same rank and multiplication of a tensor by a scalar are done component-wise. 
Multiplication of two tensors is defined as follows:
\begin{definition}
  Components of tensor $\CC[n+m] = \AA\otimes\BB[m]$ are given by
  $$
    \CCC[n+m]{\alpha_1\ldots\alpha_n\beta_1\ldots\beta_m} = \AAA{\alpha_1\ldots\alpha_n}\BBB{\beta_1\ldots\beta_m}.
  $$
\end{definition}

\begin{definition}
  $m$-th power of tensor $\AA$ is a tensor of rank $nm$ defined as
  $$
    \CC[nm] = \AA\otimes\AA\otimes\cdots\otimes\AA\quad \mbox{($m$-times)}
  $$
\end{definition}
We will mainly use powers of vectors and also consider the gradient operator as a vector
$$
  \nabla = \left[\pd{}{x}, \pd{}{y}, \pd{}{z}\right]^T
$$
(hence $\Delta = \lap = \nabla\otimes\nabla$). Also, to keep usual notation, $\nabla\AA \equiv \nabla\otimes\AA$.

\begin{definition}
  For $1\leq m \leq n$, the $m$-fold contraction of tensors $\AA$ and $\BB$ is a rank-($2n-2m$) tensor 
  $\CC[2n-2m] = \AA\underset{m}{\cdot}\BB$ with components
  $$
    \CCC[2n-2m]{\alpha_{1}\ldots\alpha_{n-m}\beta_1\ldots\beta_{n-m}} = 
    \AAA{\alpha_1\ldots\alpha_{n-m}\gamma_{n-m+1}\ldots\gamma_n}\BBB{\gamma_n\ldots\gamma_{n-m+1}\beta_{n-m}\ldots\beta_1}.
  $$
\end{definition}
Specially for $m = 1$, we get the standard inner product. 
In the case when $n = m$, we obtain the scalar $\AAA{\gamma_1\ldots\gamma_n}\BBB{\gamma_n\ldots\gamma_1}$ and 
suppress the index under the $\cdot$ sign to simplify the writing (it should be clear from the two operands and their 
rank that a total contraction over all their indices is intended). 

\begin{definition}
  For $m \leq \lfloor n/2 \rfloor$ (the integer part of $n/2$), the $m$-fold contraction of tensor $\AA$ (contraction 
  in $m$ index pairs) is a rank-($n-2m$) tensor with components
  $$
    \BBB[n-2m]{\alpha_{2m+1}\ldots\alpha_n} = \AAA{\alpha_1\alpha_1\ldots\alpha_m\alpha_m\alpha_{2m+1}\ldots\alpha_n}.
  $$
\end{definition}
When contracting in only 1 index pair, the $1$-fold contraction is simply called a contraction. The result has a special 
name:
\begin{definition}
  Contraction of tensor $\AA$ in one index pair is called \textit{trace} of $\AA$ in that pair:
  $$
    \trace \AA = \AAA[n-2]{\alpha_1\alpha_1\alpha_3\ldots\alpha_n}.
  $$ 
\end{definition}
If the trace of $\AA$ in one index pair vanishes, the tensor is said to be traceless in that pair. Obviously then, if 
the tensor does not depend on the order of indices, it is traceless in all index pairs.
\begin{definition}
  Tensor $\AA$ is \textit{totally symmetric} if it is invariant under any permutation $\perm{\alpha_1\ldots\alpha_n}$ 
  of its indices:
  $$
    \AAA{\alpha_1\ldots\alpha_n} = \AAA{\perm{\alpha_1\ldots\alpha_n}}.
  $$
\end{definition}

\begin{definition}
  Totally symmetric tensor whose trace in any index pair vanishes is called \textit{totally symmetric traceless tensor} 
  and its trace vanishes in all index pairs. We will denote totally symmetric traceless tensors as TST tensors.
\end{definition}

We can make any tensor of rank $n$ totally symmetric by applying the symmetrization operator $\Sym{\cdot}$:
\begin{definition}
  $\widetilde{\mathbb{A}}^{(n)} = \Sym\AA$ is a totally symmetric tensor with components
  $$
    \widetilde A^{(n)}_{\alpha_1\ldots\alpha_n} = \frac{1}{n!}\ \; \sum_{\mathclap{\perm{\alpha_1\ldots\alpha_n}}}\ \;\AAA{\alpha_1\ldots\alpha_n}.
  $$
\end{definition}  
As we will see shortly, there also exists an operator which makes a totally symmetric tensor totally symmetric and traceless.

\subsection{Maxwell-Cartesian spherical harmonics}\label{sec:MCSH}
It is well known that any solid spherical harmonic of degree $n$, when expressed in Cartesian coordinates: $S_n(\bv) =
S_n(v_x, v_y, v_z) = v^n Y_n(\bv/v)$, is a harmonic polynomial, homogeneous of degree $n$ (\cite[Art. 110]{Byerly}),
i.e.
$$
  \lapv S_n(\bv) = 0,\quad S_n(\lambda\bv) = \lambda^n S_n(\bv)
$$ (in fact, there is an alternative definition of solid spherical harmonic as a function satisfying these two
properties and of surface spherical harmonic as a restriction of such function to the unit sphere). The space of such
polynomials has dimension $2n + 1$\footnote{Homogeneous polynomial of degree $n$ has $\frac{(n+1)(n+2)}{2}$
coefficients; since $\lapv S_n(\bv)$ is a homogeneous polynomial of degree $n-2$ with $\frac{n(n-1)}{2}$ coefficients
expressed in terms of those of $S_n$, the condition $\lapv S_n(\bv) = 0$ reduces the number of independent coefficients
of $S_n$ to $\frac{(n+1)(n+2)}{2} - \frac{n(n-1)}{2} = 2n + 1$.}\label{pg:indepc} and is isomorphic to the space of TST
tensors of rank $n$ (\cite{Sam}).
% SEE AXLER HFT - L2(S_2) = direct sum of H_n(S_2)
In particular, when the components of such a tensor do not depend on $v_x$, $v_y$, $v_z$, (or $\Omega_x$, $\Omega_y$,
$\Omega_z$) they comprise coefficients of the solid (or surface) spherical harmonic and so also (in the
$\Sphere$-restricted case) the coefficients $\angmom{n}{m}(\br,E)$ in the expansion \eqref{eq:exp} (when transformed
back to spherical coordinates). Hence, by solving the set of tensorial equations with these tensors of ranks $n \leq N$
as unknowns, we could accomplish essentially the same as by solving the ordinary $\PN$ equations, but we could
additionally use the TST tensor character of the equations.

To construct this set, we will build on the ideas presented by Johnston (\cite{Johnston1}) and Applequist
(\cite{Applequist1}). Johnston arrived at a ``far more symmetric and compact'' (\cite[p. 1455]{Johnston1}) form of the
angularly discretized general Boltzmann-Vlasov equation by expanding the flow function in terms of
\textit{Maxwell-Cartesian spherical harmonic tensors} rather than the usual surface spherical harmonics. The term
``Maxwell-Cartesian spherical harmonic tensor'' has been coined by Applequist in \cite{Applequist2}, who in his older
paper \cite{Applequist1} showed a systematic way of obtaining a special TST tensor of any rank $n$ whose components are
spherical harmonics of degree $n$ in Cartesian frame of reference as defined by Maxwell in \cite[p. 160]{Maxwell}.
Specifically, in present notation, he showed that Maxwell's spherical harmonics based on Cartesian axes can be obtained
(up to a normalization constant) as components of $$
  \detr \bv^n \quad \mbox{(or $\detr \bomega^n$)}
$$ where $\detr$ is the so-called \textit{detracer operator} which projects a general totally symmetric tensor of rank
$n$ into the space of totally symmetric and traceless tensors of rank $n$ (see the note on p. 4311 of
\cite{Applequist1}). The result of applying the detracer operator on a totally symmetric tensor $\AA$ is a TST tensor
with components:
\begin{equation}\label{eq:detr}
  \detr \AAA{\alpha_1\ldots\alpha_n} = \suma[m]{0}{\lfloor n/2 \rfloor} \frac{(-1)^m(2n-2m-1)!! }{(2n-1)!!}\quad\sum_{\mathclap{\{\perm{\alpha_1\ldots\alpha_n}\}_{\text{d}}}}\quad\kron{\alpha_1}{\alpha_2}\cdots\kron{\alpha_{2m-1}}{\alpha_{2m}}\AAA{\beta_1\beta_1\ldots\beta_m\beta_m\alpha_{2m+1}\ldots\alpha_n},
\end{equation}
where $\{\perm{\alpha_1\ldots\alpha_n}\}_{\text{d}}$ denotes the set of all permutations that give distinct terms in 
the last sum (considering their total symmetry) and $(2n-1)!! = (2n-1)(2n-3)\cdots 3\cdot 1$ with $(-1)!! = 1$. Various 
properties and different explicit forms of the detracer operator are given in \cite[Sec. 5]{Applequist1}
%or, in a somewhat more transparent way, in \cite[Eqns. (36), (37)]{Vreoiju}
.

Being general surface spherical harmonics according to Def. \ref{defn:SSH}, Maxwell's surface spherical harmonics are
linear combination of tesseral harmonics and hence share many of the properties of the latter. An extensive presentation
of these properties is given in \cite{Applequist1,Applequist2} and we will recall some of them in further sections. We
will use the following form of Maxwell-Cartesian surface spherical harmonics of degree $n$ as components of rank-$n$ TST
tensors (called sometimes \textit{Maxwell-Cartesian tensors}):
\begin{equation}\label{eq:MCSH}
  \PP(\bomega) = \detr \bomega^n
\end{equation}
which differ from those used by Applequist in that they have coefficient $1$ at their first term but which coincide 
with those used by Johnston. 
 
We note that projection of $\PP$ along, say, $z$-axis (or any other because of the symmetry) yields (up to a
normalization factor) the Legendre polynomials:
\begin{equation}\label{eq:leg}
  \PP(\bomega)\contr\be_z^n = P^{(n)}_{\alpha_1\ldots\alpha_n}\kron{3}{\alpha_1}\cdots\kron{3}{\alpha_n} = P^{(n)}_{33\ldots3} = \frac{n!}{(2n-1)!!}P_n(\mu) \equiv C_n P_n(\mu)
\end{equation}
and hence components of $\PP$ could be explicitly obtained by extending the well-known formulas for Legendre 
polynomials, for instance
\begin{equation}\label{eq:conn1}
  \PP(\bomega) = \bomega^n - \frac{n(n-2)}{2(2n-1)}\Sym{\eye\otimes\bomega^{n-2}} + \frac{n(n-1)(n-2)(n-3)}{8(2n-1)(2n-3)}\Sym{\eye^2\otimes\bomega^{n-4}}\mp\ldots
\end{equation}
(\cite[Chap. VI]{Byerly}, \cite{Johnston1})
% or
%\begin{equation}\label{eq:conn2}
%  \PP[0](\bomega) = 1,\ \PP[n+1](\bomega) = \left[\bomega\otimes\PP[n](\bomega) - \tfrac{n^2}{4n^2-1}\eye\otimes\PP[n-1](\bomega)\right]_{\mbox{\scriptsize sym}},\ n = 1,2,\ldots.
%\end{equation}
First few Maxwell-Cartesian surface harmonics generated by this formula are shown in the table below together with the 
corresponding tesseral harmonics and Legendre polynomials.
\begin{center}
	\begin{tabular}{|A|B|C|D|}
	  \hline
		n & $\YY(\bomega)\propto$ & $\PP(\bomega)$ & $C_n P_n(\mu)$ \nl
		0 & 1 & 1 & 1\nl
		1 & $\Omega_x$, $\Omega_y, \Omega_z$ & $\Omega_x$, $\Omega_y$, $\Omega_z$ & $\mu$ \nl
		2 & $-\Omega_x^2-\Omega_y^2 + 2\Omega_z^2$, 
		    $\Omega_y\Omega_z$, 
		    $\Omega_z\Omega_x$,
		    $\Omega_x\Omega_y$,
		    $\Omega_x^2 - \Omega_y^2$ & $\Omega_x^2-\tfrac13$, 
		                                $\Omega_y^2-\tfrac13$, 
		                                $\Omega_z^2-\tfrac13$, 
		                                $\Omega_x\Omega_y$, 
		                                $\Omega_x\Omega_z$, 
		                                $\Omega_y\Omega_z$ & $\mu^2 - \tfrac13$ \nl
	\end{tabular}
	\captionof{table}{Tesseral and Maxwell-Cartesian surface spherical harmonics and Legendre polynomials up to degree 
	$n = 2$.}
	\label{tab:harmonics}
\end{center}
 The symmetric nature of the Maxwell-Cartesian surface harmonics as opposed to the tesseral harmonics can be seen from 
 the third row in the table. We also note that since there are $\frac{(n+1)(n+2)}{2}$ distinct components in a general 
 totally symmetric tensor of rank $n$, the number of Maxwell-Cartesian harmonics is greater than the number of tesseral 
 harmonics of the same degree, which is ($2n + 1$) for $n \geq 2$. However, due to the $\frac{n(n-1)}{2}$ conditions 
 arising from the tracelessness property, there are just $2n + 1$ independent components in $\PP$ (see also the footnote 
 on pg. \pageref{pg:indepc}). This also indicates the fact that unlike the tesseral harmonics with same degree but
 different orders, components of $\PP$ are not all $\Lp[2](\Sphere)$-orthogonal. Nevertheless, Maxwell-Cartesian surface
  tensors of different degrees are orthogonal in the following sense:
\begin{equation}\label{eq:OG}
  \intA{\PP[n](\bomega) \otimes \PP[m](\bomega)} = \mathbb{O}^{m+n},\quad n\neq m.
\end{equation}

The Maxwell-Cartesian tensors have been actively used for solving various electro-magnetics and quantum-mechanical
problems (see the references in \cite{Applequist2}), but there is apparently only one attempt to bring them to the field
of neutron transport. In a recent article \cite{Coppa3}, Coppa re-derived the Maxwell-Cartesian surface spherical
harmonic tensors from their analogy with Legendre polynomials (see above) and the requirement that they satisfy the
addition theorem of the form \eqref{eq:addition}. He then multiplied the BTE with these tensors and integrated over the
sphere to obtain a system of tensorial equations, each of which being itself a symmetric set of PDE's determining the
components of the flux moment tensor defined by
\begin{equation*}
  \psi^{(n)}(\br) = \intA{\psi(\br,\bomega)\PP[n](\bomega)}.\tag{\cite[Eqn. (42)]{Coppa3}}
\end{equation*}
Note that this definition implies that the flux moment tensors $\psi^{(n)}$ are totally symmetric and traceless. 
However, only the symmetry property appears to be recognized in \cite{Coppa3}. Using rather complicated tensor relations 
(which are not fully derived in the paper and only conjectured to hold for general $n$), the system is constructed again 
by analogy with the procedure leading to the $\PN$ system. This could be misleading, however. In particular, there is 
no indication of why the moments obtained by solving the set could be interpreted as coefficients in an expansion of 
angular flux. Note that in the usual $\PN$ method where the angular flux is expanded into a series of tesseral 
harmonics, the key ingredient that allows us to deduce 
%that allow us to deduce from the residual projection condition
%$$
%  \suma[n']{0}{\infty}\suma[m']{-n'}{n'}\Y{n'}{m'}(\bomega)\intA{\left[T\psi(\br,\bomega) - q(\br,\bomega)\right] \Y{n}{m}(\bomega)} = 0
%$$
%that 
%$$
%  \intA{\left[T\psi(\br,\bomega) - q(\br,\bomega)\right] \Y{n}{m}(\bomega)} = 0\quad \forall n,m
%$$
%(which leads to the moment equations analogous to \cite[Eqn. (48)]{Coppa3}) 
the system of equations analogous to \cite[Eqn. (48)]{Coppa3}) for the coefficients in the expansion is orthogonality 
(and hence linear independence) of all $\Y{n}{m}$. We do not have this property in the set of Maxwell-Cartesian 
harmonics (which seems to be neglected in \cite{Coppa3}). This issue however, can be elegantly resolved by taking into 
account the tracelessness property of the Maxwell-Cartesian tensors. We will demonstrate it by combining the results of 
Johnston (\cite{Johnston1}) and Applequist (\cite{Applequist1}). Surprisingly, this more fundamental and systematic 
approach leads to a set of equations equivalent with \cite[Eqn. (42)]{Coppa3} (explanation of why this is so deserves 
further investigation). We will also see a clear connection to the 1D $\PN$ equations (based on the expansion of angular 
flux in Legendre polynomials) which could be expected from the formal equivalence of the Maxwell-Cartesian tensors and 
Legendre polynomials mentioned above (\eqref{eq:conn1}, 
%\eqref{eq:conn2}, 
Tab. \ref{tab:harmonics})\footnote{In \cite{Coppa3}, only the equivalence with $\PN[1]$ and $\SPN[2]$ equations have 
been shown by rather laborious manipulations with the equations for $l = 0,1$ and $l = 0,1,2$, respectively.}.

\section{Derivation of the $\MCPN$ approximation}\label{sec:mcpn}
\subsection{First attempts}
Using the results of \cite[Art. 114]{Byerly}, the expansion of angular neutron flux in terms of surface spherical 
harmonics could be written as
\begin{equation}\label{eq:MC-exp1}
  \psi(\br,\bomega) = \suma[n]{0}{\infty}\frac{2n+1}{4\pi}\intA{\psi(\br,\bomega')P_n(\bomega\cdot\bomega')}.
\end{equation}
As shown in \cite[Sec. 7, Corollary II]{Applequist1}, Maxwell-Cartesian tensors satisfy the following form of 
addition theorem:
\begin{equation}\label{eq:MC-addition}
  \PP(\bomega)\contr\PP(\bomega') = C_n P_n(\bomega\cdot\bomega')
\end{equation}
($C_n$ defined in \eqref{eq:leg}). Combining the two results, we obtain
\begin{equation}\label{eq:MC-exp2}
  \psi(\br,\bomega) = \suma[n]{0}{\infty}\frac{2n+1}{4\pi C_n}\left[\intA{\psi(\br,\bomega')\PP(\bomega')}\right]\contr\PP(\bomega).
\end{equation}
If we now define the \textit{$n$-th angular flux moment tensor} by
\begin{equation}\label{eq:flux-moment}
  \fmom(\br) := \frac{2n+1}{4\pi C_n}\intA{\psi(\br,\bomega)\PP(\bomega)}
\end{equation}
equation \eqref{eq:MC-exp2} becomes
\begin{equation}\label{eq:MC-exp3}
  \psi(\br,\bomega) = \suma[n]{0}{\infty}\fmom(\br)\contr\PP(\bomega).
\end{equation}
Note that we have, like before, $\phi = \fmom[0]$, $\mathbf{J} = \bigl[\fmom[1]_x,\fmom[1]_y,\fmom[1]_z\bigr]^T$.

Although equation \eqref{eq:MC-exp3} with \eqref{eq:flux-moment} looks like a (generalized) Fourier series, we should 
keep in mind that
$$
  \intA{\PP[\ell](\bomega)\otimes\suma[n]{0}{\infty}\fmom(\br)\contr\PP(\bomega)} \neq \fmom[\ell]
$$
in general and regard \eqref{eq:flux-moment} as definition.

We analogously expand also the volumetric source term, leading to 
\begin{equation}\label{eq:MC-exp_src}
  q(\br,\bomega) = \suma[n]{0}{\infty}\qmom(\br)\contr\PP(\bomega)
\end{equation}
with
\begin{equation*}
  \qmom(\br) := \frac{2n+1}{4\pi C_n}\intA{q(\br,\bomega)\PP(\bomega)}.
\end{equation*}

To find the relations that must be satisfied by the angular expansion moments $\fmom(\br)$ in order for
\eqref{eq:MC-exp3} (or equivalently \eqref{eq:MC-exp1}) to be the solution of the BTE \eqref{eq:MC-bte}, we insert the
expansion \eqref{eq:MC-exp3} into \eqref{eq:MC-bte} (with source term represented according to \eqref{eq:MC-exp_src}).
Let us first look at the transfer part.
Applying eq. \eqref{eq:MC-addition} in the expansion \eqref{eq:exp2}, we immediately simplify the scattering
term\footnote{As in \eqref{eq:MC-exp1}, the expansion is taken exactly with $K\to\infty$ so that we can simply write
equalities between the $\Lp[2](\bomega)$ functions and their generalized Fourier series and use a single index $n =
0,1,\ldots$ in both expansions. In a practical calculation the expansion in \eqref{eq:MC-exp1} is truncated at finite
degree $N$ and the scattering cross-section moments $\sigma_{sk}$ are typically available only up to a certain
anisotropy degree $K \leq N$; in that case, however, we may always set $\sigma_{sn} = \sigma_{sk}$ for $n \leq K$ and
$\sigma_{sn} = 0$ otherwise.}:
\begin{equation*}%\label{eq:scat}
  \begin{aligned}
  \intA[']{\sigma_s(\br,\bomega\cdot\bomega')\psi(\br,\bomega')} &= 
    \suma[n]{0}{\infty}
    \frac{2n+1}{4\pi}\sigma_{sn}(\br)\intA[']{P_n(\bomega'\cdot\bomega)\psi(\br,\bomega')}\\
    & = \suma[n]{0}{\infty}
    \frac{2n+1}{4\pi C_n}\sigma_{sn}(\br)\left[\intA[']{\PP(\bomega')\psi(\br,\bomega')}\right]\contr\PP(\bomega)\\
    & = \suma[n]{0}{\infty}
    \sigma_{sn}(\br)\fmom(\br)\contr\PP(\bomega).
    \end{aligned}
\end{equation*}
Because of the orthogonality relation \eqref{eq:OG}, the fission part will have the following form:
\begin{equation}%\label{eq:scat}
  \intA[']{\frac{\nu(\br)\sigma_f(\br)}{4\pi}\psi(\br,\bomega')} = 
    \suma[n]{0}{\infty}
    \frac{\nu(\br)\sigma_f(\br)}{4\pi}\fmom(\br)\contr\intA[']{\PP(\bomega')} = \nu(\br)\sigma_f(\br)\phi(\br).
\end{equation}
Therefore, by inserting \eqref{eq:MC-exp3} into \eqref{eq:MC-bte} and using these results we obtain
\begin{equation}\label{eq:MC-bte2}
  \suma[n]{0}{\infty}\left[\bomega\cdot\nabla\fmom + \sigma_t\fmom 
  - \sigma_{sn}\fmom[n] - \kron{n}{0}\nu\sigma_f\phi - \qmom\right]\contr\PP(\bomega) = 0
\end{equation}
where each term in the brackets is dependent only on $\br$ (which is omitted for brevity).

Formulation of the moment equations for $\fmom(\br)$ is now hampered by linear dependence among certain functions in
each $\PP(\bomega)$ (so that we cannot deduce from \eqref{eq:MC-bte2} that for each $n$, all components of the
expression in brackets must vanish), as well as by the advection term which still contains $\bomega$. Both issues may be
however overcome by the so-called \textit{detracer exchange theorem} (\cite[Sec. 5.2]{Applequist2}).
\begin{theorem}\label{thm:detex}
  If $\AA$ and $\BB$ are totally symmetric tensors of rank $n$, then
  $$
    \AA\contr\detr \BB = \BB\contr\detr \AA.
  $$
\end{theorem}
\begin{proof}
  The theorem easily follows from the definitions of the detracer operator \eqref{eq:detr} and tensor contraction.
\end{proof}
Since $\fmom(\br)$ is by definition totally symmetric and traceless and $\bomega^{n}$ totally symmetric, using Theorem 
\ref{thm:detex} with $\AA \equiv \fmom(\br)$ and $\BB \equiv \bomega^{n}$ and the definition \eqref{eq:MCSH} of 
Maxwell-Cartesian tensors shows that\footnote{Recall that the detracer is a projection operator into the space of 
TST tensors and thus leaves the already TST tensor unchanged.} 
$$
  \fmom(\br)\contr\PP(\bomega) = \fmom(\br)\contr\bomega^n
$$
and the expansion \eqref{eq:MC-exp3} is thus equivalent to a power series in $\bomega$:
$$
  \psi(\br,\bomega) = \suma[n]{0}{\infty}\fmom(\br)\contr\bomega^n
$$
(similarly for \eqref{eq:MC-exp_src}). The advection term therefore simplifies as follows:
$$
  (\bomega\cdot\nabla)\fmom\contr\bomega^n = \left.
  \pw{\Omega_\beta\gradcomp\beta\fmom_{\alpha_0,\ldots,\alpha_{n-1}}\Omega_{\alpha_{n-1}}\cdots\Omega_{\alpha_0}}{if $n \geq 1$}
     {\Omega_\beta\gradcomp\beta\phi}{if $n = 0$}\right\}
   = \nabla\fmom\contr[n+1]\bomega^{n+1}
$$
for $n = 0,1,\ldots$ and by reindexing:
$$
  (\bomega\cdot\nabla)\fmom[n-1]\contr\bomega^{n-1} = \nabla\fmom[n-1]\contr[n]\bomega^{n}
$$
for $n = 1,2,\ldots$. We can therefore rewrite eq. \eqref{eq:MC-bte2} as
\begin{equation}\label{eq:MC-bte3}
  \suma[n]{0}{\infty}\left[\nabla\fmom[n-1] + \sigma_t\fmom
  - \sigma_{sn}\fmom - \kron{n}{0}\nu\sigma_f\phi - \qmom\right]\contr\bomega^n = 0
\end{equation}
with the term with negative rank discarded. Although eq. \eqref{eq:MC-bte3} closely resembles a vanishing linear 
combination of monomials
\footnote{Since we will not need to distinguish between the position vector $\br$ and the velocity vector $\bv$ here, 
we use the standard Cartesian coordinates $[x,y,z]$ for the latter as well.}
$$
  \bv^n = x^{n_1}y^{n_2}z^{n_3},\quad n_1 + n_2 + n_3 = n
$$
whose linear independence would imply that the expression in brackets must be a zero tensor for each $n$, by examining 
the not fully sensible equations that we would obtain in this way we might suspect that this is not the case. 
The reason is given in the following section.

\subsection{Sufficient conditions for deriving the moment equations}
When monomials $\bv^n$ of different degrees are restricted to the unit sphere, there appear nontrivial but vanishing
linear combinations of them (consider e.g. $\Omega_x^2 + \Omega_y^2 + \Omega_z^2 - 1$). However, when only the
components of $\bomega^n$ for a \textsl{single} $n$ are considered, they \textsl{are} linearly independent according to
the following lemma.
\begin{lemma}
  If $\AA$ is a totally symmetric tensor of rank $n$, independent of $\bomega$, and
  \begin{equation}\label{eq:lemma1}
    \AA\contr\bomega^n = 0,
  \end{equation}
  then $\AA = \mathbb{O}^{(n)}$.
\end{lemma}
\begin{proof}
  If $\AA$ is a totally symmetric tensor, then 
  $$
    \AA\contr\bomega^n = \frac{1}{v^n}\AA\contr\bv^n
  $$
  represents a linear combination of monomials of degree $n$,  with coefficients being the components of $\AA$. By 
  linear independence of these monomials we conclude that if the equality \eqref{eq:lemma1} holds, then all components 
  of $\AA$ must vanish.
\end{proof}

One possibility how to extend the above result to an arbitrary linear combination of direction tensors with different
ranks is to require the coefficient tensors to be not only totally symmetric, but also traceless.

\begin{theorem}\label{thm:MC-suffcond}
  If $\AA$ is a TST tensor independent of $\bomega$ for all $n = 0,1,\ldots$ and
  $$
    \suma[n]{0}{\infty}\AA\contr\bomega^n = 0,
  $$
  then $\AA = \mathbb{O}^{(n)}$ for all $n$.
\end{theorem}
\begin{proof}
  As discussed in Sec. \ref{sec:MCSH}, any TST tensor $\AA$ uniquely identifies a surface spherical harmonic of degree 
  $n$ by 
  $$
    Y_n(\bomega) = \AA\contr\bomega^n.
  $$
  Since surface spherical harmonics of different degrees are linearly independent,
  $$
    \suma[n]{0}{\infty} Y_n(\bomega) = 0\quad \Rightarrow \quad Y_n(\bomega) = 0 \ \ \forall{n}.
  $$
  The conclusion follows from the above lemma.
\end{proof}

\subsection{Final set of moment equations}
Using Thm. \ref{thm:MC-suffcond}, we see that we can obtain from \eqref{eq:MC-bte3} a set of moment equations if we
ensure that the expression in square brackets is a TST tensor for each $n$. Since $\fmom$ and $\qmom$ are TST by
definition, we only have to symmetrize and detrace the advection terms. We need to be careful, however, not to change
the original equation. This can be done by a clever rearranging of the terms in the sum.

First note that thanks to the symmetry of $\bomega^n$ we obviously have $$
  \nabla\fmom[n-1]\contr\bomega^n = \Sym{\nabla\fmom[n-1]}\contr\bomega^n.
$$ The symmetrized advection terms can be put into an equivalent TST form by $$
  \Sym{\nabla\fmom[n-1]} = \detr\Sym{\nabla\fmom[n-1]} - \left(\detr\Sym{\nabla\fmom[n-1]} -
  \Sym{\nabla\fmom[n-1]}\right).
$$ The result of the detracer operation on $\Sym{\nabla\fmom[n-1]}$ can be seen by using $$
  \AAA{\alpha_1\ldots\alpha_n} = \Sym{\gradcomp{\alpha_1}\fmom[n-1]_{\alpha_2\ldots\alpha_n}},\quad n \geq 2
$$ in eq. \eqref{eq:detr}. Taking into account the tracelessness of $\fmom[n-1]$, the only nonzero terms in the outer
sum will be those with $m = 0$ and $m = 1$. For $m = 0$, we obtain just the original tensor $\AA =
\Sym{\nabla\fmom[n-1]}$, while for $m = 1$, we obtain the inner sum over terms
\begin{equation*}
  \Sym{\kron{\alpha_1}{\alpha_2}\gradcomp{\beta}\fmom[n-1]_{\beta\alpha_3\ldots\alpha_n}},\quad
  \Sym{\kron{\alpha_1}{\alpha_3}\gradcomp{\beta}\fmom[n-1]_{\alpha_2\beta\alpha_4\ldots\alpha_n}},\quad\mbox{etc.},
\end{equation*}
that is $n-1$ times $\Sym{\eye\otimes\nabla\cdot\fmom[n-1]}$. We therefore get
\begin{equation}
\begin{aligned}\label{eq:MC-split0}
  \nabla\fmom[n-1]\contr\bomega^n &= \Sym{\nabla\fmom[n-1]}\contr\bomega^n\\
    & = \Sym{\nabla\fmom[n-1] - \frac{n-1}{2n-1}\eye\otimes\nabla\cdot\fmom[n-1] + \frac{n-1}{2n-1}\eye\otimes\nabla\cdot\fmom[n-1]}\contr\bomega^n\\
    & = \Sym{\nabla\fmom[n-1] - \frac{n-1}{2n-1}\eye\otimes\nabla\cdot\fmom[n-1]}\contr\bomega^n + \frac{n-1}{2n-1}\left(\nabla\cdot\fmom[n-1]\right)\contr\bomega^{n-2}
\end{aligned}
\end{equation}
where the last equality comes from the realization that $\eye\cdot\bomega^2 = \bomega\cdot\bomega = 1$ and 
$\nabla\cdot\fmom[n-1]$ is totally symmetric since $\fmom[n-1]$ is. 
Note that we have actually obtained that
\begin{equation}\label{eq:MC-split}
  \nabla\fmom[n-1]\contr\bomega^n = \detr\Sym{\nabla\fmom[n-1]}\contr\bomega^n + 
  \frac{n-1}{2n-1}\left(\nabla\cdot\fmom[n-1]\right)\contr\bomega^{n-2}.
\end{equation}
Eq. \eqref{eq:MC-split} shows that even though $\nabla\fmom[n-1]$ by itself is neither symmetric nor traceless, its 
contraction with $\bomega^n$ can be split into a sum of two contractions of TST tensors (total symmetry and 
tracelessness of the latter follows from the fact that $\fmom[n-1]$ is TST). By introducing this splitting into eq. 
\eqref{eq:MC-bte3} and regrouping the sum by $\bomega^n$, we finally obtain
\begin{equation}\label{eq:MC-bte4}
\begin{multlined}
  \suma[n]{0}{\infty}\left\{\Sym{\nabla\fmom[n-1] - \frac{n-1}{2n-1}\eye\otimes\nabla\cdot\fmom[n-1]} + \frac{n+1}{2n+3}\nabla\cdot\fmom[n+1]+\right.\\
  \left.\vphantom{\Sym{\nabla\fmom[n-1] - \frac{n-1}{2n-1}\eye\otimes\nabla\cdot\fmom[n-1]} + \frac{n+1}{2n+3}\nabla\cdot\fmom[n+1]}
     + \sigma_t\fmom - \sigma_{sn}\fmom - \kron{n}{0}\nu\sigma_f\phi - \qmom
  \right\}\contr\bomega^n = 0
\end{multlined}
\end{equation}
(again with the non-sensical tensors with negative ranks discarded).

Equation \eqref{eq:MC-bte4} is completely equivalent to eq. \eqref{eq:MC-bte3} or \eqref{eq:MC-bte2}, but regrouped to a
form of a vanishing linear combination of $\bomega^n$ with TST coefficient tensors for each $n$. Theorem
\ref{thm:MC-suffcond} therefore applies and we get the desired set of first-order partial differential equations for the
angular flux moments:
\begin{equation}\label{eq:MC-set}
  \frac{n+1}{2n+3}\nabla\cdot\fmom[n+1] + \Sym{\nabla\fmom[n-1] - \frac{n-1}{2n-1}\eye\otimes\nabla\cdot\fmom[n-1]} + 
  \sigma_t\fmom - \sigma_{sn}\fmom - \kron{n}{0}\nu\sigma_f\phi = \qmom
\end{equation}
for each $n = 0,1,\ldots$. 

\subsection{The $\MCPN$ approximation}
Note that in passing to a finite approximation by the simplest closure $\nabla\cdot\fmom[N+1] \equiv \mathbb{O}^{N}$ for
some $N \geq 0$, we do not spoil the TST character of the $N$-th coefficient tensor (in view of \eqref{eq:MC-split0},
this closure actually means that we are neglecting the nonzero traces of $\nabla\fmom[N-1]$).
The set \eqref{eq:MC-set} for $n \leq N$ may be regarded as an alternative to the ordinary $\PN$ equations and because
its solution represents the expansion of angular flux into Maxwell-Cartesian surface spherical harmonics of degrees up
to $N$, it will be called \textit{$\MCPN$ approximation}. We now explicitly state the first four equations in the
$\MCPN[3]$ set. We put $$\Sa{n} = \sigma_t - \sigma_{sn} - \kron{n}{0}\nu\sigma_f$$ and also include the $\otimes$
symbol whenever multiplication of tensors of rank $\geq 1$ occurs and finally obtain

\begin{equation}\label{eq:MCP3}
\begin{aligned}
    \tfrac{1}{3}\nabla\cdot\fmom[1] 
      +   {\Sigma}_{0}\phi
      &=  {q}^{(0)} \\[.2em]
    \tfrac{2}{5} \nabla\cdot\fmom[2] + \Sym{\nabla\otimes\phi} 
      +   {\Sigma}_{1}\fmom[1] 
      &=  {q}^{(1)}  \\[.2em]
    \tfrac{3}{7} \nabla\cdot\fmom[3] + \Sym{\nabla\otimes\fmom[1]} - \tfrac{1}{3} \Sym{\eye\otimes\nabla\cdot\fmom[1]}
      +   {\Sigma}_{2}\fmom[2]
      &=  {q}^{(2)}  \\[.2em]
                                       \Sym{\nabla\otimes\fmom[2]} - \tfrac{2}{5} \Sym{\eye\otimes\nabla\cdot\fmom[2]}
      +   {\Sigma}_{3}\fmom[3]
      &=  {q}^{(3)}
    \end{aligned}
\end{equation}

We see that the $\MCPN[1]$ set is equivalent to the $\PN[1]$ set. For higher $N$, we have more equations in the $\MCPN$
set than in the corresponding $\PN$ set, but because the coefficient tensors are bound to be TST, we can choose for each
$n$ any $2n+1$ equations and solve them for the corresponding $2n+1$ moments (which are the coefficients in
\eqref{eq:MC-exp3} at the functions forming an irreducible linearly independent subset of $\PP(\bomega)$) and express
the remaining ones as their linear combinations.

More important however appears the potential of using the symmetric and traceless structure of the $\MCPN$ equations to
provide new perspectives on some other widely used approximations or to create new ones. For instance, by projecting
each tensor along a chosen axis (say $z$), we obtain one dimensional equations which are exactly the same as the 1D
$\PN$ equations (since both symmetrized terms in every $n \geq 2$ $\MCPN$ equation become equal to
$\der{\fmom[n-1]_z}{z}$) if we take into account the definition of moments in the classical $\PN$ equations and multiply
each $\fmom[n]_z$ accordingly by a factor ${2n+1}$. This indicates the possibility to investigate the original ad-hoc
derivation of the $\SPN$ equation by formal extension of the 1D $\PN$ equations into 3D in the current tensorial
framework. Similarly, by multiplying each $\fmom[n]$ by $\frac{n!}{(2n-1)!!}$, we arrive at the system derived by Coppa
(\cite{Coppa3}).

There is also an interesting link to an old article of Selengut (\cite{Selengut}) in which the full multidimensional
$\PN[3]$ solution is obtained by solving a set of two coupled diffusion equations\footnote{much like in the $\SPN[3]$
approximation, but apparently without restrictions on dimensionality or cross-sections other than the usual isotropic
scattering and volumetric source assumptions} with special interface conditions in presence of multiple heterogeneous
regions. Selengut's derivation of the set is however quite puzzling (see also commentary in \cite[Sec. 5.2]{McClarren2})
and his equations have never been either analyzed or at least numerically tested. Assuming sufficient differentiability
of the angular flux moments $\fmom$ in eq. \eqref{eq:MCP3}, isotropic volumetric sources ($q^{(n)} = 0$ for $n\geq 1$)
and no void regions (i.e., all $\Sigma_{rn} > 0$), one can easily combine all equations into a single equation of the
form
\begin{equation}\label{eq:mom2}
  \fmom[2] = F(\phi, \fmom[2]) + \frac{1}{3\Sigma_2}\eye\Bigl(q^{(0)} - \Sigma_0\phi\Bigr)
\end{equation}
with $F$ being a differential operator. The right hand side of \eqref{eq:mom2}, however, is merely a symmetric tensqor
(matrix) with non-zero trace, and \eqref{eq:mom2} is thus not compatible with our definition of $\fmom[2]$. By 
adding the \textit{compatibility condition} $$
  \trace\left[F(\phi, \fmom[2]) + \frac{1}{3\Sigma_2}\eye\Bigl(q^{(0)} - \Sigma_0\phi\Bigr)\right] = 0,
$$
however, one can eliminate the components of $\fmom[2]$ and get the following equation for scalar flux
\footnote{This has been done so far only for the $\MCPN[3]$ set using the computer algebra system Mathematica, but a 
general procedure can be almost certainly formulated, leading to scalar flux equation with $\nabla^6$ term for
 $\MCPN[5]$ and so forth.}
\begin{equation}\label{eq:scfl}
  \frac{3}{35\Sigma_1\Sigma_2\Sigma_3}\nabla^4\phi + \left(\frac{4}{15\Sigma_1\Sigma_2} + 
  \frac{9}{35\Sigma_2\Sigma_3}\right)\nabla^2(q_0 - \Sigma_{0}\phi) - \frac{1}{3\Sigma_1}\nabla^2\phi
	  = q_0 - \Sigma_{0}\phi,
\end{equation}
which in the case of isotropic scattering reduces exactly to Selengut's Eqn. (5). Further investigation of this 
connection may be useful as it could shed some light on Selengut's derivation of interface conditions or lead to a
 new one (including the boundary conditions).


