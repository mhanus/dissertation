\chapter{Introduction}\label{chap:intro}
% \thispagestyle{empty}
It is a common agreement nowadays that nuclear energy is the only available source able to fulfill current and future
energetic demands of mankind without  polluting the Earth any further. Preceding its useful utilization, such as
generation of electricity, materials irradiation or for medical purposes, its extraction within a nuclear reactor must
occur.
 Nuclear reactors are sophisticated devices composed of diverse materials like fuel, coolant, regulation and structural
 materials. All these constituents are arranged in a highly heterogeneous manner due to various safety, technological
 and economical considerations. Depending on the time spent within the reactor, both their physical and mechanical
 properties change and some of them (like fuel elements of control rods) are also subject to repeated rearrangement.

Design of such reactors and analysis of their various operational modes is therefore a complicated task that encompasses
several areas of science and engineering. At its start, however, determination of neutronic conditions within the
reactor core plays a crucial role and has received a substantial attention in the field of reactor physics over the past
decades. Main objective of such neutronic analyses is to describe and predict the reactor state under various
circumstances and to find its optimal configuration, in which it is capable of long-term, self-sustained operation with
only a minimal human intervention.

The optimal operation is achieved by automating the regulation devices to balance neutron production from fission chain
reaction and their loss due to capture and out of core leakage. Such automation foremost requires knowledge of a
long-term behaviour of any tested core configuration, which may be obtained by performing a steady-state analysis of the
reactor core. Methods used for these analyses should accurately calculate the neutron multiplication factor
(characterizing the departure of the core from the desired equilibrium state) and spatial distribution of neutron flux
(from which subsequently the important reaction rates, power and heat distribution can be obtained for further
assessment of reactor operation). Since there is usually a multitude of possible core configurations, all calculations
should proceed swiftly to be practically applicable.

\section{Past work and motivation for improvements}
Formulation and implementation of a method suitable for the just mentioned calculation tasks has been the main subject
of the author's Bachelor thesis \cite{bib:HanusBP}.
It thoroughly describes the development of a two-dimensional solver capable of efficient calculation of the fission
multiplication factor and corresponding neutron flux distribution for an arbitrary core composition. The method has been
implemented in the MATLAB\nomenclature[X]{MATLAB}{\textbf{MAT}rix \textbf{LAB}oratory}{}{} development
environment and successfully tested on a few benchmark problems.
The current text originated as the Master thesis, building upon and referring to \cite{bib:HanusBP} from time to time.
Although it is not neccessary to read the Bachelor thesis in order to understand this book, it may help further clarify
the exposition, particularly at the places of explicit citations. Therefore, the Bachelor thesis is available
electronically at \url{https://github.com/mhanus/publications/raw/bthesis/bakule/bakalarka/thesis.pdf}.

At the end of \cite{bib:HanusBP}, several directions for future investigation and enhancements have been outlined. Also,
many new requirements  have come from the company {\v S}koda-JS\nomenclature[X]{{\v S}JS}{\textbf{\v S}koda
\textbf{J}adern{\' e} \textbf{s}troj{\' i}renstv{\' i}, a.s. ({S}koda Nuclear Machinery, Inc.)}{}{}, the industrial
partner in the project\footnote{Project No. 1M0545 of the Czech Ministry of Education} under whose support both theses
were created, or simply from the ongoing scientific progress.
To identify the major areas for modernization, reflections on the main parts of \cite{bib:HanusBP} will now be given
from the current point of view.

% \vspace*{1em}
\subsection{Improving the mathematico-physical model} 
Expressing the balance of the processes occurring in the core mathematically leads in full generality to an
integro-differential equation with seven independent variables -- the Boltzmann equation of neutron transport. This has
been thoroughly described in Section 3.1 of the thesis. The following section takes the reader through various
simplifications of the Boltzmann equation, resulting in a system of partial differential equations with only three
independent variables (spatial coordinates in three dimensions) for the steady-state calculations. The two major
approximations performed on the course are the assumption of diffusion-like behaviour of neutrons within the core and
their categorization into several groups of constant energy.

\subsubsection*{Limitation of diffusion approximation} The diffusion theory has long served as a standard tool for
whole-core neutronic calculations. It allows an efficient numerical formulation and yet produces plausible results for
many realistic reactor configurations. However, the assumptions made in its derivation severely restrict the
circumstances under which the approximation is reasonable. Neutrons can be safely expected to move  from places of their
highest concentration (corresponding to highest neutron flux) to those less occupied (according to the well-established
Fick's law of diffusion) only in highly scattering and weakly absorbing media. The scattering is the only process
contributing to neutron flux during the derivation (i.e. no sources and sinks have been assumed) and should be isotropic
or only ``slightly'' anisotropic (what is meant by that will become clear in \sref{sec:dif}).  High absorption would
lead to rapid spatial variation of neutron flux similarly to the proximity to interfaces separating different materials
and invalidate another assumption made in the original derivation of the theory (\cite[Chap. 3]{StaceyNRP}).
% \clearpage

Consequently, at the peripheral regions of the core where structural material of the pressure vessel replaces the
inner-core composition, near the control rods and localized sources in currently operating production and research
reactors, in cores loaded with high-burnup fuel (an obvious and logical trend in todays fuel management) or in cores
containing the MOX\nomenclature[X]{MOX}{\textbf{M}ixed \textbf{O}xide \textbf{F}uel}{}{} fuel adjacent to standard
UO$_2$ fuel (an effective way to dispose of plutonium originating from weapons or spent fuel reprocessing), the theory
warns us against using the diffusion approximation and the practical results of some recent difficult problems
corroborate the warning.
%  Fortunately in practice, the diffusion theory calculation can be often made accurate enough even in these cases by
% applying transport theory corrections cleverly precalculated using the Boltzmann equation, which do not change the
% overall structure of the equations (typically, they are incorporated into the input data to a diffusion method). Some
% recent studies show, however, that especially for the last mentioned problems with high-burnup and MOX fuels, even
% with such corrections unsatisfactory results are produced by diffusion methods.

There are also cases where the diffusion becomes inadequate at all due to its isotropic nature, i.e. the inability to
capture the phenomena associated with neutrons' direction of motion, like modeling the
ADS\nomenclature[X]{ADS}{\textbf{A}ccelerator \textbf{D}riven \textbf{S}ystems}{}{} systems or shielding calculations.
In order to develop a solution method capable of treating these cases, a full transport theory must be used as a basis
instead of the diffusion approximation.

\subsubsection{Limitation of the two-group approximation}
Discretization of the continuous energetic dependence of neutrons into several intervals (\textit{groups}) relies on a
careful precalculation of group-wise constant parameters for the resulting few-group system of governing equations. This
is usually done by a multigroup calculation over single or several neighboring fuel assemblies (using a high-fidelity
physical model), providing detailed energetic distribution of neutron flux (i.e. its \textit{spectrum}), which is used
in turn to collapse the multigroup data into the desired number of groups (the so called \textit{group condensation}, or
\textit{collapsing}). The resulting set of few-group constants is expected to capture most of the spectral
characteristics of the given material. Due to a very complicated energetic dependence of core constituents and their
non-trivial spectral interaction (e.g. in the mixed MOX/UO$_2$ cores or various future reactor designs), however, this
becomes hardly achievable when collapsing into only a small final number of energy groups. Modern methods should
therefore be applicable in finer energy group structures or even use a better approximation of energy spectrum and
effects associated therewith (e.g. the resonance self-shielding, as demonstrated in \cite{bib:DownarARSM}).

% \vspace*{1em}
\subsection{Improving the numerical method}
Perhaps the simplest numerical method for solving boundary value problems like that of neutron diffusion is based on the
finite difference approximation of the involved differential operators and has become a workhorse of computer neutronic
calculations since their beginning around 1960s. More computationally and physically pleasing methods have then been
developed based on the variational or conservation principles, but the stringent demands on the fineness of the
computational grid remained major obstacles to their efficient application. To overcome this drawback, the so called
nodal methods were devised in mid seventies and have matured until today. They are now used to speed-up majority of both
diffusion and transport codes.

\subsubsection{Nodal method}\label{sec:Intro1121}
Modern nodal methods are built upon a conservative discretization of the governing equations by the classical finite
volume method (FVM). Somewhat against the theory of the method, however, the finite volumes are intentionally kept quite
large so as to facilitate efficient solution of the equations for the whole core. Usually, the computational volumes (or
\textit{nodes}) are chosen as coarse as the whole fuel assembly. Discretization errors thus introduced are accounted for
by a clever correction scheme, which (similarly to a classical multigrid method) employs an additional level of
calculation at several stages of the coarse mesh calculation. The more accurate approximation (commonly termed ``higher
order approximation'') of desired quantities obtained from this second level is compared with that of the finite volume
method on the coarse mesh to determine the correction. The solution of this iterative process converges to that of the
more accurate method but with substantially less computational demands. This correction procedure is commonly known as a
\textit{coarse-mesh, finite-difference} method (since the primary part of its iteration matrix represents the
finite-difference equations obtained from the finite volume discretization over coarse volumes), or
\textit{CMFD}.%\parend

There are several ways of constructing the higher order solution and many actually do not require finer mesh. These
methods either use knowledge of the analytic solution of the governing equations within each node (i.e. fuel assembly)
or approximate the nodal flux shape by its projection into a space spanned by a finite set of suitable basis functions
defined over the node. The former are hence termed \textit{analytic nodal methods} (\textit{ANM}) and some successful
examples are described e.g. in \cite{bib:GrundmannDYN3D} or \cite{bib:ChoAFEN}.
Methods of the other class are called \textit{nodal expansion methods} (\textit{NEM}). They differ among each other
primarily in the choice of basis functions used for approximating the unknown flux function, leading to methods like
\textit{PNM} (\textit{polynomial nodal method}) or \textit{SANM} (\textit{semi-analytic nodal method}). In our opinion,
nodal expansion methods are somewhat easier to formulate and implement, particularly when finer energetic discretization
is required, than the analytic methods. They are also more easily used in conjunction with the transport model. Since
results of many numerical experiments with methods from both categories indicate comparative solution accuracy, the
nodal expansion method has become the method of choice for our previous work and is described in more detail in
\cite{bib:HanusBP}.

\subsubsection{Homogenization and reconstruction}
In the derivation of the nodal method equations, one assumes that material properties of each node are spatially
constant, i.e. the node is physically homogeneous. Since one node typically encompasses the whole fuel assembly,
however, this condition almost certainly does not hold even for a fresh assembly at the start of a new reactor campaign
(which contains complicated arrangement of fuel rods, or \textit{pins}, with varying fuel material composition, their
cladding and the moderator). Physically heterogeneous nodes are therefore first converted into homogeneous ones by a
(purely mathematical) homogenization procedure, which ensures that the solution considering the simpler homogenized
configuration will be equivalent to one that would be obtained without it, at least in the sense of preserving important
quantities like reaction rates and multiplication factor. This is often done in a similar fashion as energy group
condensation, i.e. by performing a single-assembly calculation with fictitious boundary conditions (usually
corresponding to an infinite lattice of nodes with the same properties as the one being homogenized).

Although there are applications, such as fuel optimization, where global reactor quantities (multiplication factor,
assembly average power distribution) are the only required results, there are also others (like assembly power peaking
estimation or thermal-hydraulic calculations) that require knowledge of detailed, pin-by-pin neutron flux distribution.
Because nodal methods yield only the integral averages of neutron flux over the nodes, determination of the node-wise
flux distribution from these averaged values is certainly not self-evident. A common way to resolve this issue of nodal
methods is to utilize for the pinwise flux reconstruction the information about its characteristic shape, previously
obtained during homogenization.
% Note that using only a one-level transverse integration procedure when formulating the nodal equations makes the
% reconstruction process easier, since a complete radial flux distribution is directly obtained from the homogenized
% nodal solution.

\subsubsection{Calculation procedure}

The solution of a steady state core problem translates either into a fixed source calculation, in which the neutron flux
produced by the specified external neutron source is determined, or into an eigenvalue problem, in which the dominant
eigenpair is sought as it is the only one corresponding to real physical quantities, namely the reactor multiplication
factor and neutron flux (\cite[Sec. 3.3.1.1]{bib:HanusBP}). In the reference, we dealt exclusively with the latter as it
is the most frequent type of calculation performed in reactor analyses. As has been demonstrated there, traditional
numerical methods like the power or Rayleigh-Ritz method can be used to determine the dominant eigenpair.

However, in many real-world reactor problems, these simple iterative methods converge very slowly. Many techniques have
been devised in need for an effective acceleration, like the Wielandt's shift method, Chebyshev or asymptotic source
extrapolation. Although these methods can significantly reduce the number of iterations in some cases, there are
problems in which even their convergence rate may become too slow. Furthermore, they depend rather sensitively on one or
more parameters that must be guessed a priori. Recently, however, Krylov subspace methods have begun to be used in
numerical core studies as they are believed to bring a solution at least to the last problem. In summary, acceleration
schemes for the eigenvalue calculations are certainly a valuable addition to a core calculation code, although their
usage should not be automatic and is likely to require an expert input.

There are many areas of nuclear reactor studies where the calculation performance is highly important. In the fuel
loading optimization process, for example, the steady state equation has to be evaluated many times in order to find the
optimal condition. Another area, where the need for a fast core neutronic solver is even aggravated because a
quasi-static approach cannot be used, is the analysis of the transient states of the reactor, such as the startup,
shutdown or abrupt control rod movement during an accident. Although not being the subject of this thesis, a good method
for steady-state reactor calculations should be prepared for a dynamical extension. Selection of mathematical and
numerical models should therefore be guided with computational efficiency in mind.


% \subsection{Improving the implementation}  Production nuclear reactor analysis codes are written in a low-level
% programming language, such as C or Fortran, taking advantage of their reliability, execution speed and portability.
% Due to their complexity, prototyping, rapidly changing and testing the code in these languages is impractical. There
% are two tools that are in my opinion perfectly suited for such work in the sphere of scientific computing -- the
% MATLAB and Mathematica development environments. The former is primarily designed for numerical computation while the
% latter is better suited for symbolic manipulations with involved mathematical expressions, although this
% categorization is of course not strict. \parend MATLAB has already been used to develop the previous nodal code and
% proved to be quite efficient both in terms of the programming and execution time. Perhaps the single most important
% feature of the new versions released in the course of development of the present nodal code is their refined object
% oriented capability. The new code could have been therefore designed and written in a modern, object oriented fashion,
% without losing the advanced data structures and functions available in MATLAB. \parend Relatively to the previous
% nodal method, conformal mapping and transport methodology increased the complexity of mathematical formulations that
% have to be represented in the computer code. Since a majority of them is only needed to be evaluated and manipulated
% once and just the results may be used in the final code, Mathematica proved to be an invaluable companion to MATLAB in
% developing the code.

\section{Present work}

During the improvement of the CMFD and nodal methods developed at the author's institute, a new requirement has arisen
to create a fine-mesh, finite-difference (FMFD) solver. Although this may seem as a step backwards, the fine mesh
heterogeneous solver is actually a key ingredient in the homogenization procedure, as will be shortly described in
\sref{sec:Homogenization}. Moreover, after its calibration on available acknowledged results from literature, it has
proven to be an invaluable aide in validating the developed nodal method against the input data from the partner company
{\v S}-JS.

Most notably, however, the FMFD discretization approach provides a simple way to deal with spatial dependence of the
Boltzmann equation and allows thus to concentrate on the remaining independent variables in the equation. The main topic
of this work is the exploration of the transport theoretical models and methods capable of improving the accuracy of the
diffusion based solvers, that is, without the need for rewriting the whole existing code from scratch. It tries to
provide a complete (although not at all exhaustive) overview of the whole process of transforming the neutron transport
equation into a readily programmable numerical method.

\subsection{Structure of the text}
Chapter 2 is devoted to the mathematical theory of neutron transport, which is in the literature often either silently
omitted (in the physicists' works) or tremendously involved (the mathematicians' works). Section \ref{sec:ntt} presents
the fundamental equation of the theory together with a physical explanation of its terms and some basic
simplifications.\\
\indent Section \ref{sec:ms} then describes a functional-analytic setting of the whole theory, which proves very useful
for describing the methods generally applicable to any transport approximation (including diffusion) in the subsequent
parts.\\
\indent The two problems of primary interest for applications are presented next, in \sref{sec:two_problems} -- namely
the eigenvalue (core criticality) calculation and the calculation with specified non-changing neutron sources. Overview
of the mathematical conditions necessary to prove the existence of their solution and a connection to the corresponding
physical conditions are also given.\\
\indent The chapter is concluded with sections \ref{sec:angular} and \ref{sec:tr-slab}, which concentrate on the
complicated directional dependence of the processes governed by the neutron transport equation and develops their
convenient mathematical representation.

Chapter 3 focuses on the actual discretization of the energetic and directional dependence of Boltzmann's equation.
Energetic dependence is attended to first in \sref{sec:multigroup}, which describes the multigroup approximation and the
natural solution technique of the resulting quasi-discrete equations (the \textit{source iteration}).\\
\indent The angular variable is addressed next in \sref{sec:angular_approximation}. Two widely used methods -- that of
\textit{discrete ordinates} and that of \textit{spherical harmonics} -- are described in the section, again with a link
to the classical methods of numerical mathematics. The latter is then chosen for a closer study, ultimately leading to
the so called \textit{simplified spherical harmonics} approximation.\\
\indent In the final \sref{sec:mgsp3}, the results of the energetic discretization are linked together with those of the
directional one to formulate the multigroup simplified spherical harmonics approximation.

Chapter 4 addresses the remaining continuous variable in the transport equation by describing the spatial discretization
using a finite volume scheme. As the  main targetted application domain is the neutronic analysis of thermal light water
reactors (LWR\nomenclature[X]{LWR}{\textbf{L}ight \textbf{W}ater \textbf{R}eactor}) with hexagonal fuel assemblies, the
spatial discretization is performed accordingly -- by the coarse-mesh, finite-difference method accompanied by a nodal
method to reduce discretization errors, and by the fine-mesh, finite-difference method. The first approach is described
only very briefly, leaving more space to the latter. The FMFD method is then shown to give consistent results with the
acknowledged fine-mesh codes in the three-dimensional diffusion and transport-theory calculations.

% \clearpage
