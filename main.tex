\documentclass[14pt,twoside]{extbook}

% fonts
\usepackage{mathrsfs}
\usepackage{stmaryrd}

% math
%\usepackage[intlimits]{amsmath}
\usepackage{amsmath}
\usepackage{amsfonts}
\usepackage{amssymb}
\usepackage[amsmath,thmmarks,thref]{ntheorem}
%\usepackage{stmaryrd}
%\DeclareMathAlphabet{\mathpzc}{OT1}{pzc}{m}{it}
\usepackage{mathtools}
\mathtoolsset{prescript-sup-format=\mathrm}

%\usepackage{bm}
%\usepackage{txfonts}
\usepackage[T1]{fontenc}
%\usepackage{mathptmx}
%\usepackage{lmodern}

\usepackage{footmisc}
\renewcommand\footnotelayout{\fontsize{11}{11}}

% graphics
\usepackage{graphicx}
\usepackage{color}
%\usepackage{subfigure}
\usepackage{rotating}
%\newskip\rotFPtop \rotFPtop=2cm
%\newskip\rotFPbot \rotFPbot=-2cm
% hypertext
\usepackage[colorlinks=false,dvips]{hyperref}
\usepackage{breakurl}

% additional
\usepackage{isotope}
\usepackage{sistyle}
\usepackage{reference}
\usepackage{textcomp}
\usepackage{ifthen}

% main stylefile
\usepackage{mythesis}
\usepackage[font=footnotesize]{subfig}

%\newcommand{\parend}{\\[.5em]\indent}
%\newcommand{\parend}{\\ \indent}
%%%%%%%%%%%%%%%%%%%%%%%%%%%%%%%%%%%%%%%%%%%%%%%%%%%%%%%%%%%%%%%%%%%%%%%%%%%%%%%%
% MACRO DEFINITIONS

% for nomenclature

\newcommand{\refchap}[1]{~\itshape\footnotesize(in Chap.~#1)}
\newcommand{\nomwide}[1]{\protect\parbox{\textwidth}{#1}}

% abbreviations

\newcommand{\ttef}{transport theory expression for~}
\newcommand{\dtef}{diffusion theory expression for~}

\def\eqrefs#1#2{(\protect\ref{#1}--\protect\ref{#2})}
\newcommand{\mat}[5][cc]{%
	\left[\begin{array}{#1}%
		#2 & #3\\%
		#4 & #5%
	\end{array}\right]
}
\renewcommand{\vec}[3][c]{%
	\left[\begin{array}{#1}%
		#2\\%
		#3%
	\end{array}\right]
}
%\newcommand{\gmat}[1]{%
%	\left[\begin{array}{cccc}%
%		#1^1	& 			& 			& 		\\
%				&	#1^2	& 			& 		\\
%				&			& \ddots	& 		\\
%				&			&			& #1^G
%	\end{array}\right]
%}
\newcommand{\gmate}[1]{%
	\left[\begin{array}{ccc}%
		#1^1	& 				& 			\\
				&	\ddots	& 			\\
				&				&	#1^G
	\end{array}\right]
}
\newcommand{\gvece}[2]{%
	\left[\begin{array}{c}%
		#1^1\\
		\vdots\\
		#1^G
	\end{array}\right]\it{#2}
}
%\newcommand{\gmat}[1]{\left[\vphantom{\tilde{S_0}}#1\right]}
%\newcommand{\gvec}[2]{\left\{\vphantom{\tilde{S_0}}#1\it{#2}\right\}}
\newcommand{\smat}[1]{\bigl[#1\bigr]}
\newcommand{\sgmat}[1]{[[#1]]}
\newcommand{\gmat}[1]{\boldsymbol{#1}}
\newcommand{\gvec}[2]{\boldsymbol{#1}\it{#2}}
\newcommand{\lvec}[1]{\left[#1\right]}
\newcommand{\lvect}[1]{\left[#1\right]^T}


\newcommand{\nullmat}{\mathbf{0}}

\newcommand{\iter}[2]{\\\hspace*{#2\medskipamount}#1}

% Arabic subequation numbering
\makeatletter
%Arabic subequations counter
\newcounter{arsub}
\setcounter{arsub}{-1}
\renewcommand{\thearsub}{%
\theparentequation{}-\arabic{arsub}%
}
% save the original counter
\newcommand{\c@org@eq}{}
\let\c@org@eq\c@equation
\newcommand{\org@theeq}{}
\let\org@theeq\theequation
%\setarsub sets arabic subequations counting
\newcommand{\setarsub}{%
\begingroup %
\let\c@equation\c@arsub %
\let\theequation\thearsub}
%\setorig sets the original
\newcommand{\setorig}{%
\setcounter{arsub}{-1} %
\let\c@equation\c@org@eq %
\let\theequation\org@theeq %
\endgroup}
\makeatother



% gray box around major steps in algorithm writings
\newlength{\savedparindent}
\newcommand{\SaveIndent}{\setlength{\savedparindent}{\parindent}}
\newcommand{\RestoreIndent}{\setlength{\parindent}{\savedparindent}}
\newcommand{\InGray}[1]{%
	\SaveIndent{} %
	\noindent{} \fcolorbox[rgb]{0,0,0}{0.95,0.95,0.95}{
		\begin{minipage}{0.965\linewidth} %
			\RestoreIndent{}%
			#1
		\end{minipage}
	}%
}

\newcommand{\alb}{\begin{aligned}}
\newcommand{\ale}{\end{aligned}}
\newcommand{\gatb}{\begin{gathered}}
\newcommand{\gate}{\end{gathered}}
\newcommand{\VV}{\mathcal{D}}
\newenvironment{piecewise}{\begin{cases}\displaystyle}{\end{cases}}
\newcommand{\pw}[4]{
	\begin{cases}\displaystyle
		#1 & \mbox{#2},\\ 
		#3 & \mbox{#4}
	\end{cases}
}
\DeclareMathOperator*{\esssup}{\mathrm{ess}\,\sup}
\DeclareMathOperator*{\MFP}{\mathrm{MFP}}
\DeclareMathOperator*{\diag}{\mathrm{diag}}
\DeclareMathOperator*{\col}{\mathrm{col}}

%
% commands introduced in
%

% chapter 2
\newcommand{\polar}{\vartheta}
\newcommand{\azimuthal}{\varphi}
\newcommand{\muav}{\overline{\mu}_0}

\newcommand{\be}{\ensuremath{\mathbf{e}}}

\newcommand{\sint}{\sin \polar}
\newcommand{\sinp}{\sin \azimuthal}
\newcommand{\cost}{\cos \polar}
\newcommand{\cosp}{\cos \azimuthal}

\newcommand{\unit}[1]{\,[\SI{}{#1}]}
\renewcommand{\SS}{\ensuremath{\mathcal{S}}}
\newcommand{\EE}{\mathcal{E}}
\newcommand{\sooriented}{\uparrow\uparrow}
\newcommand{\norm}[2][]{\Vert #2 \Vert_{#1}}
\newcommand{\pX}[1][]{\partial X^{#1}}
\newcommand{\ptX}[1][]{\partial \tilde{X}^{#1}}
\newcommand{\tX}{\tilde{X}}
\newcommand{\R}[1][]{\ensuremath{\mathbb{R}^{#1}}}
\newcommand{\Ef}{\epsilon_{\mathrm{f}}}
\newcommand{\bomegain}{\bomega_{\mathrm{in}}}
\newcommand{\PP}[1]{\ensuremath{\mathrm{P}_{#1}}}
\newcommand{\DP}[1]{\mathrm{DP}_{#1}}
\newcommand{\SN}[1]{\ensuremath{\mathrm{S}_{#1}}}
\renewcommand{\P}[1]{P_{#1}}
\newcommand{\SP}[1]{\ensuremath{\mathrm{SP}_{#1}}}
\newcommand{\Y}[2]{Y_{#1}^{#2}}
\newcommand{\Yc}[2]{\bar{Y}_{#1}^{#2}}
\newcommand{\kron}[2]{\delta_{#1#2}}
\newcommand{\keff}{\ensuremath{k_{\mathrm{eff}}}}

\newcommand{\aint}[1][]{\displaystyle\int_{4\pi}\d{\bomega#1}}
\newcommand{\azint}[1][]{\displaystyle\int_0^{2\pi}\d{\azimuthal}\!}
\newcommand{\eint}[2][]{
  \ifthenelse{\equal{#2}{}}
		{\int_{\EE}\d{E#1}}
		{\int_{#2}\d{E#1}}
}
\newcommand{\xint}[1][]{\displaystyle\int_{X}\d{x#1}}
\newcommand{\muint}[1][]{\int_{-1}^{1}\d{\mu#1}}
\newcommand{\bndint}[2][]{\displaystyle\int_{#2}\db{x#1}}
\newcommand{\vint}[1][V]{\int_{#1}\d{\br}}
\newcommand{\zint}[1][V_z]{\displaystyle\int_{#1}\d{z}}
\renewcommand{\d}[1]{\ensuremath{\mathrm{d}#1\,}}
\newcommand{\db}[1]{\ensuremath{\mathrm{d_b}#1\,}}
\newcommand{\angmom}[2][]{\ensuremath{\phi^{#1}_{#2}}}
\newcommand{\der}[3][]{\ensuremath{\frac{\mathrm{d}^{#1}{#2}}{\mathrm{d} {#3}^{#1}}}}


\newcommand{\rR}{R}
\newcommand{\rRR}{\mathscr{R}}
\newcommand{\rP}{\mathscr{P}}
\newcommand{\rF}{\mathscr{R}_f}
\newcommand{\tF}{\phi}
\newcommand{\rNF}{\mathscr{N}}

\newcommand{\Lp}[2][]{L_{#1}^{#2}}
\newcommand{\Wp}[2][]{W_{#1}^{#2}}
\newcommand{\Id}{I}

\newcommand{\br}{\ensuremath{\mathbf{r}}}
\newcommand{\angflux}{\ensuremath{\psi}}
\newcommand{\bj}{\ensuremath{\mathbf{j}}}
\newcommand{\bn}{\ensuremath{\mathbf{n}}}
\newcommand{\bomega}{\ensuremath{\mathbf{\Omega}}}
\renewcommand{\d}[1]{\ensuremath{\mathrm{d}#1\,}}

\newcommand{\angmomgen}[2]{\ensuremath{\phi_{#1#2}}}
\newcommand{\Qmomgen}[2]{\ensuremath{Q_{#1#2}}}
\newcommand{\Qmom}[1]{\ensuremath{Q_{#1}}}

\newcommand{\abs}[1]{\ensuremath{\left\vert#1\right\vert}}

\newcommand{\ra}{\ensuremath{\!\shortrightarrow\!}}
\newcommand{\laa}{\ensuremath{\!\leftarrow\!}}
\newcommand{\raa}{\ensuremath{\!\rightarrow\!}}
\newcommand{\pV}{\ensuremath{\partial V}}
\newcommand{\pD}{\ensuremath{\partial W}}
\newcommand{\pO}{\ensuremath{\partial \Omega}}
\newcommand{\suma}[3][l]{\sum_{#1=#2}^{#3}}
\newcommand{\subsubparagraph}[1]
	{\par\hspace*{0.5em}$\bullet$\hspace*{0.5em}{\textit{#1}}\indent}
\newcommand{\capparagraph}[1]
	{\vspace*{0.5em plus 0.5em minus 0em}\par{\textit{\textbf{#1}}}\par\vspace*{0.5em}}
\newcommand{\pd}[3][]{\ensuremath{\frac{\partial^{#1}#2}{\partial #3^{#1}}}}

\newcommand{\bJ}{\ensuremath{\mathbf{J}}}
\newcommand{\bo}{\ensuremath{\mathbf{0}}}
\newcommand{\bM}{\ensuremath{\mathbf{M}}}
\newcommand{\bF}{\ensuremath{\mathbf{F}}}
\newcommand{\bphi}{\ensuremath{\mathbf{phi}}}

% chapter 3
\newcommand{\llop}{\ensuremath{\mathbf{L}}}
\newcommand{\LLL}[1]{L_{#1}}

\newcommand{\hhop}[1][]{\ensuremath{\mathbf{H}}_{#1}}
\newcommand{\HHH}[1]{H_{#1}}
\newcommand{\FFF}[1]{F_{#1}}

\newcommand{\ffop}{\ensuremath{\mathbf{F}}}
\newcommand{\bbop}{\ensuremath{\mathbf{B}}}
\newcommand{\ssop}{\ensuremath{\mathbf{S}}}
\newcommand{\ttop}{\ensuremath{\mathbf{T}}}
\newcommand{\QQ}{\ensuremath{\mathbf{Q}}}

\newcommand{\tangflux}{\tilde{\angflux}}
\newcommand{\vangflux}{\ensuremath{\boldsymbol{\psi}}}
\newcommand{\valbedo}{\ensuremath{\boldsymbol{\beta}}}
\newcommand{\vflux}{\ensuremath{\boldsymbol{\phi}}}
\newcommand{\vangmom}[1]{\ensuremath{\boldsymbol{\phi}_{#1}}}
\newcommand{\amom}[2][]{\ifthenelse{\equal{#1}{}}{\Phi}{\Phi^{#1}}_{#2}}
\newcommand{\vamom}{\ensuremath{\boldsymbol{\Phi}}}

\renewcommand{\th}{\ensuremath{^{\mathrm{th}}}\ }
\newcommand{\nd}{\ensuremath{^{\mathrm{nd}}}\ }
\newcommand{\st}{\ensuremath{^{\mathrm{st}}}\ }

\renewcommand{\it}[1]{\ifthenelse{\equal{#1}{}}{}{\ensuremath{^{(#1)}}}}
\newcommand{\angfluxin}{\angflux_{\mathrm{in}}}

\newcommand{\CC}{\mathbf{C}}
\newcommand{\GG}{\mathbf{G}}
\newcommand{\Sa}[2][]{#1{\Sigma}_{a#2}}
\newcommand{\SSa}{\boldsymbol{\Sigma}_{a}}
\newcommand{\SSf}{\boldsymbol{\Sigma}_{f}}
\newcommand{\SSs}{\boldsymbol{\Sigma}_{s}}
\newcommand{\Src}{\boldsymbol{S}}
\newcommand{\LL}[1]{\ifthenelse{\not\equal{#1}{1}}{\mathbf}{}{L}_{#1}}
\newcommand{\BB}[1]{\ifthenelse{\not\equal{#1}{1}}{\mathbf}{}{B}_{#1}}
\newcommand{\HH}[1]{\ifthenelse{\not\equal{#1}{1}}{\mathbf}{}{H}_{#1}}
\newcommand{\FF}[1]{\ifthenelse{\not\equal{#1}{1}}{\mathbf}{}{F}_{#1}}

\renewcommand{\div}{\nabla\cdot}
\newcommand{\grad}{\nabla}

\renewcommand{\O}[1][eps]{\ensuremath{
	\ifthenelse{\equal{#1}{eps}}
		{\mathcal{O}(\varepsilon)}
		{
			\ifthenelse{\equal{#1}{}}
				{\mathcal{O}}
				{
					\ifthenelse{\equal{#1}{1}}
						{\mathcal{O}(1)}
						{\mathcal{O}(\varepsilon^{#1})}
				}
		}
	}
}
\newcommand{\jin}[2]{j_{\mathrm{in},#1}^{#2}}
\newcommand{\vjin}[1]{\mathbf{j}_{\mathrm{in}}^{#1}}

\renewcommand{\o}{\mathbf{o}}

\newcommand{\brund}{\underline{\br}}
\newcommand{\bjund}{\underline{\bj}}
\newcommand{\bJund}{\underline{\bJ}}
\newcommand{\bnund}{\underline{\bn}}
\newcommand{\ound}{\underline{\o}}
\newcommand{\NNund}{\underline{\underline{\mathbf{N}}}}
\newcommand{\vangmomund}[1]{\underline{\boldsymbol{\phi}_{#1}}}
\newcommand{\ggamma}{\boldsymbol{\Gamma}}
% chapter 4
\newcommand{\dom}{\ensuremath{\Omega}}
\newcommand{\V}{\ensuremath{\mathcal{V}}}
\newcommand{\U}{\ensuremath{\mathcal{U}}}
\newcommand{\T}{\ensuremath{\mathcal{T}}}
\newcommand{\nequiv}{\not\equiv}

\newcommand{\ddx}[2][]{\ensuremath{\frac{\mathrm{d}^{#1}{#2}}{\mathrm{d} {x}^{#1}}}}

\newcommand{\DD}[1][]{{\ensuremath{\mathbf{D}_{#1}}}}

% chapter 4

\newcommand{\flux}{\ensuremath{\phi}}
\newcommand{\aflux}{\ensuremath{\phi}}
\newcommand{\face}[2][i]{\ensuremath{\Gamma_{#1,#2}}}
\newcommand{\fint}[1][\xi]{\int_{\face{#1}}\!\d{A}\,}
%\renewcommand{\fint}[1][\xi]{\int_{\face{#1}}\!\d{A}\,}
\newcommand{\mV}{\abs{\V_i}}
\newcommand{\ms}{\ell}
\newcommand{\mf}[2][i]{\abs{\face[#1]{#2}}}
\newcommand{\jf}[2][i]{j_{#1,#2}}
\newcommand{\hl}[2][i]{h_{#1}^+}
\newcommand{\hr}[2][i+\xi]{h_{#1}^-}
\newcommand{\D}[1][i]{\ensuremath{D_{i,\xi}^g}}
\newcommand{\JF}[2][g]{\ensuremath{J_{#2}^{#1}}}


%
% environments
%
%\newcommand{\comment}{1}{}

\theoremstyle{nonumberbreak} \theoremheaderfont{\scshape}
\theorempreskipamount 1em plus 0em minus .5em \theorembodyfont{\slshape} \theoremseparator{\\[.5em]}
\newtheorem{remark}{Notational remark}

\theoremstyle{break}
\theoremheaderfont{\normalfont\bfseries}\theorembodyfont{\slshape}
\theoremsymbol{\ensuremath{\diamondsuit}}
\theoremseparator{:}
\newtheorem{Theorem}{Theorem}

%%%%%%%%%%%%%%%%%%%%%%%%%%%%%%%%%%%%%%%%%%%%%%%%%%%%%%%%%%%%%%%%%%%%%%%%%%%%%%%%
% DOCUMENT START
%
\begin{document}

% F R O N T    M A T T E R

%  external front_matter project


%         T O C

\pagestyle{fancy} 
\pagenumbering{roman}
\renewcommand{\chaptermark}[1]{\markboth{#1}{}} 
\renewcommand{\sectionmark}[1]{\markright{#1}} 
\fancyhf{}
\fancyhead[LE,RO]{\thepage} 
\fancyhead[LO]{\textup{\rightmark}} 
\fancyhead[RE]{\textsc{\leftmark}} 
\renewcommand{\headrulewidth}{0.33pt}
\tableofcontents

\newpage

\thispagestyle{empty} 
\fancyhead[LE,RO]{\thepage} 
\fancyhead[LO,RE]{\textsc{\nomname}} 
\printnomenclature[3.15cm]

%\newpage
%
%    L I S T I N G S

%\addcontentsline{toc}{chapter}{List of Figures}
%\listoffigures

\cleardoublepage


\pagestyle{fancy} 
\renewcommand{\chaptermark}[1]{\markboth{#1}{}} 
\renewcommand{\sectionmark}[1]{\markright{#1}} 
\fancyhf{}
\fancyhead[LE,RO]{\thepage} 
\fancyhead[LO]{\textup{\rightmark}} 
\fancyhead[RE]{\textsc{\leftmark}} 
\renewcommand{\headrulewidth}{0.33pt}
\pagenumbering{arabic}

% M A I N    M A T T E R

\chapter{Introduction}

Computer simulation of radiative transfer of energy is an important task in many engineering and research areas, as
diverse as biomedicine, astrophysics, optics or nuclear engineering. In nuclear engineering, the area of primary
interest in this work, there are two main goals of evaluating computer models of radiative transfer. The first is
to simulate short-term transient behavior of nuclear devices under given initial conditions such as geometry and material
configuration. The second is to determine under which conditions such devices (in this case typically nuclear reactor
cores) will be capable of long-term, stable operation satisfying certain safety, technical and economical limitations,
with only a minimal human intervention. Repeated calculations of the second type form the basis for designing new
nuclear reactors or optimizing fuel reloading of existing ones. Optimization of fuel reloading schemes for nuclear
reactors is the topic of a major research and development project that is being investigated at author's department
\footnote{Project TA01020352 -- Increasing utilization of nuclear fuel through optimization of an inner fuel cycle and
calculation of neutron-physics characteristics of nuclear reactor cores. Principal investigators: R. {\v C}ada
(University of West Bohemia) and J. Rataj (Czech Technical University). \label{ftn:TACR}}.
Author's participation in this project during the course of his doctoral studies involved the development of a
neutron-physical calculation module, employed by the overall optimization suite to evaluate fitness of its candidate
configurations. This fact largely influenced the direction for development and analysis of mathematical models and
numerical methods described in this thesis.

All forms of radiative transfer are governed by physical laws that can be abstracted by a single mathematical model,
represented by the Boltzmann transport equation. In nuclear reactors, the effects of neutron-induced reactions dominate
those caused by other types of particles and we will therefore consider the transport equation for neutrons in this
thesis, even though it has the same form for other types of non-charged particles, such as photons. While the short-term
transient simulations require accurate solution methods for the time-dependent NTE, a quasi-steady state solution (a
sequence of steady state calculations) is generally sufficient to capture slow changes in core configuration and
material characteristics during its long-term stable operation.
Author's work focus on the latter application domain further narrows scope of this thesis to the \textit{steady state
neutron transport equation}, shortly NTE \index{NTE}. It is worth recalling, however, that many common numerical methods
for solving transport problems involve repeated execution of methods designed for steady-state problems.

\paragraph{Numerical modelling of neutron transport}
We will introduce the NTE in Chap. \ref{chap:nte-review} as an integro-differential equation in 6 dimensions.
This high dimensionality requires either the use of direct simulation of neutrons and using statistical methods to obtain the
required physical quantities (the Monte Carlo approach) or a deterministic approach involving multiple discretizations.
As the second approach is still preferable in terms of overall efficiency, we choose it as a basis for our methods and
study classical discretization methods for the NTE in Chap. \ref{chap:nte-methods}. We will focus on some of
their intrinsic properties that have important consequences for their practical use. Although these properties are
generally known, they are usually described in an intuitive manner, so we will try to uncover their roots in the
mathematical structure of the corresponding equations.

\paragraph{The $\MCPN$ approximation and its relation to the $\SPN$ approximation} 
One of the traditional discretization methods that converts the 6-dimensional integro-differential NTE into a system of
familiar 3-dimensional partial differential equations (PDEs\nomenclature[Z]{PDE}{Partial differential equation}) is the
\textit{method of spherical harmonics}, denoted $\PN$\nomenclature[Z]{\PN}{Method of spherical harmonics of order
$N$}\nomenclature[Z]{\SN}{Method of discrete ordinates of order $N$}. This method is recalled and its properties are
studied in Chap. \ref{chap:nte-methods} and a new alternative set (equivalent to the original $\PN$ set) is developed in
Chap. \ref{chap:mcpn}. From this set, it is possible to directly derive another traditional approximation of neutron
transport -- the \textit{simplified spherical harmonic method}, or $\SPN$ \nomenclature[Z]{\SPN}{Method of simplified spherical
harmonics of order $N$} -- which is the subject of the second part of Chap. \ref{chap:mcpn}. So far, the author has only
been able to derive in this way the $\SPN$ equations in an interior of a homogeneous region, so in order to obtain a
practical solution method, the usual $\SPN$ interface and boundary conditions will be applied. As a result, the standard
$\SPN$ method is obtained, so the contribution of Chap. \ref{chap:mcpn} is rather theoretical, showing a new way of
arriving at the $\SPN$ equations (and an alternative form of the classical $\PN$ equations).

The $\SPN$ method (particularly the $\SPN[3]$) already simplifies the NTE to the extent that it is applicable to
day-by-day whole-core calculations on usual workstations with a few computational cores or small-scale parallel machines
with tens to a few hundred cores, which are typical machines available to nuclear engineering
companies\footnote{\label{sjsexp}from personal experience of the author coming out of the long-term collaboration with
the Czech nuclear engineering company {\v S}koda JS; more generally, see the discussion in \cite[Sec.
2.4]{Sanchez7}}. As is well known and will be recalled in \ref{chap:SPN}, when the $\SPN[3]$ method is applied to the
typical reactor core calculations, its solution captures most of the features of the true solution of the NTE. Combined
with its efficiency that allows this method to be used ``off the desk'' (without the need of submitting the job to some
supercomputing center, waiting for it to come to the front of an execution queue and gathering the results) makes it
attractive for physicists to quickly test their empirical approximations used throughout their production code, which is
usually based on the strongest transport approximation -- the diffusion approximation.

\myparagraph{Finite element framework for 2D neutron diffusion or transport problems}
The neutron diffusion approximation, whereby the NTE is reduced to a second-order elliptic PDE (or, when energy
dependence is taken into account implicitly, a weakly coupled non-symmetric system of second-order PDEs with
positive-definite symmetric part -- the so-called \textit{multigroup neutron diffusion approximation}), also forms the
basis of our neutronics module for the optimization code\footnote{Note that the $\SPN[1]$ model is actually equivalent
to the neutron diffusion equation.}
However, to obtain the final discrete algebraic system of equations, it uses the finite element method. 
This distinguishes it from the majority of other codes used for similar purposes, which are usually based on the so-called \textit{nodal method}\footnote{See e.g.
\cite{opt1,opt2,opt3} for the specific application area of core reloading optimization; some other nodal codes widely
employed in various whole-core calculations are tabulated in \cite{mox-bench}. One of the two Czech standardized codes
for reactor calculations -- ANDREA \cite{ANDREA} -- also falls into this category (the other -- MOBY-DICK
\cite{MOBYDICK} -- is based on a finite-difference method on a structured mesh)\label{ftn:nodal}}.
Generally speaking, a nodal method is a coarse mesh finite volume method iteratively combined with fine-level correction steps, specially tailored to the
neutron diffusion (or recently $\SPN$) model and reactor core domain (i.e., typically, coarse level cells
correspond to real fuel assemblies and the correction consists of analytic solution of the diffusion equation in a
geometrically simple homogeneous region). Its advantage is the speed and overall efficiency, but it is greatly limited
in geometrical flexibility. Because of the way the nodal equations are derived, it also requires the
\textit{homogenization} procedure to represent each coarse cell by a single set of material coefficients and the 
corresponding \textit{dehomogenization} procedure to reconstruct the fine structure of the solution needed for further 
computations\footnote{We note that in our verification examples in Chap. \cref{chap:coupled}, we also use homogenized
input data (provided as part of the benchmarks) and we expect such data to be used also during the optimization runs for the
sake of efficiency; however, it is not the \textit{requirement} of the code.} (where the latter, in particular, is difficult to formulate in general cases).
Moreover, the convergence and stability of the method is, to the knowledge of the author, not very well understood 
(\cite{ZiminComm}).

Although the author also participates in the development of a nodal module for the code MOBY-DICK (see footnote
\ref{ftn:nodal}; \cite{Hanus2}, \cite[Chap. 4]{Hanus3}), only finite-element approximation of the diffusion and
 $\SPN$ models will be considered in this thesis. In fact, one of the objectives of the thesis is to show the
feasibility of using finite element approximation in a quasi-steady state whole-core calculation code. In order to
provide a viable alternative to coarse nodal and structured finite-difference discretizations, which often incorporate
hard-coded formulas also for the solutions of the associated discrete algebraic systems, a need for automatic mesh
adaptivity and fast algebraic solvers has been recognized from the beginning. Rather than developing an adaptive finite
element code from scratch, it has been decided to modify an existing library for the needs of neutron transport
modelling. This led the author to the finite-element C++ framework Hermes (\cite{Hermes-project}) with advanced
\textit{hp-adaptivity} capabilities (\cite{Hermes-hanging-nodes}) and a unique way of assembling coupled systems of PDEs (\cite{Hermes-thermoelasticity}).
Author's main contributions to the Hermes project involved:
\begin{itemize}
    \item development of an interface for various existing sparse,
direct and iterative algebraic solvers (which also required reworking the CMake build system of Hermes),
	\item development of a multigroup neutron diffusion framework, simplifying and unifying the formulation of
	multiregion, multigroup neutron diffusion problems within Hermes,
	\item development of the discontinuous Galerkin framework (together with L. Korous, the main developer of Hermes at
	present time) and
	\item extension of the h-adaptivity capabilities by the standard a-posteriori error estimation for elliptic problems (which
involves solution jumps over element interfaces and thus uses elements of the discontinuous Galerkin framework). 
\end{itemize}
More details on these contributions will be given in \cref{chap:hermes}, where also the extension of the multigroup
neutron diffusion framework to include the $\SPN$ model (for orders $N \leq 9$, but easily extendable to an arbitrary
order) will be presented, together with specially tailored error estimate to drive the hp-adaptivity of Hermes. To
assess the benefits of using the $\SPN$ model over the simpler diffusion model, the author also implemented (still on
top of the neutronics framework) a discontinuous Galerkin discretization of the \textit{discrete ordinates}
approximation of the transport equation (the $\SN$ method, introduced in \Sref{sec:1-SN}), which, theoretically as $N\to\infty$, converges to the true solution of
the NTE (unlike the $\SPN$ approximation in multidimensional, multiregion domains). This will be addressed in Sec.
\sref{sec:DO}.

\comment{
\myparagraph{3D coupled neutron-physical finite element code based on the multigroup diffusion approximation}
Unfortunately, the 3D version of Hermes has not yet been released and thus the main utility of the neutronics framework
introduced in previous paragraph is in testing new ideas on hp-adaptivity for neutron transport problems. For the
purposes of the research project underlying this thesis (see the footnote \ref{ftn:TACR} on pg. \pageref{ftn:TACR}), a
3D finite element framework was needed. The final choice has been the FEniCS/Dolfin framework (\cite{dolfin1, dolfin2}).
The final code, that the author developed within this framework, is described in \cref{chap:coupled} and demonstrated on
several industrial and custom benchmarks. It allows parallel assembly of the multigroup neutron diffusion problem and
solution of the obtained algebraic problem using the well-established PETSc/SLEPc solvers (\cite{petsc1, slepc1})
wrapped by FEniCS.
This can be repeated in a feedback loop (until a steady state is found), in which the computed neutron flux directly influences
thermal/hydraulic properties of the core, the change of which in turn leads to a change of coefficients in the diffusion
equations. On top of that loop, another loop representing fuel burnup can be executed. At this point, the author would
like to acknowledge the work of his colleagues -- R. Ku{\v z}el (the coordinator of the whole effort and also the author
of a GPU eigensolver module), J. Egermaier and H. Kopincov{\' a} (who implemented the thermal/hydraulics
module) and Z. Vastl (who generated the meshes for the MOX-UO$_2$ and VR-1 benchmarks).

%\myparagraph{Well-posedness of the $\SPN$ equations}
%While there are many papers where the finite element method has been \textit{used} to find the weak solution of
%both diffusion and $\SPN$ equations (e.g., \cite{Ragusa1, Hermes-nuclear, Ragusa2}), the question of well-posedness of
%the corresponding variational formulation is neither addressed in these papers, nor do they provide references that
%answer it. In the case of multigroup diffusion, well-posedness has been proved in \cite{Bourhrara1}. However, the
%% author couldn't find any paper dealing with the well-posedness of the $\SPN$ equations. It is therefore proved in
% Sec.
%\alert{ref}.

\myparagraph{New method for solving large eigenvalue problems}
Determination of reactor steady state requires finding the dominant eigenpair of a generalized eigenvalue problem. This
has been traditionally performed by the simple power method, accelerated by eigenvalue shifting or Chebyshev
combination of previous iterates. Recently, Krylov subspace methods have been applied (\cite{warsa}, \cite{Subramanian},
or basically any reference from the ``Nuclear Engineering'' section at
\url{http://www.grycap.upv.es/slepc/material/appli.htm}). While the SLEPc library also serves as the main eigenvalue
solver of our neutron-physical code, there is another eigensolver module that is based on a new method
proposed by author's colleagues R. Ku{\v z}el and P. Van{\v e}k. Even though the implementation is not yet competitive with the well-optimized
solvers from the SLEPc library (in terms of execution speed), the method itself has some interesting properties
connected with solving large-scale eigenvalue problems. The paper describing this development has not yet been 
published, so it is included here as \cref{chap:evc} with minor corrections and comments. This last chapter also
includes numerical experiments on simpler problems, which confirm the theory presented in the paper.
}
\newpage


% B I B L I O G R A P H Y

\addcontentsline{toc}{chapter}{Bibliography}
\bibliographystyle{alphaurl}
\bibliography{text}

\appendix
%\input{Appendix.tex}

\end{document}
