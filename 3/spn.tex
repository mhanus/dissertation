\chapter{The simplified $\PN$ approximation}\label{chap:SPN}

The simplified $\PN$ ($\SPN$) approximation was proposed in the early 1960's by E. Gelbard \cite{Gelbard1,Gelbard2}) to
circumvent the problem of increasing complexity of the $\PN$ approximation in multiple dimensions. Its derivation was
completely formal at the beginning -- amounting to a simple replacement of differential operators $\textstyle\der{}{z}$
in the 1D $\PN$ system by their multidimensional counterparts $\nabla$ and $\nabla\cdot$ and recasting those scalar
unknowns operated upon by the latter as vector quantities.
Despite this mathematically weak derivation, the $\SPN$ solution has been found to be equivalent to the solution of the
multidimensional $\PN$ equations in the case of a homogeneous medium and, comparing to either diffusion or $\PN$ models,
provided encouraging results both in terms of accuracy and efficiency even in more realistic cases. This is a rather
remarkable fact -- as we will see below, the $\SPN$ approximation for odd $N$ consists of $2N-1$ coupled elliptic
partial differential equations (and reduces to the diffusion approximation for $N = 1$), which is significantly lower
than the $(N+1)^2$ equations of the full $\PN$ model (and also than the $N(N+1)/2$ strongly coupled elliptic equations
to which the full $\PN$ model can be reduced). The method has thus become particularly attractive as its implementation
required only modification of existing multigroup diffusion codes.

After some time, however, special transport problems for which the simple diffusion
approximation actually provided better results have been contrived (see, e.g., \cite[p. 247]{Coppa1}). Validity of
Gelbard's formal derivation therefore became questioned and the $\SPN$ equations have not been seriously considered as 
a robust enough improvement of the diffusion model for some time. This has changed in the 1990's with the extension of
the asymptotic analysis originally performed for the diffusion approximation. Larsen, Morel and McGhee
\cite{Larsen1} have shown that under the scaling of the transport equation by a ``diffusivity'' parameter $\varepsilon$
that makes the diffusion equation agree with the transport equation up to terms of order $\Oh(\varepsilon^3)$ as 
$\varepsilon \to 0$ (as already discussed in \sref{sec:diffusive}), the $\SPN[3]$ equations are
equivalent to the transport equation up to terms of order $\Oh(\varepsilon^7)$ provided that that the transport solution shows a nearly 
one-dimensional behaviour in the vicinity of interfaces of different materials by having there sufficiently weak
tangential derivatives. The approach used by the authors is sufficiently general to show
that $\SPN$ equations of increasing order provide asymptotic corrections of the NTE of increasing order, which has been
confirmed at least experimentally (e.g., \cite{Olbrant} or \cite{McClarren1}).

Brantley and Larsen \cite{Brantley1} contributed to the theoretical justification of the method by
variational analysis in which they showed that the $\SPN[3]$ equations are the approximate Euler-Lagrange equations
whose solution makes stationary a special but physically reasonable functional characterizing arbitrary reaction rates. By
including boundary terms in the functional, the authors also arrived at natural boundary conditions for the method,
missing from the asymptotic approach. At internal interfaces, however, an assumption of one-dimensional behavior of
solution was again required as in the asymptotic derivation of Larsen et al.

Together with other asymptotic and variational analyses (e.g. \cite{Pomraning1}), the range of validity of the
approximation had been finally determined by the end of the 1990's. Although it turned out that this range is not
significantly larger than that of the diffusion theory (\cite{Larsen1}), the $\SPN$ approximation has recently been shown to produce more 
accurate results than the diffusion model under these conditions and regained attention
\cite{Frank1,McClarren1,Ragusa1,Larsen3,Kirschenmann1,Olbrant}.

The $\SPN$ method is particularly suitable for solving reactor criticality problems, where the asymptotic conditions
predominantly hold. Downar \cite{Downar} compared the \SN[16], \SPN[3] and diffusion approximations over several model
problems, with results that the \SPN[3] method well agrees with the high-order transport solution of the \SN[16] method
and provides more than 80\% improvement in reactor critical number and 50\% to 30\% improvement in pin
powers\footnote{eq. \eqref{eq:power} integrated over elementary cells comprising fuel assemblies} over the diffusion
approximation.
Somewhat smaller but still well-noticeable improvement has been obtained by Brantley and Larsen in \cite{Brantley1}.
Similarly to Downar and others, however, they conclude that \SPN[3] captures most of the transport effects in diffusive
regimes of nuclear reactors (and that higher orders than 3 are not usually necessary, as also shown e.g. by Cho et al.
in \cite{ChoAxialSPN}).
The authors also warned, however, that more careful spatial discretization than in the
diffusion methods is required in order to capture the sharper boundary layer behaviour of the more transport-like
\SPN[3] approximation.

Even though the $\SPN$ solution does not tend to the
exact solution of the NTE as $N\to\infty$ in general, there are several cases in which it is equivalent to the
convergent $\PN$ expansion (see some recent papers like \cite{Coppa2,McClarren2,Larsen2}) and further research of the
$\SPN$ model and its connections to the NTE appears to be an interesting topic. This work contributes to this research
by describing a new way of deriving the $\SPN$ equations from a specially formulated $\PN$ approximation, which will be
the subject of \cref{chap:mcpn}. In this chapter, we will recall the standard derivation and conclude by a simple observation (that
we, however, could not find published as of yet) that allows to prove well-posedness of the weak form of the $\SPN$
equations via the standard Lax-Milgram lemma. 

\section{Derivation of the $\SPN$ equations}
To illustrate the original Gelbard's approach, let us consider the case of one-dimensional symmetry, in which neutron
transport is characterized by neutron distribution that is spatially varying only along one coordinate direction and,
moreover, that is symmetric with respect to rotations about that axis. Without loss of generality, we may choose the
principal direction of variation along the $z$-axis. This situation may arise for example when the system is composed of
slabs, each with homogeneous properties and extents in the $x$ and $y$ directions much larger than in the principal
direction, so that dependence on $x$ and $y$ may be neglected. We will identify the inflow/outflow boundaries $\pV[\pm]$
with points $z_{\pm}$ and interior points $\br\in\VV$ with $z\in (z_-, z_+)$.

Under these assumptions, spherical harmonic functions
reduce to Legendre polynomials in $\mu = \cost = \Omega_z$ and partial derivatives $\pd{}{x}$, $\pd{}{y}$ vanish, so that the set of $\PN$ equations
\eqref{eq:pn1} (where we assume $N$ odd) becomes
\begin{equation}
    \label{eq:PN1d}
\frac{n+1}{2n+1}\der{\angmom{n+1}{}(z)}{z} + \frac{n}{2n +1}\der{\angmom{n-1}{}(z)}{z} + \Sigma_n(z)\angmom{n}{}(z) =
q_n(z),
\end{equation}
where $n = 0,1,\ldots, N$ (discarding the non-sensical moments $\angmom{n}{}$ for negative $n$),
$$
\Sa{n} = \sigma_t - \kappa_n = \sigma_t - \sigma_{sn} - \kron{n}{0}\nu\sigma_f
$$
and the moments are defined as
$$
\angmom{n}{} = \muint{\P{n}{}(\mu)\angflux(\cdot,\mu)},\quad q_n = \muint{\P{n}{}(\mu)q(\cdot,\mu)}
$$
(note that the definition of $\kappa_{n}$ by \eqref{eq:scattering_exp} is still valid with $\mu_0 = \mu\mu'$). 

To proceed as in the derivation of the diffusion equation, we again assume that $\Sigma_n \geq \underline{\Sigma_n} > 0$
for $n = 1,3,\ldots,N$ and $q_n = 0$ for $n \geq 1$. Then by solving the
odd-order equations for the odd-order flux moments in terms of a derivative of the even-order flux moments and using the
result to eliminate the odd-order flux moments from the even-order equations, we obtain the one dimensional $\SPN$ 
equations. To write them in a convenient form, we define the auxiliary $\SPN$ moments:
\begin{equation}\label{eq:SP3_mom}
	\phi^s_n := (n+1) \phi _{n}+(n+2) \phi _{n+2},\quad n = 0,2,\ldots,N-1,
\end{equation}
(setting $\phi_{N+1} = 0$)
and $\SPN$ ``diffusion coefficients''
\begin{equation}\label{eq:SP3_dif}
	D^s_n := \frac{1}{(2 n+1) \Sigma _{n}},\quad n = 1,3,\ldots,N
\end{equation}
so that $\SPN$ currents could be defined as
$$
	J^s_n \equiv \phi_{2n+1} = -D^s_{2n+1}\der{\phi^s_n}{z}, \quad n = 0,2,\ldots,N-1.
$$
Notice that $J^s_1 = J$ (neutron current in the one-dimensional symmetry) and that scalar flux is given by
\begin{equation}\label{eq:SPN_scalar_flux}
	\phi_0 = \phi^s_0 - {\frac23}\phi^s_2 + {\frac{8}{15}}\phi^s_4 - \ldots = \suma[n]{0}{(N+1)/2}{F_{n}}\phi^s_{2n},\quad
	F_{n} = (-1)^n\frac{2^nn!}{(2n+1)\text{!!}}
\end{equation}
where
$(2n+1)!! = (2n+1)(2n-1)\cdots 3\cdot 1$.

One may note that these definitions are somewhat arbitrary and indeed, there have been several ``$\SPN$ approximations''
reported in literature. We have compared several of them and found that they are all equivalent\footnote{For instance,
going from ``our'' $\SPN[3]$ system to that used by Brantley \cite{Brantley1} for his variational analyses (which is
actually the same as that used by Larsen, Morel and McGhee in \cite{Larsen1} for asymptotic analyses) amounts to multiplying the
first $\SPN[3]$ equation of Brantley by $5/9$ and the second by $3$ and the use of \eqref{eq:SP3_mom} (and
analogously for the boundary conditions).}.
The formulation obtained with the above definitions is particularly convenient as it allows to easily obtain 
well-posedness of its corresponding weak form (at least under some additional constraints on the higher-order
anisotropic scattering moments).


\subsection{The \SPN[3] case}
As the practical usefulness of the $\SPN$ equations has been experimentally verified to be limited by orders up
to around $N = 7$ (we refer to above mentioned papers and reports), we do not delve into technical derivation of the
general form of the $\SPN$ equations here and rather consider the case $N=3$ (with cases $N = 5,7$ included in App.
\ref{app:SPN} \footnote{The equations were generated by a simple Mathematica script that can
be used for any reasonably high order $N$ if needed.}). 

The one-dimensional $\PN[3]$ system reads 
\begin{equation}
    \label{eq:P3_1d}
	\begin{aligned}
\der{\phi_1(z)}{z} + \Sigma_{0}(z)\phi_0(z)&=q_0(z),\\[.25em]
\frac{1}{3} \,\der{\phi_0(z)}{z} +
\frac{2}{3} \,\der{\phi_2(z)}{z} + \Sigma_{1}(z)\phi_1(z)&=q_1(z),\\[.25em]
\frac{2}{5} \,\der{\phi_1(z)}{z} + 
\frac{3}{5} \,\der{\phi_3(z)}{z} + \Sigma_{2}(z)\phi_2(z)&=q_2(z),\\[.25em]
\frac{3}{7} \,\der{\phi_2(z)}{z} + \Sigma_{3}(z)\phi_3(z)&=q_3(z)
	\end{aligned}
\end{equation}
and the Marshak approximation of albedo boundary conditions \eqref{eq:nte3}
\begin{equation}\label{eq:Marshak}
\begin{aligned}
\frac{\phi _0(z_{\pm})}{4}+\frac{5 \phi _2(z_{\pm})}{16}\mp\frac{\phi _1(z_{\pm})}{2} &= \alpha(z_{\pm})
\left[\frac{\phi _0(z_{\pm})}{4}\pm\frac{\phi _1(z_{\pm})}{2}+\frac{5 \phi _2(z_{\pm})}{16}\right]\\
-\frac{\phi _0(z_{\pm})}{16} +\frac{5 \phi _2(z_{\pm})}{16}\mp\frac{\phi _3(z_{\pm})}{2} &= \alpha(z_{\pm})
\left[-\frac{\phi _0(z_{\pm})}{16} +\frac{5 \phi _2(z_{\pm})}{16}\pm\frac{\phi _3(z_{\pm})}{2}\right].
\end{aligned}
\end{equation}


Using the approach described above together with the auxiliary $\SPN[3]$ definitions, we obtain the following
one-dimensional $\SPN[3]$ system
\begin{equation}\label{eq:sp3_1d}
\begin{gathered}
	-\der{}{z} \mathbf D^s(z) \der{}{z} \Phi^s(z) + \mathbf{C}^s(z) \Phi^s(z) = \mathrm{Q}^s(z), \quad z \in
	(z_{-},z_{+})\\
	\mat{D}^s(z) \der{}{z} \Phi^s(z) + \gamma(z_{\pm}) \mat{G}^s \Phi^s(z_{\pm})
	\quad
	\gamma(z\pm) =
	\frac{1-\alpha(z_{\pm})}{2(1+\alpha(z_{\pm}))}
\end{gathered}
\end{equation}
where $\gamma$ is the same albedo coefficient as in the diffusion case \eqref{eq:diffusion_bc} and
\begin{equation}\label{eq:SP3_mat}
\begin{gathered}
	\Phi^s = [\phi^s_0, \phi^s_2]^T,\quad \mathrm{Q}^s = [q_0, -\tfrac{2}{3} q_0]^T, \quad 
	\mat{D}^s = \diag \left\{\frac{1}{3\Sigma_1}, \frac{1}{7\Sigma_3}\right\},\\[.3em]
	\mat{C}^s = \left[
\begin{array}{cc}
 \Sigma _0 & -\frac{2 \Sigma _0}{3} \\
 -\frac{2 \Sigma _0}{3}  & \frac{4 \Sigma _0}{9}+\frac{5 \Sigma _2}{9}
\end{array}
\right],\quad 
\mat{G}^s = \left[
\begin{array}{cc}
 1 & -\frac{1}{4} \\
 -\frac{1}{4} & \frac{7}{12} \\
\end{array}
\right].
\end{gathered}
\end{equation}

Using the Gelbard's ad-hoc approach, the multidimensional equations are just
\begin{equation}\label{eq:sp3}
\begin{gathered}
	-\nabla\cdot \mat{D}^s(\br) \nabla \Phi^s(\br) + \mat{C}^s(\br) \Phi^s(\br) = \mathrm{Q}^s(\br), \quad \br \in
	\VV,\\
	\bn(\br)\cdot\mat{D}^s(\br) \nabla \Phi^s(\br) + \gamma(\br) \mathbf{G}^s \Phi^s(\br), \quad \br\in\pV,
\end{gathered}
\end{equation}
where $\nabla\Phi^s$ is the Jacobian matrix of $\Phi^s$:
$$
	\left[\nabla\Phi^s\right]_{i,\alpha} = \pd{\phi^s_{2i-2}}{x_\alpha},\quad i = 1,2
$$
(recall the convention that Greek letters index the Cartesian coordinate axes), 
$$
	\nabla = \left[\pd{}{x}, \pd{}{y}, \pd{}{z}\right],\ \bn = [n_x, n_y, n_z]
$$
and for $\bv = [v_x, v_y, v_z]$
$$
	\bv \cdot \mat{A} = \sum_{\alpha=1}^3 v_\alpha A_{i\alpha}.
$$

Let $\mathbb{H}^1(\VV) =
[H^1(\VV)]^2$. To introduce some new notation, let us write the inner product on this space as
\begin{equation}\label{eq:ipHH1}
	(\mathrm{U}, \mathrm{V})_{\mathbb{H}^1(\VV)} = \int_{\VV}\left(\nabla\mathrm{U} : \nabla\mathrm{V} + U\cdot
	V\right)\d{\br};
\end{equation}
here $\cdot$ denotes the usual inner product and $:$ the double inner product of matrices:
$$
	\mat{A} : \mat{B} = \sum_{i,j}A_{ij}B_{ij}.
$$
The weak formulation forming the basis for the finite
element solution can now be stated as follows:
\begin{problem}\label{prb:sp3}
Given $q_0 \in \Lp[2](\VV)$, find $\Phi^s = \col \{\phi^s_0, \phi^s_2\} \in \mathbb{H}^1(\VV)$ such that
\begin{equation}\label{eq:spn_weak}
\begin{gathered}
	a(\Phi^s, \mathrm{V}) = f(\mathrm{V}) \quad \forall \mathrm{V}\in \mathbb{H}^1(\VV),\\[.3em]
	a(\mathrm{U}, \mathrm{V}) := \int_{\VV} \bigl(\mat{D}^s\nabla\mathrm{U} : \nabla \mathrm{V} +
	\mat{C}^s \mathrm{U}\cdot\mathrm{V}\bigr)\d{\br} + \mat{G}^s \int_{\pV} \gamma\mathrm{U}\cdot\mathrm{V}
	\,\d{S},\\
 	f(\mathrm{V}) = \int_{\VV}\mathrm{Q}^s\cdot\mathrm{V}\,\d{\br},\quad \bigl(\mat{D}^s\nabla\mathrm{U}\bigr) : \nabla
 	\mathrm{V} = \sum_{i,\alpha}D^s_i\pd{u_i}{x_\alpha}\pd{v_i}{x_\alpha}
\end{gathered} 
\end{equation}
\end{problem}

Note that eq. \eqref{eq:spn_weak} represent a set of weakly coupled diffusion equations. The case of $N=1$ also reduces
to the weak form of the usual diffusion approximation, eq. \eqref{eq:dif_bil_ex}. We also note that in the multigroup
approximation of energetic dependence, the diffusion approximation has the same form \eqref{eq:sp3} (with weak form
\eqref{eq:spn_weak}), with 
$$
	\mat{D} = \diag\{D^g\}_G,\quad
	[\mat{C}]_{gg'} = \sigma_t^g \delta_{gg'} - \sigma_{s}^{gg'} - \chi^g\nu\Sigma_f^{g'}, \quad
	[\gamma\mat{G}]_{gg'} = \gamma^{gg'} 
$$
where $g,g' = 1,2,\ldots,G$. The extension to the multigroup $\SPN$ case is obvious, with the weak formulation posed in 
$\mathbb{H}^1(\VV) = [H^1(\VV)]^{(2N-1)\times G}$.

In the following subsection, we will study the properties of the coupling matrices of the (mono-energetic) $\SPN[3]$
method, which allow us to establish well-posedness of Problem \ref{prb:sp3}.

\subsection{Well-posedness of the $\SPN[3]$ formulation}


The $\SPN[3]$ matrices $\mat{D}^s$, $\mat{C}^s$ and $\mat{G}^s$ are symmetric, with elements bounded
a.e. in $\VV$, hence the bilinear form $a(\mathrm{U}, \mathrm{V})$ is also bounded on $\mathbb{H}^1(\VV)$ (with norm
induced by the inner product \eqref{eq:ipHH1}; for the boundary term,  recall that $\gamma \in [0,0.5]$ with value $0$ at perfectly 
reflecting boundary and value $0.5$ at vacuum boundary and use the standard trace inequality in
$H^1(\VV)$ on each term in the sum of boundary integrals). The linear form $f$ is also obviously bounded when the
isotropic source term $q_0 \in \Lp[2](\VV)$.

For coercivity, note that the matrix $\mat{D}^s$ is positive definite, as is $\mat{G}^s$ (being symmetric and strictly
diagonally dominant, it follows from the Gerschgorin theorem). 

To show positive-definiteness of $\mat{C}^s$, let us
again split 
$$
	\Sigma_n = \sigma_t - \kappa_n.
$$
Then 
\begin{equation}\label{eq:Cs}
	\mat{C}^s = \sigma_t\mat{C}^s_t - \kappa_0 \mat{C}^s_0 - \kappa_2 \mat{C}^s_2,
\end{equation}
where
\begin{equation}\label{eq:Cs2}
\begin{gathered}
	\mat{C}^s_0 = \left[
\begin{array}{rr}
 1 & -\frac{2}{3} \\
 -\frac{2}{3} & \frac{4}{9} \\
\end{array}
\right],\quad
\mat{C}^s_2 = 
\left[
\begin{array}{rr}
 0 & 0 \\
 0 & \frac{5}{9} \\
\end{array}
\right],\\
\mat{C}^s_t = \mat{C}^s_0 + \mat{C}^s_2 = \left[
\begin{array}{rr}
 1 & -\frac{2}{3} \\
 -\frac{2}{3} & 1 \\
\end{array}
\right]
\end{gathered}
\end{equation}
As the Legendre polynomials $P_n(\mu_0) < 1 = P_0(\mu_0)$ a.e. in $[-1,1]$ and fission part of the collision kernel
$\kappa$ is isotropic, we have for $n \geq 1$
\begin{equation}\label{eq:kappa_inequality}
	\kappa_n = \sigma_{sn} = 2\pi \muint[_0]{\sigma_s(\mu_0)\P{n}{}(\mu_0)} < 2\pi
	\muint[_0]{\sigma_s(\mu_0)} = \sigma_{s0} < \sigma_{s0} + \nu\sigma_f = \kappa_0. 
\end{equation}
If we now define the relation ``$<$'' between matrices as
$$
	\mat{A} < \mat{B}\quad \Leftrightarrow \quad \mat{x}^T \mat{A} \mat{x} < \mat{x}^T \mat{B} \mat{x},\ \ \forall \mat{x}
	\neq \mat{0}, 
$$
we have from \eqref{eq:kappa_inequality}, \eqref{eq:Cs} and \eqref{eq:Cs2}
$$
	\mat{C}^s > \sigma_t\mat{C}^s_t - \kappa_0(\mat{C}^s_0 + \mat{C}^s_2) = (\sigma_t - \kappa_0)\mat{C}^s_t.
$$
Under the subcriticality conditions in $\Lp[2](\XE)$ (Def. \ref{def:subcriticality}), we have 
$$
	\sigma_t > \sigma_s + \nu\sigma_f = \kappa_0
$$
(cf. the representation \eqref{eq:scattering_ratio}). As the matrix $\mat{C}^s_t > 0$ (again because its strict
diagonal dominance, symmetry and positivity of diagonal elements), we have thus proved that $\mat{C}^s$ is also positive
definite. From the Lax-Milgram lemma \ref{lem:lax-milgram}, it now directly follows

\begin{theorem}
	Let the subcriticality conditions in $\Lp[2](\XE)$ hold. Then Problem \ref{prb:sp3} has a unique solution and there
	exists constant $\alpha > 0$ such that
	$$
		\norm[{\Hp[1](\VV)}]{\Phi^s} < \frac{1}{\alpha} \norm[{\Lp[2](\VV)}]{\mathrm{Q}^s}
	$$
\end{theorem}

We note that for higher order $\SPN$ approximations, the previously stated properties for $\mat{D}^s$, $\mat{G}^s$ are
still valid and the decomposition \eqref{eq:Cs} has the following form:
$$
	\mat{C}^s = \sigma_t\mat{C}^s_t - \kappa_0 \mat{C}^s_0 - \kappa_2 \mat{C}^s_2 - \kappa_4 \mat{C}^s_4 \ldots -
	\kappa_{2N-1} \mat{C}^s_{2N-1}.
$$
The matrix 
$$
	\mat{C}^s_t = \sum_{n=0}^{\floor{N/2}} \mat{C}^s_{2n}
$$
is no longer strictly diagonally dominant, but it is still positive definite, as can be verified
by explicitly computing the eigenvalues (see App. \ref{app:SPN} for the cases $N = 5,7$).

