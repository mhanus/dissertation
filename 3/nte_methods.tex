\ifpdf
	\graphicspath{{3/pic/PNG/}{3/pic/PDF/}{3/pic/}}
\else
	\graphicspath{{3/pic/EPS/}{3/pic/}}
\fi

\chapter{Numerical methods for neutron transport}\label{chap:nte-methods}

As mentioned in the introductory chapter, we will focus on deterministic methods for solving the NTE, requiring proper
discretization of \eqref{eq1} and solution of the resulting system of algebraic equations. 
In the following sections, we review some of the most widely used (in author's opinion) semi-discretizations with respect to
energy, angle and spatial variables and finish this chapter with a general discussion on solving large sparse systems of
algebraic equations, resulting from the combination of semi-discretizations that are most suitable for our purposes. We
will focus on the fixed-source problem, whose solution is a neccessary part in practically all numerical methods for
solving the generalized eigenvalue problem \eqref{eq:critical}.

\section{Approximation of energetic dependence}

The continuous dependence on energy, $\psi = \psi(\cdot, E)$, is typically resolved by the so called \textit{multigroup
approximation}. In this approximation, the interval of neutron energies is divided as follows\footnote{Note that the
energy intervals (groups) are numbered in a descending order, i.e. a group with larger index contains lower energies than a group with
lesser index.}:
$$
\begin{multlined}
  \bigl[\Emin,\Emax] = \bigl[E^{G}-\dEh{G},E^{G}+\dEh{G}\bigr]\cup \ldots\\
  \ldots \cup \bigl[E^{g}-\dEh{g}, E^{g}+\dEh{g}\bigr] \cup \ldots \cup
  \bigl[E^1-\dEh{1},E^{1}+\dEh{1}\bigr],
\end{multlined} 
$$
and equations \eqrefs{eq:nte1}{eq:nte3} are integrated over each energy group range 
\linebreak
\mbox{$\bigl[E^{g}-\dEh{g}, E^{g}+\dEh{g}\bigr]$}.
The NTE \eqref{eq1} is thus transformed into a finite system of integro-differential
equations, each governing the flux of neutrons with energies within a particular range (in this context called
\textit{group}):
\begin{equation}\label{eq:psi-MG}
\begin{multlined}
  \psi^g(\br, \bomega) = \frac{1}{\Delta E^{g}}\int_{g}\psi(\br, \bomega, E),\d{E} \equiv
  \frac{1}{\Delta E^{g}}\int_{E^g-\Delta E^{g}/2}^{E^g+\Delta E^{g}/2}\psi(\br, \bomega, E),\d{E},\\ g = 1, 2,\ldots
  G.
\end{multlined}
\end{equation} 
This conventional procedure leads to the following set of $G$ coupled neutron transport equations
\begin{equation}\label{eq:mg}
	\left\{
	  \begin{aligned}
      &T_G\{\psi^g(\br,\bomega)\} = \{q^g(\br,\bomega)\},\\
      &\Dom{T_G} = \bigl\{\{\psi^g\}\in \bigl[\Hp[1](X)\bigr]^G,\ \psi^g\vert_{\pX[-]} = 0,\ g = 1,\ldots,G\bigr\},
    \end{aligned}
  \right.\\[.2em]
\end{equation}
where the variable sets of $G$ elements are understood: $\{f^g\} \equiv \{f^g\}_{g=1}^G$. The multigroup transport
operator has the following form:
\begin{equation*}
\begin{gathered}
    T_G\{\psi^g(\br,\bomega)\} = \left\{\left(A + \Sigma_r^g\right)\psi^g(\br,\bomega) - \summ{g'=1,g'\neq g}{G}
    K^{gg'}\psi^{g'}(\br,\bomega)\right\},\\
    \Sigma_r^g = \sigma_t^g(\br) - \intA[']{\kappa^{g\sla g}(\br,\bomega\cdot\bomega')},\quad K^{gg'} = 
    \intA[']{\kappa^{g\sla g'}(\br,\bomega\cdot\bomega')}
\end{gathered}
\end{equation*}
where the terms with superscript $g$ or $g'$ represent quantities suitably averaged over 
\mbox{$\bigl[E^{g}-\dEh{g}, E^{g}+\dEh{g}\bigr]$}, e.g. $K^{gg'}$ is (in theory) obtained as
\begin{equation}\label{eq:mg-kappa}
	\kappa^{gg'}(\br,\bomega\cdot\bomega') = \frac{\int_{g}\int_{g'} \kappa(\br,\bomega\cdot\bomega',E \sla
	E')\psi(\br,\bomega,E')\,\d{E'}\d{E}}{\int_{g}\psi(\br, \bomega, E)\,\d{E}}.
\end{equation}
It is customary to move the \textit{self-scattering} (diagonal) part of the
transfer operator to the collision operator. Since the reactions in which energetic distribution of both the incoming 
and outgoing neutrons lies within the same group are included in both $\sigma_t$ and $\kappa$  (compare equations
\eqref{eq:ddifxs} and \eqref{eq:st}), this transformation makes $\Sigma_r^g\psi^g$ represent the actual rate of neutron
removal from the group, while $K^{gg'}\psi^{g'}$ the rate of neutron addition to that group.

Although the multigroup system of neutron transport equations has a relatively simple form, finding an optimal grouping
of energies and determining the associated group-averaged coefficients is not an easy task in most practical
applications because of the highly complicated energetic dependence of nuclear processes, as illustrated by the
typical dependence of the (microscopic) fission cross-section of \isotope[235][92]{U} in the so-called resonance range
of energies and corresponding multigroup approximation in Fig. \ref{fig:xs}.
\begin{figure}[htp]
\begin{center}
  \includegraphics[scale=.4]{U235fg}
  \caption{Microscopic fission cross-section of U235}
  \label{fig:xs}
\end{center}
\end{figure}
Suitable approximation of the unknown exact solution in \eqref{eq:mg-kappa} is also highly non-trivial, albeit essential
for the success of the multigroup method. Even though an alternative to the finite-volume like approximation
\eqref{eq:psi-MG} has been proposed recently  in \cite{Douglass} -- using Galerkin projection of angular flux onto a
space of functions supported over subregions of the energy range (a finite-element like approach) --  the multigroup approximation still remains the most universally used approach to simplify the energetic dependence (see, e.g., \cite[Chap.~5]{Cacuci1} or \cite{Cho1}). However, we will
not specifically address this issue and always assume that the multigroup coefficients appearing in the equations 
are given as input.

A standard way of iterative solution of the multigroup system is the \textit{source iteration}:
$$
  \left(A + \Sigma_r^g\right)\psi^g_{(i+1)} = \summ{g'\leq g-1}{} K^{gg'}\psi^{g'}_{(i+1)} + \summ{g' \geq g+1}{}
  K^{gg'}\psi^{g'}_{(i)} + q^g,\quad g = 1,\ldots,G,\ \ i = 0,1,\ldots
$$ This is a Gauss-Seidel iteration for the operator equation \eqref{eq:mg}, where the matrix operator $T_G$ has been
split into its lower-triangular part $A + \Sigma_r^g - K^{gg'}$ ($g'\leq g$) and its upper triangular part $K^{gg'}$
($g'> g$) and the lower triangular part is being inverted by forward substitution.
Note that a mono-energetic transport problem in group $g$ has to be solved in each iteration, and if neutron advection
can be approximated by a symmetric operator $\tilde A$, the problem would become symmetric with implications for
efficient numerical solution (see sections \ref{sec:second-order} and \ref{sec:algebraic} below). Convergence of this 
scheme can become slow when the upper triangular part (representing neutron up-scattering from lower energies to 
higher energies) is dominating. Therefore, when preparing the multigroup data, an effort is also put into finding
such an energy grouping that minimizes the up-scattering\footnote{Note that we assume here that the mono-energetic
problem can be solved exactly. When transport approximations that take into account angular dependence are used for
this task (see the sections below), another iteration level is introduced to resolve scattering-induced angular coupling
of the within-group fluxes. This iteration can also become very slow if scattering of neutrons with given energy dominates
their absorption (for more details, see e.g. \cite[Chap. 1]{Azmy1}, or \cite[Sec. VII.A]{Adams})}.

In the remainder
of this chapter, we will focus on the approximation of neutron flux in a single
group (index of which will be omitted), described by the corresponding within-group equation in which contributions from
other groups are encapsulated in the source term $\src$.

\section{Approximation of angular dependence}

A big number of methods have been proposed for approximating the angular dependence of neutron
flux. Many of them are still used and actively developed today as their characteristics make them more suitable for one
application area than other methods, which are preferred in different areas.

\subsection{Lattice calculations} \label{sec:lattice}
As a first example, we consider the class of
methods originally derived from the equivalent integral form of the NTE (see \sref{sec:advection}). Typical
representatives of this class are the method of collision probabilities or the method of characteristics (see e.g.
\cite{Cho2,Wu1,Hursin1,Petkov1,Sanchez1}). As the integral form of the NTE represents the global neutron balance over
the domain, the corresponding algebraic systems (obtained after spatial discretization) are full and
their solution demanding on computer resources. On the other hand, these methods quite naturally handle complex
geometries. Taking into consideration smaller geometric features of the domain, we are effectively coming from a
macroscopic scale to a mesoscopic range where the neutrons direction of motion as well as their kinetic energy become
more significant. High degree of spatial coupling and the requirement of fine resolution of angular and energetic
dependence does not make these methods suitable for whole-core reactor calculations.
However, these small-scale,
high-fidelity calculations\footnote{called \textit{lattice calculations} as they are typically performed on a single
representative subdomain of the core (one or several neighboring assemblies of fuel pins, or the fuel pin itself)
with reflective boundary conditions, simulating an infinite lattice of such subdomains} are indispensable for
generating appropriately averaged coefficients for the computationally more feasible larger scale calculations.
This \textit{spatial homogenization} and \textit{energy group condensation} (already mentioned in the previous
section), as these averaging procedures are traditionally called in nuclear engineering field, are employed by many existing whole-core simulators (see e.g.
\cite[Chap. 17]{Reuss1} or the review in first two sections of \cite{Sanchez7}). To simulate a long-term nuclear reactor
operation, it is furthermore neccessary to perform these procedures under varying physical conditions of the core and
generate many sets of averaged coefficients corresponding to these conditions. The code system described in
\cref{chap:coupled} expects these coefficient sets on input, i.e., it is not designed for lattice calculations.

\vspace*{1em}
More suitable for whole-core calculations are methods derived from the integro-differential
version of the NTE, eq.
\eqref{eq1}, which lead to sparse algebraic systems. The most successful and well-established are the method
of discrete ordinates ($\SN$) and the method of spherical harmonics ($\PN$).
Both arise from applying in the
angular domain a classical well known approach for constructing finite numerical
approximations of PDEs. Although in the final code, we ultimately use the lowest order approximation that can be obtained
equivalently from both approaches (the neutron diffusion model) we will briefly introduce the main ideas behind the
$\SN$ and $\PN$ methods and expose their most important properties in the next two subsections. These properties are
generally known, but their origin in mathematical structure of the equations is often overlooked in literature (rare
exceptions will be cited below). 

\subsection{The $\SN$ method}\label{sec:1-SN}
The $\SN$ method uses the collocation approach in which a set of directions (\textit{ordinates})
$\omega = \{\bomega_n\}_{n=1}^{M}$ is chosen and the solution is approximated as:
\begin{equation}\label{eq:sn_approx} 
	\psi(\br, \bomega) \approx 
		\pw{\psi(\br,\bomega_n)}{if $\bomega = \bomega_n$ for $\bomega_n \in \omega$}
		   {0}{if $\bomega \not \in \omega$.} 
\end{equation}
Equation \eqref{eq1} is then evaluated at these $M = M(N)$ isolated directions.
Note that for the traditional direction sets, we have $M = N(N+2)$ if the given problem does not possess any symmetries; the
method of discrete ordinates using such a number of directions is traditionally refered to as the method of discrete
ordinates of order $N$, shortly $\SN$. 

To evaluate the integral term on the right hand side of the equation, the set of directions is accompanied by a
corresponding set of weights $\{w_n\}_{n=1}^{M}$, together defining a quadrature of the sphere $\Sphere.$ The
requirement of accurate evaluation of the integral term of eq.
\eqref{eq1} as well as accurate integration of the angular flux function over the sphere (which defines the
physically important scalar flux \eqref{eq:scalar_flux}) constitutes the main guideline for the choice of
directions and weights. We will return to this matter later in \sref{sec:DO}; for now it suffices to say that for
three-dimensional problems without any symmetries, \mbox{$M = \left|\{\bomega_n, w_n\}\right| = \Oh(N^2)$} (with the
value of $M$ for typically used quadrature sets stated above).

To write  the final result of the $\SN$ approximation, let us first denote for a set $s = \{c_k\}_{k=1}^n$ the column
vector with entries $c_1,c_2,\ldots,c_n$ as $\col s$ and similarly the diagonal matrix whose diagonal is
given by elements of $s$ as $\diag(s)$. Then, denoting
\begin{equation}\label{eq:sn_vecs}
\psi_n(\br) \equiv \psi(\br,\bomega_n), \quad q_n(\br) \equiv
q(\br,\bomega_n),\quad n = 1,\ldots, M	
\end{equation}
we will define the vector of angular fluxes and the vector angular sources as
\begin{equation}\label{eq:sn_vars}
\Psi(\br) := \col\{\psi_n(\br)\}_{\idxset{M}},\quad
\mat{Q}_{S_N}(\br) := \col\{q_n(\br)\}_{\idxset{M}},
\end{equation}
where we used the index set of $\SN$ variables: $\idxset{M} = \{1,2,\ldots,M\}$.
The $\SN$ approximation consists of the following set of $M$ PDEs in spatial domain\footnote{Differentiation and
integration of vector functions (such as the term $\pd{\Psi(\br)}{x}$ in eq. \eqref{eq:sn1}) will be understood
component-wise throughout this thesis.}:
\begin{equation}\label{eq:sn1} 
\mat{A}_{S_N}^x\pd{\Psi(\br)}{x} + \mat{A}_{S_N}^y\pd{\Psi(\br)}{y} +
\mat{A}_{S_N}^z\pd{\Psi(\br)}{z} + \bigl[\sigma(\br)\mat{I} - \mat{K}_{S_N}(\br)\bigr]\Psi(\br) = \mat{Q}_{S_N}(\br),
\end{equation}
where $\mat{I}$ is the $M\times M$ identity matrix and
$$
	\mat{A}_{S_N}^x = \diag\{\Omega_{n,x}\}_{\idxset{M}},\ \mat{A}_{S_N}^y = \diag\{\Omega_{n,y}\}_{\idxset{M}},\
	\mat{A}_{S_N}^z = \diag\{\Omega_{n,z}\}_{\idxset{M}}.
$$
Equation
\eqref{eq:sn1} represents a system of advection-reaction equations with constant advection field given by the angular 
matrices $\mat{A}_{S_N}^x, \mat{A}_{S_N}^y, \mat{A}_{S_N}^z$. The system is weakly coupled through the absolute term
$\mat{K}_{S_N}\Psi$, which is the quadrature representation of the original integral term, but not through the
differential terms. This fact and the relative smallness of the coupling term is utilized in a classical solution
technique for the $\SN$ approximation -- the \textit{source iteration} -- in which the system \eqref{eq:sn1} is fully
decoupled. Each equation is solved separately using any method suitable for an advection-reaction PDE, using $\psi_n$
from previous iteration to evaluate $\mat{K}_{S_N}\Psi$ (both Jacobi and Gauss-Seidel schemes may be used to update
$\Psi$ during the iteration process).

\subsubsection{Ray effects}
A major drawback of the $\SN$ angular approximation comes from the fact that it assumes the radiation to propagate only
in a discrete set of directions. Only in the limiting case of infinite number of directions will the whole phase space
be covered. Otherwise, there will remain under-treated regions, while other regions will receive more radiation in order
to satisfy the global balance. This will lead to spatial oscillations of the scalar flux (obtained as a weighted sum of
 contributions from the individual directions) known as the \textit{ray effects}. This problem may be seen to be the
consequence of weak coupling between the unknowns in the $\SN$ system, which is however one of its biggest strenghts
from the computational point of view\footnote{This also exhibits the class of problems suffering the most from ray
effects -- namely problems without fission and scattering that lack the coupling term $\mat{K}$} .
It is therefore very difficult to reduce the ray effects while keeping the advantages of the $\SN$ approximation (for an
overview and some heuristic observations, see \cite{Li1}). 

It is usually mentioned in literature that the root cause of ray
effects in the multidimensional $\SN$ approximation is the loss of rotational invariance. Being intuitively obvious,
this statement has not been treated in any more detail (as far as the author knows). Nevertheless, we give here a
mathematical explanation of this fact, as we would like to use the same reasoning later to show the rotational 
invariance of the $\PN$ system, and as it also provides an insight of what actually happens when \eqref{eq1} is 
transformed into the approximate system \eqref{eq:sn1}.

\myparagraph{Rotational invariance of the NTE}
We will say that an operator equation
\begin{equation}\label{eq:op1} 
Au = f
\end{equation}
is rotationally invariant, if $Au = f$ implies $ARu = Rf$ for any operator
$\op{R}$ corresponding to a rotation $\mat{R}\in\R[3\times3]$ of coordinate system around origin:
\begin{equation}\label{eq:def_rot}
\begin{gathered}
\op{R}: f(\br,\bomega) \mapsto f(\mat{R}^T\br,\mat{R}^T\bomega)\\
\mat{R}^T\mat{R} = \mat{R}\mat{R}^T = \mat{I},\quad \det \mat{R} = 1.
\end{gathered}
\end{equation}
Operator $\op{R}$ is defined by its associated rotation matrix $\mat{R}$, which is conventionally characterized as an
element of the special orthogonal group in $\R[3]$. We will also
consider the operator itself to be an element of that group, i.e.
$\op{R} \in \mathrm{SO}(3)$. The following lemma shows that equation \eqref{eq:op1} is rotationally invariant if and
only if its operator commutes with rotations.
\begin{lemma}\label{lemma1}
	$Au = f \ \Rightarrow ARu = Rf\quad \forall R\in \mathrm{SO}(3)$ if and only if $AR = RA$.
\end{lemma}
\begin{proof}
	Sufficiency is obvious by operating with $R$ on both sides of eq. \eqref{eq:op1}.
	We will show neccessity indirectly, i.e. we will assume that there exists $R\in\mathrm{SO}(3)$ such that $AR \neq RA$
	and show that then we can have $Au = f$ but $ARu \neq Rf$. Indeed, assuming $Au = f$ and operating by $R$, we get  
	$$
		Rf = RAu \neq ARu,
	$$
	which concludes the proof.
\end{proof}

If we write eq. \eqref{eq1} in the form of \eqref{eq:op1}:
\begin{equation}\label{eq1op}
	T\psi \equiv (\op{L} - \op{K})\psi = \op{q},
\end{equation} 
where we suppose $\op{T}:V\to V$ for some suitable vector function space $V$ (precise mathematical setting
will be given in \alert{ref}), then it is not difficult to see that 
\begin{equation}\label{eq:commut_NTE}
	\op{R}T = T\op{R}
\end{equation}
(if we also consider $R$ as an operator from $V$ to itself),
provided that the coefficient functions $\sigma$ and $\kappa$ are also invariant under the action of $\op{R}$. This
will be the case, e.g., if both $\br$ and $\mat{R}^T\br$ point into a single homogeneous region where scattering from
$\bomega'$ to $\bomega$ (represented by $\kappa$) depends only on the cosine of the two directions, see (see, e.g.,
\alert{Zweifel}). Therefore, under these conditions, the continuous NTE is rotationally invariant for any admissible
right hand side.

\myparagraph{Operator form of the $\SN$ system}
To see how rotation affects the $\SN$ equations, note first that the $\SN$ approximation concerns only the angular
variable. To reduce clutter, we will therefore drop the spatial variable in the following
discussion.
Consider now an arbitrary $\SN$ approximation specified by the given set of ordinates
$\omega = \{\bomega_n\}_{\idxset{M}}$ and a corresponding set of quadrature weights.
%by using the Dirac measure $\delta_{\bomega}(\bomega')$ with the property
%\begin{equation}\label{eq:dirac}
%	\int_{\Sphere} f(\bomega')\d{\delta_{\bomega}(\bomega')} = f(\bomega).
%\end{equation}
For a vector (of functions of spatial variable) $\mat{F} = \colset{f_n}{\idxset{M}}$, we define the mapping
\begin{equation}\label{eq:map_SN}
\begin{gathered}
\PiSN: \mat{F} \mapsto f\in V_{S_N}\subset V,\\
\bigl(\PiSN \Psi\bigr)(\bomega) = 
\pw{\psi_n}{if $\bomega = \bomega_n$ for $\bomega_n \in \omega$}{0}{if $\bomega \not
	\in \omega$}
\end{gathered}
\end{equation}
where $V_{S_N}$ is a space of functions vanishing (as functions of $\bomega$) everywhere on $\Sphere$ except at $\bomega
\in \omega$. Note that this mapping converts a vector of directional fluxes to a function on $\Sphere$. Corresponding to
it is the mapping that produces the discrete ordinates representation of a function $f\in\Sphere$: 
\begin{equation}\label{eq:map_SN_inv}
	\PihSN f(\bomega) := \col\left\{f_n \right\}_{\idxset{M}}
	 = \col\left\{f(\bomega_n)\right\}_{\idxset{M}}
\end{equation}
(which we implicitely used in \eqref{eq:sn_vars} with $f(\cdot,\bomega) = \psi(\cdot,\bomega)$ and
$f(\cdot,\bomega) = q(\cdot,\bomega)$, respectively)\footnote{This operator is also used to obtain the $\SN$ boundary
conditions from conditions placed on $\psi(\cdot,\bomega)$.}.
With these two mappings, we can rewrite the $\SN$ system \eqref{eq:sn1} in terms of the transport operators from eq.
\eqref{eq1op}:
\begin{equation}\label{eq:sn_op}
	\PihSN \op{L}\PiSN \Psi = \PihSN\tilde K\Psi + \mat{Q}_{S_N},
\end{equation}
where $\tilde K$ hides the $\SN$ quadrature representation of the original integral operator $K$; as it
will not be needed for the argument below, we will not consider it explicitely and include it in the source term.

To study the behavior of the $\SN$ system under the action of the rotation operator $\op{R}$
defined by \eqref{eq:def_rot}, we need to formulate it in the space where $\op{R}$ operates, i.e. include the definition
of the $\SN$ unknowns $\Psi$ and $\mat{Q}_{S_N}$ in the $\SN$ operator itself:
\begin{equation}\label{eq:sn_op}
	\PihSN \op{L}\PiSN \PihSN \psi = \PihSN q,
\end{equation}
From the definitions \eqref{eq:map_SN} and \eqref{eq:map_SN_inv}, we may see that on $V_{S_N}$, $\PihSN
 = \PiSN^{-1}$.
 As \mbox{$\Rng (\PiSN) = V_{S_N}$} (and operating by $L$ does not leave this space), by operating on both sides of
 \eqref{eq:sn_op} with $\PiSN$, we obtain
\begin{equation}\label{eq:sn_op2}
	\op{L}\PiSN \PihSN \psi = \PiSN\PihSN q.
\end{equation}
Note that:
\begin{equation}\label{eq:proj_sn}
	\Projop[S_N] := \PiSN \PihSN, \quad \Projop[S_N] : V \to V_{S_N}
\end{equation} 
is a projection operator into the $S_N$ approximation space that formalizes the approximation \eqref{eq:sn_approx}.

\myparagraph{Lack of rotational invariance of the $\SN$ system}
We are now ready to prove that commutativity of the rotation operator with the transport operator \eqref{eq:commut_NTE} 
does not carry over to the $\SN$ approximation. 

\begin{theorem}\label{thm:commut_NTE}
Let there be given a transport problem 
$$
	L\psi = q
$$
with $L$ satisfying
$RL = LR$ for all $R\in\mathrm{SO}(3)$. 
Then the $\SN$ approximation given by
$$
\op{L}\Projop[S_N] \psi = \Projop[S_N] q,
$$
(with $\Projop[S_N]$ defined by \eqref{eq:proj_sn}, \eqref{eq:map_SN}, \eqref{eq:map_SN_inv} and a particular
ordinates set \mbox{$\omega = \{\bomega_n\}_{\idxset{M}}$}) satisfies
$$
\op{R}\op{L}\Projop[S_N] = \op{L}\Projop[S_N]\op{R} \forall R\in\mathrm{SO}(3)
$$
if and only if
\begin{equation}\label{eq:thm1_cond}
	\forall n\in \idxset{M}\ \exists m\in\idxset{M}: \mat{R}\bomega_n = \bomega_m \quad \forall \mat{R}\in\mathrm{SO}(3).
\end{equation} 
\end{theorem}
\begin{remark}
	Note that the conditions \eqref{eq:thm1_cond} can be satisfied only in the limit \mbox{$M\to\infty$}.
\end{remark}
\begin{proof}
Because of the commutativity of $L$ and $R$, it suffices to show that $\Projop[S_N]$ commutes
with $\op{R}$, i.e.,
\begin{equation}\label{eq:thm1_point}
	R \PiSN \PihSN \psi = \PiSN \PihSN R \psi\quad  \forall \psi \in V, 
\end{equation}
if and only if the condition \eqref{eq:thm1_cond} holds. First, for any $\psi\in V$,
\begin{equation*}
	\PihSN R\psi(\bomega) = \colset{R\psi(\bomega_n)}{\idxset{M}} = \colset{\psi(\mat{R}^T\bomega_n)}{\idxset{M}}.
\end{equation*}
Applying $\PiSN$ thus yields a function $\psi_1\in V_{S_N}$ such that
\begin{equation}\label{eq:thm1_f}
f_\psi(\bomega) = \pw{\psi(\mat{R}^T\bomega_n)}{if $\bomega = \bomega_n$ for $\bomega_n\in\omega$}{0}{if $\bomega \not
	\in \omega$.}
\end{equation}
On the other hand, let $u = \Projop[\SN]\psi = \PiSN\PihSN\psi$. Then
$$
u(\bomega) = \pw{\psi(\bomega_n)}{if $\bomega = \bomega_n$ for $\bomega_n\in\omega$}{0}{if $\bomega \not
	\in \omega$}
$$ 
and the rotated function $g_\psi(\bomega) = R\PiSN\PihSN\psi = R u(\bomega)$ is given by
$$
g_\psi(\bomega) = u(\mat{R}^T\bomega) = \pw{\psi(\bomega_n)}{if $\mat{R}^T\bomega = \bomega_n$ for
$\bomega_n\in\omega$}{0}{if $\mat{R}^T\bomega \not \in \omega$}
$$
or, equivalently, by
\begin{equation}\label{eq:thm1_g}
g_\psi(\bomega) = \pw{\psi(\bomega_n)}{if $\bomega = \mat{R}\bomega_n$ for
$\bomega_n\in\omega$}{0}{if $\bomega \not \in \mat{R}\omega$}
\end{equation}
(where action of $\mat{R}$ on the set $\omega$ is understood element-wise). If the ordinate set $\omega$ contains for
any ordinate $\bomega_n$ also its rotated copy $R\bomega_n$, i.e., condition \eqref{eq:thm1_cond} holds, then also
$$
	\bomega_n = \mat{R}^T\bomega_m
$$
and after simple substitution in \eqref{eq:thm1_g} and renumbering in \eqref{eq:thm1_f}, we obtain
$$
	f_\psi(\bomega) = g_\psi(\bomega).
$$
In order for this equality to hold true for any $\psi \in V$ (and hence for \eqref{eq:thm1_point} to hold true), the
condition \eqref{eq:thm1_f} is also necessary.
\end{proof}
\mbox{}\begin{figure}[htp]
\begin{center}
  \includegraphics[scale=1]{rot.pdf}
  \caption[Rotation of $\psi(\bomega)$]{Rotation of $\psi(\bomega)$}
  \label{fig:rot}
\end{center}
\end{figure}

As a consequence of the above theorem and lemma \ref{lemma1}, the $\SN$ approximation is not rotationally invariant.
Therefore, for example, if the conditions of theorem \ref{thm:commut_NTE} are satisfied and the sources (and boundary
conditions) are rotationally invariant ($q = Rq$), the true solution of NTE is necessarily spherically symmetric,
whereas its approximation computed by the $\SN$ method not. It is clear from the above analysis that a different
definition of the projection operator $\Proj$ is needed to eliminate this deficiency, which is precisely what the other
well-established neutron transport method -- the method of spherical harmonics -- does.

\comment{ Because components of vector $\Psi$ do not
depend on $\bomega$, they are invariant under rotations\footnote{we keep in mind that in fact $\op{R}\Psi = \op{R}\Psi(\br)= \Psi(\mat{R}^T\br)$; operator $\PiSN$ however doesn't see the spatial variable, justifying our ommission of the spatial variable} :
$$
	 \PiSN \op{R} \Psi = \PiSN \Psi.
$$
(where we understand the component-wise action of $\op{R}$ when it is applied on a vector of functions).
On the other hand, taking into account \eqref{eq:map_SN}, we observe that
$$
	 \op{R} \PiSN \Psi = \bigl(\PiSN \Psi\bigr)(\mat{R}^T\bomega) = 0
$$
whenever $\mat{R}^T\bomega \not\in \{\bomega_n\}_{\idxset{M}}$. Shifting now our attention to the inverse mapping, we can
see from \eqref{eq:map_SN_inv} that it always yields a vector of angular flux evaluations at the fixed set of directions
corresponding to the particular discrete ordinates set, hence
$$
	\op{R}\PihSN \psi(\bomega) = \PihSN \op{R} \psi(\bomega). 
$$
Using these findings in the left-hand side of eq. \eqref{eq:sn_op} together with rotational invariance of the continuous
NTE, we arrive at the final result:
For $\mat{R}^T\bomega \not\in \{\bomega_n\}_{\idxset{M}}$,
$$
	\op{R} \PihSN \op{L}\PiSN \Psi = \PihSN \op{L} \op{R} \PiSN \Psi = 0\,
	\footnote{zero vector with $M$ components}, 
$$
whereas
$$
	\PihSN \op{L}\PiSN \op{R} \Psi = \PihSN \op{L} \PiSN \Psi \neq 0,
$$ 	
and hence the $\SN$ approximation of the NTE is not rotationally invariant in cases where the NTE is.
Note that we would have $\PihSN = \PiSN^{-1}$ only if $\PihSN$ was defined on a space
of functions that
}

\subsection{The $\PN$ method}

Instead of the collocation method used by the $\SN$ approximation, the $\PN$ method uses the Galerkin weighted residuals
approach in angular domain. That is, the angularly dependent quantities in \eqref{eq1} are expanded into infinite
series of properly chosen functions that span a complete basis on the unit sphere, the equation is multiplied by each member of the basis in turn
and integrated over the sphere. The properties of the basis functions are then used to derive equations for
the expansion coefficients. Only a finite number of the expansion terms is considered to allow practical computation.
Usually, the expansion is truncated to a finite length of $K = K(N)$ terms\footnote{The length of the expansion $K$
should not be confused with the operator $K$ introduced in the previous section; it will be always clear from context 
which meaning the letter $K$ currently has.} by setting all expansion coefficients with higher index to 0 (although there exist alternative closure methods that may have favorable properties in certain situations, see e.g.
\cite{Frank0}). Then we speak of the $\PN$ approximation:
\begin{equation}\label{eq1.1}
  \psi(\br,\bomega) \approx \sum_{k=1}^{ K} \phi_k(\br) f_k(\bomega).
\end{equation}
A natural function space to support this procedure is the Hilbert space of 
square-integrable functions on the sphere $\Lp[2](\Sphere)$, equipped with the inner product
\begin{equation}\label{eq:s2_ip}
	(\psi, \varphi)_{\Lp[2](\Sphere)} = \intA{\psi(\bomega) {\overline\varphi(\bomega)}}.
\end{equation}
(overbar denotes complex conjugation). 
We will therefore assume \mbox{$\psi(\cdot,\bomega)\in\Lp[2](\Sphere)$} in this section. 

The set of spherical basis functions that were used in the original $\PN$ method are the
\textit{spherical harmonic functions}, which form a complete orthonormal system on $\Lp[2](\Sphere)$ and simplify
the algebraic manipulations needed to arrive at the conditions for coefficients $\phi_k$ (called \textit{angular
moments}). In one dimension, the spherical harmonic functions reduce to Legendre polynomials and $ K(N) = N$. For
general three-dimensional problems, there are $2n + 1$ linearly independent spherical harmonics for each degree $n$ and
$$
	 K(N) = \sum_{n=0}^{N} 2n + 1 = (N+1)^2.
$$
The approximation \eqref{eq1.1} is usually rewritten as
\begin{equation}\label{eq:pn_approx}
	\psi(\br,\bomega) \approx \sum_{k=1}^{ K} \phi_k(\br)\Y{k}{}(\bomega) \equiv
	\suma[n]{0}{N}\suma[m]{-n}{n}\angmom{n}{m}(\br)\Y{n}{m}(\bomega)
\end{equation}
where $\Y{n}{m}(\bomega)$ is the spherical harmonic function of degree $n$ and order $m$ \footnote{Formal definition of
spherical harmonics is given in \ref{app:SH}.} and in the first term on right, we consider the single index $k$ ($1 \leq k \leq  K$) that covers all the combinations of $n$ and $m$  
($0 \leq n \leq N$, $-n\leq m \leq n$) appearing in the second term. We finally arrive at a system of $ K$
partial differential equations in spatial domain which is of comparable size as the system of $\SN$ equations and has the following form:
\begin{equation}\label{eq:pn1}
	\mat{A}^x_{P_N}\,\pd{\Phi(\br)}{x} + \mat{A}^y_{P_N}\,\pd{\Phi(\br)}{y} + \mat{A}^z_{P_N}\,\pd{\Phi(\br)}{z} +
	\bigl[\sigma(\br)\mat{I} - \mat{K}_{P_N}(\br)\bigr]\Phi(\br) = \mat{Q}_{P_N}(\br),
\end{equation}
where 
\begin{equation}\label{eq:pn_vectors}
	\Phi = \col\left\{\phi_k\right\}_{\idxset{K}} \ \text{ and }\ 
	\mat{Q}_{P_N} = \col\left\{q_k\right\}_{\idxset{K}}
\end{equation}
are, respectively, the vector of angular flux
moments and angular source moments, and the index set of $\PN$ variables is defined analogously to the $\SN$ case:
$$
\idxset{K} = \{1,2,\ldots,K\}.
$$
The
Galerkin procedure results in their special form
\begin{equation}\label{eq:pn_angmom}
	\phi_k = (\psi, \Y{k}{})_{\Lp[2](\Sphere)},\quad q_k = (q, \Y{k}{})_{\Lp[2](\Sphere)}
\end{equation}
which, in view of \eqref{eq:pn_approx} and the completeness and orthogonality properties of spherical harmonics, also 
shows that the angular flux in the $\PN$ method is approximated by its orthogonal projection onto the finite-dimensional
subspace $\Lp[2]_K(\Sphere)\subset\Lp[2](\Sphere)$:
\begin{equation}\label{eq:PN_proj}
	\Projop\psi(\cdot,\bomega) := \sum_{k=1}^{ K}\ips{\psi}{\Y{k}{}} \Y{k}{}(\bomega).
\end{equation}

The system has the same form 
that for the $\SN$ approximation, but the angular matrices:
$$
	\left[A_s^P\right]_{k,l} = \intA{\Omega_s \Y{k}{}(\bomega)\Yc{l}{}(\bomega)},\quad s\in\{x,y,z\},\ 
	1 \leq k,l \leq  K
$$
are no longer diagonal, but rather banded with each moment being coupled to at
most six other moments (see, e.g., \cite{Sanchez8}).


\subsubsection{Rotational invariance of the $\PN$ equations}
The increased coupling between the unknowns in the $\PN$ system is the price for rotational invariance of spherical
harmonics that prevents ray effects appearing in $\SN$ solutions. To prove it, we will use some well known facts about
spherical harmonics (see e.g, \cite[Chap.
3]{Sansone}, \cite[Sec. 3.9]{Schreiner}). 

\myparagraph{Orthogonal decomposition of $\Lp[2](\Sphere)$}
Spherical harmonic functions of given degree $n$ generate a rotationally invariant subspace
of $\Lp[2](\Sphere)$, which we denote by $\Lambda_n$:
\begin{equation}
	\label{eq:subspace}
    	\Lambda_n = \text{Span}\bigl\{\Y{n}{m}; -n \leq m \leq n\bigr\},
\end{equation}
This means that $\op{R}(\Lambda_n) \subset \Lambda_n$ for any rotation transformation $\op{R}\in\mathrm{SO}(3)$ (cf. the
paragraph on ray effects in \Sref{sec:1-SN}). Moreover, there is no proper subspace $\Lambda_n' \subset \Lambda_n$ that
is rotationally invariant by itself. For given $n$, $\Lambda_n$ is the eigenspace associated with
the $n$-th eigenvalue of the Laplace operator on $\Sphere$:
$$
	\lap_{\Sphere} \Y{n}{m}(\bomega) = -n(n+1)\Y{n}{m}(\bomega) \quad \forall -n \leq m \leq n.
$$

Being finite-dimensional eigenspaces of a self-adjoint operator
corresponding to different eigenvalues, $\Lambda_n$ for $n=0,1,\ldots$ are mutually orthogonal, closed subspaces of
$\Lp[2](\Sphere)]$ and $\Lp[2](\Sphere) = \bigoplus_{n=0}^{\infty}\Lambda_n$\nomenclature[S]{$\bigoplus$}{direct sum of
spaces}.
Restricting to a finite direct sum, we can hence write the $\PN$ projection \eqref{eq:PN_proj} as
\begin{equation}\label{eq:pn_approx2}
	\Projop\psi(\cdot,\bomega) = \suma[n]{0}{N}\Projop[\Lambda_n]\psi(\cdot,\bomega),
\end{equation}
where 
$$
	\Projop[\Lambda_n]\psi(\br,\bomega) = \suma[m]{-n}{n}\angmom{n}{m}(\cdot)\Y{n}{m}(\bomega)
$$
is the orthogonal projection onto $\Lambda_n$.  

\myparagraph{Operator form of the $\PN$ system}
As in the $\SN$ case, we will now define the mappings that take the vector $\Phi(\cdot)$ of angular flux moments to
angular flux $\psi(\cdot,\bomega)\in\Lp[2]_K(\Sphere)$ and vice versa. The general operators are
$$
\bigl(\PiPN\mat{F}\bigr)(\bomega) := \sum_{k=1}^{ K} f_k\Y{k}{}(\bomega), \quad
\PihPN f(\bomega) = \col \left\{(\psi, \Y{k}{})_{\Lp[2](\Sphere)}\right\}_{\idxset{K}}.
$$
Using \eqref{eq:pn_angmom}, \eqref{eq:pn_vectors} and \eqref{eq:PN_proj}, we can see that 
$$
\PiPN\Phi = \PiPN\PihPN \psi = \Projop\psi.
$$ 
Also, on the space of functions that (as functions of $\bomega$) can be expressed as linear combination of
spherical harmonics up to degree $N$, $\PihPN = \PiPN^{-1}$.

The $\PN$ system \eqref{eq:pn1} can now be rewritten in terms of the transport operators from eq. \eqref{eq1op}:
$$
	\PihPN (L-K) \PiPN\Phi = \PihPN q
$$
or
$$
	\Projop (L-K) \Projop\psi = \Projop q
$$
Commutativity of the rotation operator 
Thanks to \eqref{eq:pn_approx2} and linearity of the rotation operator, we thus obtain

Using linearity of the rotation operator and rotational
invariance of each $\Lambda_n$, it follows that 
$$
	\op{R}\PiPN = \PiPN\op{R}
$$
and, since the $\PN$ system \eqref{eq:pn1} is actually obtained by applying $\PiPN$ to the left and right
hand side of eq. \eqref{eq1} and using linear independence of the spherical harmonic functions. 

\subsubsection{Drawbacks of the $\PN$ approximation}
Using the results of the previous paragraph and well-known results from the theory of Hilbert spaces, we can see that
the sum \eqref{eq:pn_approx2} (or \eqref{eq:pn_approx}) converges in the $\Lp[2](\Sphere)$ norm to the true solution of
eq. \eqref{eq1} as $N\to\infty$. However, the convergence may be very slow if the true solution to the NTE is not
sufficiently regular in the angular variable. In particular, pointwise convergence is hindered in the neighborhood of
phase space points where the solution of the NTE has jump discontinuity in $\bomega$ (which may occur for example when a
narrow beam of neutrons is freely streaming through a non-interacting medium, but also in a more typical case of domains
with multiple regions with different materials, bounded by piecewise polygonal boundary; see also \alert{discussion on
regularity properties of NTE}) and spurious oscillations are introduced to the approximate solution at these points.
These oscillations spread over the whole angular domain and slow down the norm-wise convergence. This is a well-known
property of Fourier series (which the expansion \eqref{eq:pn_approx2} generalize) known as \textit{Gibbs phenomenon}.
Moreover, these oscillations do not vanish as more terms in the series are retained. However, there are several ways of
circumventing the Gibbs phenomenon. For example, when considering \eqref{eq:pn_approx2} as a means of deriving the $\PN$
system, we may note that using a finite expansion obtained by truncating \eqref{eq:pn_approx2} at $n=N$ is not the only
way of obtaining a closed system of equations -- different closures are possible as already discussed above. This fact
has been utilized in \cite{McClarren3} where the expansion has been adjusted to mitigate the oscillations by controlling
angular gradients\footnote{Note that the expansion \eqref{eq:pn_approx2} represents the best $\Lp[2](\Sphere)$
approximation of $\psi$ by spherical polynomials up to a given degree, but absence of angular gradients in the
$\Lp[2](\Sphere)$ norm permits arbitrary oscillations.}. For other similar approaches in the context of general spectral
methods, see e.g. \cite{Tanner}.

As shown in \cite{McClarren4}, there is also another issue connected with time-dependent $\PN$ approximation that must
be kept in mind particularly when solving coupled problems. This issue is inherent in the structure of the $\PN$ system
and cannot be completely removed without losing some of its attractive properties. Namely, the authors proved that
without sources and reactions, the linear hyperbolic character of eq. \eqref{eq:pn1} (with an additional time derivative
term) together with rotational invariance allows negative solutions for positive, isotropic data in two or three
dimensions. To prevent negative solutions, one could either give-up linearity (e.g. by using a non-linear closure in a
similar way as described above), rotational invariance (thus introducing ray-effects into the solutions) or
hyperbolicity (thus changing the speed at which radiation propagates throughout the domain) -- none of which is a
generally satisfactory remedy. The authors also demonstrated that negative solutions can appear even in heterogeneous
domains containing regions with reactions or sources.

\section{Approximation of spatial dependence in $\SN$ and $\PN$ methods}
As we have seen in the previous subsections, both the $\SN$ and $\PN$ approximations lead to a system of linear
hyperbolic PDE's in spatial variables. The final approximation step typically consists of laying out a mesh over the
spatial domain and using finite difference (FD), finite volume (FV) or finite element (FE) methods to discretize the
PDE's. In the case of the $\SN$ approximation, the approach traditionally favored by the nuclear engineering community
uses the source iteration technique to decouple the system into single-direction equations, each of which is then solved
by a \textit{transport sweep} from inflow to outflow boundaries of mesh cells. The finite volume method is used to link
the mesh cells through cell-averaged and interface unknowns. Without fission and scattering (i.e., the integral term
$\op{K}$ in \eqref{eq1op}) and reflective boundaries (\alert(ref)), the system is already decoupled and only one
sweep for each direction is sufficient to determine $\op{L}^{-1}$ and hence the solution. In the other case, source
iteration is required and when the integral term is dominant (like in the case of whole-core reactor calculations) some
acceleration is usually required (see the discussion in Sec. \ref{:}).

Similarly to the explicit schemes for usual time-dependent advection-reaction problems, this direction-sweeping scheme
requires careful choice of numerical approximation of interface unknowns to ensure stability (restricting and
intertwinning the angular and spatial resolution). An alternative, attractive in particular when an unstructured
spatial mesh is used (and even more in three dimensions), is the implicit solution of of the angularly discretized
system. This is also the method of choice for the $\PN$ system, where diagonalization of the angular matrices for each
differently oriented cell would be required for the sweeping procedure. However, the stability constraints do not
disappear completely -- in order not to introduce unwanted oscillations into the solution, a stable spatial
discretization must be used to obtain the algebraic system. In the case of the finite element method, for example, it is
well known that the approximation of the solution of an advection-reaction PDE by piecewise continuous functions (i.e.,
the continuous Galerkin method) is not stable and allows arbitrarily large derivatives of the solution in the direction
of flow (and hence the oscillations, see \alert{ref}).
Therefore, either the discontinuous Galerkin (DG) method or some version of stabilized continuous Galerkin method are
required for spatial discretization. We will describe the simplest DG method for the $\SN$ system in \alert{section};
(see e.g.
\cite{Meinkohn} for the application of the streamline-upwind Petrov Galerkin method).


In any case, the final system of linear algebraic equations is generally sparse but -- given the complicated geometry
and material arrangement of realistic problems -- very large. It is usually computationally infeasible to resolve all
local features of the solution by a uniform mesh and some sort of adaptivity is employed. In some cases (typically in
engineering application like the simulation of heterogeneous nuclear reactor cores), the long-term experience may be
used to create the mesh by hand using well-established geometry and mesh generation software like the commercial CUBIT
(\cite{CUBIT}) or the open-source GMSH (\cite{GMSH}). When the a-priori knowledge of important solution features is not
available, some sort of automatic adaptivity needs to be employed, which we will discuss in the following section.

\subsection{Adaptivity}
In real applications, automatic adaptivity of the discrete phase space is usually needed to obtain sufficiently accurate
results sufficiently fast. Except for the scheme described in the Ph.D. thesis of H. Park \cite{Park} -- a rare case of coupled angular and spatial adaptivity -- all literature
available to the author describes schemes where angular and spatial adaptivity is performed separately (and there is
none about adaptivity with respect to the energetic variable). In the context of the $\PN$ method, there are examples
where the order of the spherical harmonic expansion is varied throughout the spatial domain (see e.g. \cite{Ackroyd2})
based on material properties and physical reasoning. Properties of spherical wavelets have been used in \cite{Buchan} to
drive automatic selection of the order of the expansion (in this case with respect to the wavelet interpolation basis of
$\Lp[2](\Sphere)$ instead of the spherical harmonics basis) based on the increasing size of expansion coefficients
corresponding to wavelets supported over underresolved areas of $\Sphere$. Angular adaptivity in the $\SN$ approximation
(i.e., adaptive control of the number of discrete ordinates in local areas on the sphere) is described in
\cite{Jarrell}.

More widespread use has found the adaptivity in spatial domain, using techniques developed for finite element
approximations of general hyperbolic systems. They are usually used in DG $\SN$ methods, see e.g.
\cite{Fournier,Duo,ragusa2010two} for adaptivity based on a-posteriori estimation of global $\Lp[2]$ norm of solution
error or \cite{LathouwersGoal, Wang2} for a method based on goal oriented adaptivity. Spatial adaptivity for the
standard $\PN$ approximation is not used as widely, probably because the approximation itself is not so widely used for
larger scale calculations. However, there are various reformulations of the $\PN$ system as a system of second-order
PDEs for which many usual a-posteriori estimates for
elliptic PDEs have been successfully applied. These formulations will be introduced in the
following section.

The discontinuous Galerkin framework used by majority of $\SN$ methods doesn't impose any constraints on the solution
continuity accross elements and allows the solution to be represented by completely different functions on each element.
As such, it is well-suited for implementing both the mesh refinement and polynomial order variation procedure, paving a
way for hp-adaptive FE solution.
Nevertheless, all the references above are employing h-adaptivity where the mesh is refined with fixed polynomial
approximation degree $p$. Prevalence of h-adaptivity and use of $p=1$ (linear) finite element spaces is caused partly
by the difficulty of implementation of an hp-adaptive FE code itself, partly by the fear of the well known limited
regularity of the exact solution of \eqref{eq1} even for smooth input data\footnote{By the method of characteristics, we
can expect the angular fluxes $\angflux(\cdot,\bomega)$ to be differentiable in the direction of $\bomega$, but not in
any other direction. As shown in \cite{Johnson}, the scalar fluxes, i.e. integrals of $\angflux$ over $\Sphere$, belong
at most to $H^{3/2-\epsilon}(\VV)$ where $0 < \epsilon \ll 1$ and $H^{k}(\VV)$ denotes the usual Sobolev space of order
$k$.}.
However, similarly to the experience with hp-adaptive methods in different fields, the limitation of asymptotic
convergence rate (as $h\to 0$ where $h$ is the diameter of the largest element in the mesh) dictated by a-priori error
estimates involving solution regularity typically doesn't appear until very late in the mesh refinement process or at
all (\cite{wang2009convergence}). Hence, utilizing higher order approximations still makes sense to accelerate
pre-asymptotic convergence rate as much as possible (for an attempt to use hp-adaptivity with DG $\SN$ methods -- to
author's knowledge first of its kind, see \cite{FournierDGHP}). Implementation of an $\SN$ solver in the general hp-FEM
framework Hermes by the author of this thesis could be seen as a first step for future investigations in this direction.



% REGULARITY


  

\subsection{Second-order formulations}\label{sec:second-order}

The set of steady state $\PN[1]$ equations, implying that the neutron flux varies only linearly in angle, can be under
some additional physically justifiable assumptions (\cite{Stacey, Reuss}) recast (even in 3D) into a single elliptic
equation\footnote{The same holds also for the $\SN[2]$ equations.}. This is the familiar \textit{diffusion approximation} -- thanks to its
simplicity and also the efficiency of the numerical solution techniques available for this approximation, it has always
served as a ``workhorse computational method of nuclear reactor physics" \cite[p. 43]{Stacey1}. The model is indeed
sufficiently accurate for whole core calculations of contemporary reactors, assuming that the significant finer-scale
neutron transport processes have been resolved by higher-fidelity NTE solvers applied in previous solution stages (as
already discussed in \sref{}).
The self-adjoint diffusion equation can then be solved using e.g. the finite element method in conjunction with both
powerful and theoretically well-established conjugate gradient method with symmetric preconditioners like the modern
algebraic multigrid. Solution efficiency may be improved even further also by using adaptive mesh refinement based on
highly developed a posteriori error estimates for elliptic problems. Note that the self-adjoint property of the
diffusion model can only be spoiled by the multigroup energy discretization, where energy transfers in neutron
collisions result in non-symmetric coupling of the multigroup system -- this can be however easily prevented by moving
the non-symmetric parts to the right-hand side and solving the resulting system iteratively. The simplest of such
iterative splittings (which is nevertheless used by many existing neutron transport codes, often in conjunction with
some acceleration technique) is the so-called \textit{source iteration}, which will be briefly introduced in Sec.
\ref{sec:MG}.

Although methods based on the diffusion approximation have been experimentally proven to be widely applicable for
nuclear reactor analyses, there are situations where this approximation is just too coarse and, as some recent reports
indicate \cite{Hejzlar1,Cho1}, these cases are likely to grow soon with the advent of new reactor and fuel designs. This
approximation, of course, can also be hardly expected to produce acceptable results for more general problems with
strong transport effects (leading to steep solution gradients or even discontinuities), such as those arising in the
radiation shielding studies. One possibility then is to treat the diffusion solves not as a means of obtaining the final
solution, but as preconditioners in an iteration involving a rigorous transport solution. Particularly popular became
such coupling between the diffusion calculation and discrete ordinates source iteration, which got the name
\textit{diffusion synthetic acceleration} (\cite{Alcouffe1}). Research in this field is still very active, focusing for
instance on diffusion preconditioning of Krylov subspace iterations involving discrete ordinates matrices (\cite[Chap.
1]{Azmy1}).

Another possibility is to look for other ways of transforming the first-order neutron transport equation into a set of
second-order ones, keeping the favorable mathematical and numerical properties of the diffusion equation and yet not
losing the ability to capture the most important transport effects. The simplest approach appears to be to generalize
the procedure used to obtain the diffusion equation from the zeroth and first moment equations of the $\PN[1]$ set.
Although this leads to an attractive system of weakly coupled diffusion-reaction equations in 1D, a complicated system
of strongly coupled equations with mixed second-order partial derivatives results in more dimensions (\cite{Capilla}).
Another, more general approach uses alternative formulations of the NTE itself as an integro-differential equation with
second-order spatial derivatives. Probably best known of such formulations is the even (or odd) parity form. Its
derivation begins by writing eq. \eqref{eq1} for $\bomega$ and $-\bomega$ and adding and subtracting the two resulting
equations. Two first-order equations are thus obtained for the even/odd parity fluxes  $\psi(\cdot,\bomega)_{\pm} =
\tfrac12\bigl[\psi(\cdot,\bomega) \pm \psi(\cdot,-\bomega)\bigr]$, coupled only through zero-order terms. The parity
forms of the NTE are then obtained by eliminating either $\psi_+$ or $\psi_-$ between these two equations\footnote{The
even parity equation appears to be used more often than the odd parity one; it is in fact the Euler equation for a
functional that was used in the earliest variational characterization of neutron transport (see \cite[Chap. 2]{Azmy1}
and references therein).}. One can now apply any angular discretization technique (such as the above mentioned $\PN$ or
$\SN$ methods) on the equations thus obtained to construct a self-adjoint system amenable to stable continuous FEM
solution with the powerful algebraic solution techniques mentioned above.

An improvement of this parity formulation is the \textit{self-adjoint angular-flux equation} (SAAF, see \cite{Morel1}).
It removes some of the deficiencies of the parity equations caused by the fact that the latter provide only one (even or
odd) component and not the whole angular flux. Upon angular discretization, the SAAF equation is also related to the
weighted least-squares approximation, and to the Galerkin least-squares method (a popular stabilization technique for
continuous finite element approximation of general first-order hyperbolic PDE's). This latter form has the advantage of
natural error estimator provided by the least-squares functional, which may be used to drive spatial (or even angular)
adaptivity.

\subsubsection{The simplified $\PN$ approximation}

Although the significant accuracy improvements of the second-order approximations with respect to the diffusion equation
secure their place in difficult transport calculations, their usage for the analysis of contemporary nuclear reactor
cores still seems to be very limited. This is despite the fact that the presence of materials with very different
neutron-physical properties within these cores%
% \footnote{such as those loaded with the MOX/UO2 fuel}
pronounce the transport effects. More preferred method for such calculations is the \textit{simplified $\PN$
approximation ($\SPN$)} with origins dating back to the early 1960's (see the Gelbard's seminal papers
\cite{Gelbard1,Gelbard2}). Its derivation was completely formal at the beginning -- amounting to a simple replacement of
differential operators $\textstyle\der{}{z}$ in the 1D $\PN$ system by their multidimensional counterparts $\nabla$ and
$\nabla\cdot$ and recasting those scalar unknowns operated upon by the latter as vector quantities. Despite this
mathematically weak derivation, the $\SPN$ solution was known to be equivalent to the exact one in the case of a
homogeneous medium and, comparing to either diffusion or full $\PN$ models, provided encouraging results both in terms
of accuracy and efficiency even in more realistic cases. The method was also very attractive thanks to the relative
simplicity of implementation, requiring only modification of existing multigroup diffusion codes instead of writing
completely new ones (as was the case with the other transport methods mentioned above).

After some time, however, the analysts were able to devise special transport problems for which the simple diffusion
approximation actually provided better results (see, e.g., \cite[p. 247]{Coppa1}). Validity of Gelbard's formal
derivation therefore became questioned and the $\SPN$ equations have not been seriously considered as a robust enough
improvement of the diffusion model for some time. This has changed in the 1990's when the asymptotic and variational
analyses \cite{Larsen1,Brantley1,Pomraning1} theoretically justified the method and also rigorously determined the range
of validity of the approximation. Although it turned out that this range is not significantly larger than that of the
diffusion theory (\cite{Larsen1}), the $\SPN$ approximation has recently been shown to produce more correct results than
the diffusion model under these conditions and is gaining popularity again
\cite{Frank1,McClarren1,Ragusa1,Larsen3,Kirschenmann1}. Moreover, even though the $\SPN$ solution does not tend to the
exact solution of the NTE as $N\to\infty$ in general, there are several cases in which it is equivalent to the
convergent $\PN$ expansion (see some recent papers like \cite{Coppa2,McClarren2,Larsen2}) and further research of the
$\SPN$ model and its connections to the NTE appears to be an interesting topic.

\subsubsection{Solution of the associated algebraic systems}

Nevertheless, both approaches lead to very large systems (easily of the order of $10^8$ even for a crude energetic
discretization), often ill-conditioned as a result of highly irregular material properties or meshes
(particularly when automatic mesh refinement is employed). Using even the advanced sparse direct solvers like
UMFPACK or MUMPS (\cite{UMFPACK,MUMPS}) to solve such systems is not practical.
Moreover, it is intuitively obvious and easily proved (see e.g. \cite{Arioli}) that the total approximation error is
given by the sum of di

Moreover, there are cases (typically
in engineering applications like the simulation of heterogeneous nuclear reactor cores) where it is common to create
initial spatial mesh by some specialized CAD and mesh-generation system based on the long-time operating experience. ,
The need to resolve a complicated dependence on 6 independent variables may easily result in systems with

Robust solvers that Also, because of the possibly highly heterogeneous domains with large jumps in material coefficients
between subdomains, conditioned
