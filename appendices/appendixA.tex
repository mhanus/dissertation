\chapter{Spherical harmonics}\label{app:SH}
The (complex)
spherical harmonic function of degree $n$ and order $m$\\ ($n\in\mathbb{N}_0,\ m\in\mathbb{Z}, 0\leq \abs{m} \leq n$) 
is defined as
\begin{equation}\label{eq:Ynm}
\nomenclature[a]{$\Y{n}{m}$}{spherical harmonic function of degree $n$ and order $m$\nomrefeq}
    \Y{n}{m}(\bomega) = \Y{n}{m}(\polar,\azimuthal) = 
    \sqrt{\frac{2n+1}{4\pi}\frac{(n-m)!}{(n+m)!}} \P{n}{m}(\cos\polar)e^{i m \azimuthal},
\end{equation}
or (since the dependence on polar angle $\polar$ is only through its cosine)
$$
	\Y{n}{m}(\mu,\azimuthal) = \sqrt{\frac{2n+1}{4\pi}\frac{(n-m)!}{(n+m)!}} \P{n}{m}(\mu)e^{i m \azimuthal},
$$
where
\begin{equation}\label{eq:associated_Pn}
    \nomenclature[a]{$\P{n}{m}(\mu)$}{associated Legendre polynomial of degree $n$ and order $m$\nomrefeq} 
    \P{n}{m}(\mu) = 
    \begin{cases} 
    	(-1)^m\sqrt{(1-\mu^2)^m}\ \der[m]{\P{n}(\mu)}{\mu} & \mbox{if } \hphantom{-}\,0 \leq m \leq n,\\[.5em]
    \displaystyle (-1)^{-m}\frac{(n+m)!}{(n-m)!}\P{n}{-m}(\mu) & \mbox{if } -n \leq m < 0 
    \end{cases}
\end{equation}
are the \textit{associated Legendre functions}. Note that ordinary Legendre polynomials are recovered for $m = 0$. 

These functions form a closed orthonormal system on $\Lp{2}(\Sphere)$ with respect to the standard inner product
$$
(\psi, \varphi)_{\Lp[2](\Sphere)} = \intA{\psi(\bomega) {\overline\varphi(\bomega)}}
$$
(where the overline denotes complex conjugation), that is
$$
\begin{aligned}
\intA{\Y{n}{m}(\bomega)\Yc{n}{m}(\bomega)} &=
\int_{0}^{2\pi}\d{\azimuthal}\int_{0}^{\pi}\sin\polar\d{\polar}  
\Y{n}{m}(\polar,\azimuthal)\Yc{l'}{m'}(\polar,\azimuthal)\\[.25em]
& = \int_{0}^{2\pi}\d{\azimuthal}\int_{-1}^{1}\d{\mu}
\Y{n}{m}(\mu,\azimuthal)\Yc{l'}{m'}(\mu,\azimuthal) = \kron{m}{m'}\kron{n}{l'}
\end{aligned}
$$
where $\kron{i}{j} = 1$ if $i = j$ and $\kron{i}{j} = 0$ otherwise (Kronecker delta symbol).

A real basis of spherical harmonics can be obtained from complex linear combination of the original basis and can be
written in the following form:
\begin{gather*}
    \sqrt{2}\y{n}{m}(\polar,\azimuthal) = 
    \pw{C_n^m \P{n}{m}(\cos\polar)\cos(m \azimuthal) = \mathrm{Re}\Y{n}{m}(\polar,\azimuthal)}{$m\geq 0$}
    {C_n^{-m} \P{n}{-m}(\cos\polar)\sin(-m \azimuthal) = \mathrm{Im}\Y{n}{-m}(\polar,\azimuthal)}{$m < 0$,}\\
    C_n^m = \sqrt{\frac{2n+1}{4\pi}\frac{(n-m)!}{(n+m)!}}.
\end{gather*}
Functions $\y{n}{m}$ are usually called \textit{tesseral spherical harmonics of degree $n$ and order $m$} (sometimes the
case $n = m$ is distinguished as \textit{sectorial spherical harmonics}). Linear combination of tesseral spherical
harmonics of degree $n$ produces a \textit{surface spherical harmonic of degree $n$}:
\begin{equation*}
\begin{multlined}
    \mathcal{Y}_n(\bomega) = \mathcal{Y}_n(\polar,\azimuthal) = \\A_0 P_n(\cos\polar) + \suma[m]{1}{n}\left[ A_m
    \cos(m\azimuthal)\P{n}{m}(\cos\polar) + B_m \sin(m\azimuthal)\P{n}{m}(\cos\polar)\right]
  \end{multlined}
  \end{equation*}
Surface spherical harmonics are formally defined as restrictions of homogeneous harmonic polynomials of degree $n$ to
unit sphere $\Sphere$ (\cite[Art. 110]{Byerly}, \cite[Def. 3.22]{Schreiner}).

\newpage
\begin{sidewaysfigure}
	\centering
		\includegraphics[scale=1.2]{pic/sh.eps}
		\caption[Spherical harmonics]{Spherical harmonics up to $3^{\mbox{rd}}$ degree, plotted in spherical coordinates 
		as specified in the labels above the graphs and colored green if $\mathrm{Re}\,\Y{n}{m} \geq 0$ and red
		otherwise (picture generated by Mathematica 7.0).}%
	\label{fig:SH} 
\end{sidewaysfigure}



