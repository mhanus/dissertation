\chapter{Matrices of the $\PN$ method}\label{app:C}
Recall from \sref{sec:pn_op} that the steady-state $\PN$ system can be written in the operator form as follows (eq.
\eqref{eq:pn_op}):
\begin{equation}\label{eq:PN_ss_app}
	\PihPN (A+\Sigma_t-K) \PiPN\Phi = \PihPN q,
\end{equation}
where (in the monoenergetic case) 
$$
\begin{gathered}
A\psi(\br,\bomega) = \bomega\cdot\nabla\psi(\br,\bomega),\quad
\Sigma_t\psi(\br,\bomega) = \sigma_t(\br)\psi(\br,\bomega)\\
K\psi(\br,\bomega) = \intA[']{\kappa(\br,\bomega\cdot\bomega')\psi(\br,\bomega')}.
\end{gathered}         
$$
and for $\mat{F} = \col \{f_k\}_{\idxset{K}}$,
\begin{equation}\label{eq:pn_op_def_app}
\bigl(\PiPN\mat{F}\bigr)(\bomega) := \sum_{k=1}^{ K} f_k\Y{k}{}(\bomega), \quad
\PihPN f(\bomega) = \col \left\{(f, \Y{k}{})_{\Lp[2](\Sphere)}\right\}_{\idxset{K}}
\end{equation}
(def. \eqref{eq:pn_op_def}). We will first investigate the form of the collision matrix $\mat{K}_{P_N} = \PihPN K
\PiPN$. As before, we will suppress the spatial dependence as it is not relevant for the following analysis.

\section{Collisions}
\begin{lemma}\label{lem:appC}
The spherical harmonic functions $\Y{n}{m}$ diagonalize the collision operator $K$ and
$$
	K\Y{n}{m} = \kappa_n \Y{n}{m},\quad n = 0,1,\ldots,\quad -n \leq m \leq n,
$$
where
$$
	\kappa_n = 2\pi \muint[_0]{\P{n}{}(\mu_0)\kappa(\mu_0)},\quad \mu_0 = \bomega\cdot\bomega'
$$
is the $n$-th Legendre moment of the collision kernel $\kappa$.
\end{lemma}  
\begin{proof}
	As we assume the collision kernel to be a square integrable function of the collision cosine  
	$\mu_0 \equiv \cos \polar_0 = \bomega\cdot\bomega'$ (see \fref{fig:scatter})
	and the Legendre polynomials \eqref{eq:leg} form a complete orthogonal system on $\Lp[2]([-1,1])$, we can express the
	collision kernel as a sum of the Fourier series
	$$
		\kappa(\mu_0) = \suma[n]{0}{\infty}\frac{2n+1}{4\pi}\kappa_n\P{n}{}(\mu_0),\quad
		\kappa_n = 2\pi \muint[_0]{\kappa(\mu_0)\P{n}{}(\mu_0)}.
	$$
	Then for any $n = 0,1,\ldots,\quad -n \leq m \leq n$,
	$$
\begin{aligned}
	K\Y{n}{m}(\bomega) &= \intA[']{\kappa(\bomega\cdot\bomega')\Y{n}{m}(\bomega')} \\
	&= \intA[']{\suma[r]{0}{\infty}\frac{2r+1}{4\pi}\kappa_r\P{r}{}(\bomega\cdot\bomega')\Y{n}{m}(\bomega')} \\
	&= \suma[r]{0}{\infty}\kappa_r\suma[s]{-r}{r}\Y{r}{s}(\bomega)\intA[']{\Y{r}{s}(\bomega')\Y{n}{m}(\bomega')} \\
	&= \kappa_n\Y{n}{m}(\bomega).
\end{aligned}
	$$
	where the addition theorem \eqref{eq:additionThm} has been used on third line and orthogonality relation \eqref{eq:og}
	on the fourth, completing the proof.
\end{proof}

\begin{corollary}
	Matrix $\mat{K}_{P_N} = \PihPN K \PiPN$ is diagonal, with entries given by the (repeated) Legendre moments $\kappa_n$.
\end{corollary}
\begin{proof}
	The $j$-th column of $\mat{K}_{P_N}$ is given by
	$$
		\mat{K}_{P_N}\mat{e}_j = \PihPN K \PiPN \mat{e}_j,
	$$
	where $\mat{e}_j$ is the $j$-th canonical basis vector in $\R[K]$. By definition, each index \mbox{$j = 0,1,\ldots K$}
	corresponds to a unique double index ${}_n^m$ ($n = 0,1,\ldots N$, $-n \leq m \leq n$), so that $\Y{j}{} \equiv
	\Y{n}{m}$.
	We can therefore write 
	$$
		\bigl(\PiPN \mat{e}_j\bigr)(\bomega) = \Y{j}{}(\bomega) = Y_n^m(\bomega).
	$$
	Using lemma \ref{lem:appC}, we have 
	$$
		K \PiPN \mat{e}_j = \kappa_n \Y{n}{m}
	$$
	so, when associating the index $i$ to a double index ${}_r^s$ and using the orthogonality relation \eqref{eq:og},
	$$
	\left[\mat{K}_{P_N}\right]_{ij} = \left[\PihPN K \PiPN \mat{e}_j\right]_i = (\kappa_n \Y{n}{m},
		\Y{r}{s})_{\Lp[2](\Sphere)} = \kappa_n\delta_{n}^{r}\delta_{m}^{s} = \kappa_n\kron{i}{j}
	$$
	Note that there is a single value $\kappa_n$ for all $2n + 1$ functions  $\Y{n}{m}$, so the $\mat{K}_{P_N}$ consists of
	$N+1$ diagonal blocks having consecutively $\lambda_0$, 3 times $\lambda_1$, 5 times $\lambda_3$, etc.  as their
	entries.
\end{proof}
\begin{corollary}
The complete ``capture'' matrix 
$$
	\mat{C}_{P_N} = \PihPN C \PiPN \equiv \PihPN (\Sigma_t - K) \PiPN 
$$
(corresponding to the capture cross-section $\sigma_c$ in \eqref{eq:st} and characterizing net neutron loss due to
all types of neutron-nuclei interactions) is diagonal.
\end{corollary}
\vspace*{1em}
\begin{remark}[\textsc{Isotropic scattering moment}]\label{rem:app:c}
	Fix a direction $\bomega\in\Sphere$ (the outgoing direction). Then the $0$-th Legendre moment of the scattering
	component of the collision kernel $\kappa$ (i.e. the first term on the right of \eqref{eq:splitting}), satisfies (see
	\fref{fig:scatter} for notation) 
	$$
		\sigma_{s0} = 2\pi \muint[_0]{\P{0}{}(\mu_0)\sigma_s(\mu_0)} = \int_{0}^{2\pi} \int_{0}^{\pi}
		\sigma_s(\cos \polar_0) \sin \polar_0\d{\polar_0} = 
		\int_{\Sphere} \sigma_s(\bomega\cdot\bomega') \d{\bomega'}.
	$$
	Comparing with def. \eqref{eq:ddifxs}), 
	$$
		\sigma_{s0} = \eta\sigma_s
	$$
	i.e., the $0$-th Legendre moment of scattering represents the expected number of neutrons coming out of 
	scattering collisions in any direction.
\end{remark}
\newpage
\section{Advection}\label{sec:app-adv}
Now, let us investigate the advection matrices $A_{P_N}^s$ ($s = x,y,z$). It is more
convenient for this analysis to consider the steady-state $\PN$ equations \eqref{eq:PN_ss_app} as a limit of the 
time-dependent equations
\begin{equation}\label{eq:PN_ss_app}
	\PihPN \left(\pd{}{t} + A + C\right) \PiPN\Phi = \PihPN q,
\end{equation}
in which $\pd{\psi}{t} \to 0$, for time-independent boundary conditions and sources \linebreak[4]\mbox{$q(\cdot,\cdot,t)
= \text{const}$}.
Then, the advection matrices describe advection of neutrons introduced into the system by the boundary and internal
sources, which is in the steady-state limit perfectly balanced by their capturing represented by $\mat{C}_{P_N}$. We
will limit the exposition to the case of a $\PN[3]$ approximation (the statements below have been computationally
verified to hold for $N = 1,2,\ldots,11$ using the symbolic system Mathematica 9.0).
\newpage

$$
\scriptsize
\hspace*{-2cm}
\begin{aligned}
\mat{A}_{P_N}^x &=
\left[
\begin{array}{cccccccccccccccc}
 0 & 0 & 0 & \frac{1}{\sqrt{3}} & 0 & 0 & 0 & 0 & 0 & 0 & 0 & 0 & 0 & 0 & 0 & 0 \\
 0 & 0 & 0 & 0 & \frac{1}{\sqrt{5}} & 0 & 0 & 0 & 0 & 0 & 0 & 0 & 0 & 0 & 0 & 0 \\
 0 & 0 & 0 & 0 & 0 & 0 & 0 & \frac{1}{\sqrt{5}} & 0 & 0 & 0 & 0 & 0 & 0 & 0 & 0 \\
 \frac{1}{\sqrt{3}} & 0 & 0 & 0 & 0 & 0 & -\frac{1}{\sqrt{15}} & 0 & \frac{1}{\sqrt{5}} & 0 & 0 & 0 & 0 & 0 & 0 & 0 \\
 0 & \frac{1}{\sqrt{5}} & 0 & 0 & 0 & 0 & 0 & 0 & 0 & \sqrt{\frac{3}{14}} & 0 & -\frac{1}{\sqrt{70}} & 0 & 0 & 0 & 0 \\
 0 & 0 & 0 & 0 & 0 & 0 & 0 & 0 & 0 & 0 & \frac{1}{\sqrt{7}} & 0 & 0 & 0 & 0 & 0 \\
 0 & 0 & 0 & -\frac{1}{\sqrt{15}} & 0 & 0 & 0 & 0 & 0 & 0 & 0 & 0 & 0 & \sqrt{\frac{6}{35}} & 0 & 0 \\
 0 & 0 & \frac{1}{\sqrt{5}} & 0 & 0 & 0 & 0 & 0 & 0 & 0 & 0 & 0 & -\sqrt{\frac{3}{35}} & 0 & \frac{1}{\sqrt{7}} & 0 \\
 0 & 0 & 0 & \frac{1}{\sqrt{5}} & 0 & 0 & 0 & 0 & 0 & 0 & 0 & 0 & 0 & -\frac{1}{\sqrt{70}} & 0 & \sqrt{\frac{3}{14}} \\
 0 & 0 & 0 & 0 & \sqrt{\frac{3}{14}} & 0 & 0 & 0 & 0 & 0 & 0 & 0 & 0 & 0 & 0 & 0 \\
 0 & 0 & 0 & 0 & 0 & \frac{1}{\sqrt{7}} & 0 & 0 & 0 & 0 & 0 & 0 & 0 & 0 & 0 & 0 \\
 0 & 0 & 0 & 0 & -\frac{1}{\sqrt{70}} & 0 & 0 & 0 & 0 & 0 & 0 & 0 & 0 & 0 & 0 & 0 \\
 0 & 0 & 0 & 0 & 0 & 0 & 0 & -\sqrt{\frac{3}{35}} & 0 & 0 & 0 & 0 & 0 & 0 & 0 & 0 \\
 0 & 0 & 0 & 0 & 0 & 0 & \sqrt{\frac{6}{35}} & 0 & -\frac{1}{\sqrt{70}} & 0 & 0 & 0 & 0 & 0 & 0 & 0 \\
 0 & 0 & 0 & 0 & 0 & 0 & 0 & \frac{1}{\sqrt{7}} & 0 & 0 & 0 & 0 & 0 & 0 & 0 & 0 \\
 0 & 0 & 0 & 0 & 0 & 0 & 0 & 0 & \sqrt{\frac{3}{14}} & 0 & 0 & 0 & 0 & 0 & 0 & 0 \\
\end{array}
\right]\\[.3em]
\mat{A}_{P_N}^y &=
\left[
\begin{array}{cccccccccccccccc}
 0 & \frac{1}{\sqrt{3}} & 0 & 0 & 0 & 0 & 0 & 0 & 0 & 0 & 0 & 0 & 0 & 0 & 0 & 0 \\
 \frac{1}{\sqrt{3}} & 0 & 0 & 0 & 0 & 0 & -\frac{1}{\sqrt{15}} & 0 & -\frac{1}{\sqrt{5}} & 0 & 0 & 0 & 0 & 0 & 0 & 0 \\
 0 & 0 & 0 & 0 & 0 & \frac{1}{\sqrt{5}} & 0 & 0 & 0 & 0 & 0 & 0 & 0 & 0 & 0 & 0 \\
 0 & 0 & 0 & 0 & \frac{1}{\sqrt{5}} & 0 & 0 & 0 & 0 & 0 & 0 & 0 & 0 & 0 & 0 & 0 \\
 0 & 0 & 0 & \frac{1}{\sqrt{5}} & 0 & 0 & 0 & 0 & 0 & 0 & 0 & 0 & 0 & -\frac{1}{\sqrt{70}} & 0 & -\sqrt{\frac{3}{14}} \\
 0 & 0 & \frac{1}{\sqrt{5}} & 0 & 0 & 0 & 0 & 0 & 0 & 0 & 0 & 0 & -\sqrt{\frac{3}{35}} & 0 & -\frac{1}{\sqrt{7}} & 0 \\
 0 & -\frac{1}{\sqrt{15}} & 0 & 0 & 0 & 0 & 0 & 0 & 0 & 0 & 0 & \sqrt{\frac{6}{35}} & 0 & 0 & 0 & 0 \\
 0 & 0 & 0 & 0 & 0 & 0 & 0 & 0 & 0 & 0 & \frac{1}{\sqrt{7}} & 0 & 0 & 0 & 0 & 0 \\
 0 & -\frac{1}{\sqrt{5}} & 0 & 0 & 0 & 0 & 0 & 0 & 0 & \sqrt{\frac{3}{14}} & 0 & \frac{1}{\sqrt{70}} & 0 & 0 & 0 & 0 \\
 0 & 0 & 0 & 0 & 0 & 0 & 0 & 0 & \sqrt{\frac{3}{14}} & 0 & 0 & 0 & 0 & 0 & 0 & 0 \\
 0 & 0 & 0 & 0 & 0 & 0 & 0 & \frac{1}{\sqrt{7}} & 0 & 0 & 0 & 0 & 0 & 0 & 0 & 0 \\
 0 & 0 & 0 & 0 & 0 & 0 & \sqrt{\frac{6}{35}} & 0 & \frac{1}{\sqrt{70}} & 0 & 0 & 0 & 0 & 0 & 0 & 0 \\
 0 & 0 & 0 & 0 & 0 & -\sqrt{\frac{3}{35}} & 0 & 0 & 0 & 0 & 0 & 0 & 0 & 0 & 0 & 0 \\
 0 & 0 & 0 & 0 & -\frac{1}{\sqrt{70}} & 0 & 0 & 0 & 0 & 0 & 0 & 0 & 0 & 0 & 0 & 0 \\
 0 & 0 & 0 & 0 & 0 & -\frac{1}{\sqrt{7}} & 0 & 0 & 0 & 0 & 0 & 0 & 0 & 0 & 0 & 0 \\
 0 & 0 & 0 & 0 & -\sqrt{\frac{3}{14}} & 0 & 0 & 0 & 0 & 0 & 0 & 0 & 0 & 0 & 0 & 0 \\
\end{array}
\right]\\[.3em]
\mat{A}_{P_N}^z &=
\left[
\begin{array}{cccccccccccccccc}
 0 & 0 & \frac{1}{\sqrt{3}} & 0 & 0 & 0 & 0 & 0 & 0 & 0 & 0 & 0 & 0 & 0 & 0 & 0 \\
 0 & 0 & 0 & 0 & 0 & \frac{1}{\sqrt{5}} & 0 & 0 & 0 & 0 & 0 & 0 & 0 & 0 & 0 & 0 \\
 \frac{1}{\sqrt{3}} & 0 & 0 & 0 & 0 & 0 & \frac{2}{\sqrt{15}} & 0 & 0 & 0 & 0 & 0 & 0 & 0 & 0 & 0 \\
 0 & 0 & 0 & 0 & 0 & 0 & 0 & \frac{1}{\sqrt{5}} & 0 & 0 & 0 & 0 & 0 & 0 & 0 & 0 \\
 0 & 0 & 0 & 0 & 0 & 0 & 0 & 0 & 0 & 0 & \frac{1}{\sqrt{7}} & 0 & 0 & 0 & 0 & 0 \\
 0 & \frac{1}{\sqrt{5}} & 0 & 0 & 0 & 0 & 0 & 0 & 0 & 0 & 0 & 2 \sqrt{\frac{2}{35}} & 0 & 0 & 0 & 0 \\
 0 & 0 & \frac{2}{\sqrt{15}} & 0 & 0 & 0 & 0 & 0 & 0 & 0 & 0 & 0 & \frac{3}{\sqrt{35}} & 0 & 0 & 0 \\
 0 & 0 & 0 & \frac{1}{\sqrt{5}} & 0 & 0 & 0 & 0 & 0 & 0 & 0 & 0 & 0 & 2 \sqrt{\frac{2}{35}} & 0 & 0 \\
 0 & 0 & 0 & 0 & 0 & 0 & 0 & 0 & 0 & 0 & 0 & 0 & 0 & 0 & \frac{1}{\sqrt{7}} & 0 \\
 0 & 0 & 0 & 0 & 0 & 0 & 0 & 0 & 0 & 0 & 0 & 0 & 0 & 0 & 0 & 0 \\
 0 & 0 & 0 & 0 & \frac{1}{\sqrt{7}} & 0 & 0 & 0 & 0 & 0 & 0 & 0 & 0 & 0 & 0 & 0 \\
 0 & 0 & 0 & 0 & 0 & 2 \sqrt{\frac{2}{35}} & 0 & 0 & 0 & 0 & 0 & 0 & 0 & 0 & 0 & 0 \\
 0 & 0 & 0 & 0 & 0 & 0 & \frac{3}{\sqrt{35}} & 0 & 0 & 0 & 0 & 0 & 0 & 0 & 0 & 0 \\
 0 & 0 & 0 & 0 & 0 & 0 & 0 & 2 \sqrt{\frac{2}{35}} & 0 & 0 & 0 & 0 & 0 & 0 & 0 & 0 \\
 0 & 0 & 0 & 0 & 0 & 0 & 0 & 0 & \frac{1}{\sqrt{7}} & 0 & 0 & 0 & 0 & 0 & 0 & 0 \\
 0 & 0 & 0 & 0 & 0 & 0 & 0 & 0 & 0 & 0 & 0 & 0 & 0 & 0 & 0 & 0 \\
\end{array}
\right]
\end{aligned}
$$

By computing the norms of matrices
$$
D_{xy} := \left(\mat{A}_{P_N}^x\right)^T \mat{A}_{P_N}^y - \left(\mat{A}_{P_N}^y\right)^T \mat{A}_{P_N}^x
$$
(similarly for the remaining combinations of $x$,$y$), i.e. (for the largest singular value norm)
$$
	\norm{D_{xy}} = \norm{D_{yz}} = \norm{D_{xz}} = \frac{2}{5}
$$
we observe that the matrices
$A_{P_3}^s$ ($s = x,y,z$) do not commute and hence could not be simultaneously 
diagonalized by a common eigenvector matrix. Consequently, the radiation advected by these matrices cannot be decomposed into plane-waves propagating in distinct directions (as in 
the case of the $\SN$ approximation), but rather consists of a combination of waves propagating in the infinitely many 
directions in $\R[3]$.

Let us now take an arbitrary fixed direction $\bn = [n_x,n_y,n_z]$ from this infinite set. The eigendecomposition of
the matrix
$\mat{A}_{P_N}^{\bn} = n_x \mat{A}_{P_N}^x + n_y \mat{A}_{P_N}^y + n_z \mat{A}_{P_N}^z$,
shows that speed of propagation is uniform for all $\bn\in\R[3]$, given by the eigenvalues corresponding to the case
$\norm{\bn} = 1$ (written with their multiplicities):
$$
\begin{multlined}
\textstyle
\left\{0,0,0,0,-\sqrt{\frac{3}{7}},-\sqrt{\frac{3}{7}},\sqrt{\frac{3}{7}},\sqrt{\frac{3}{7}},-\frac{1}{\sqrt{7}},
-\frac{1}{\sqrt{7}},\frac{1}{\sqrt{7}},\frac{1}{\sqrt{7}},\right.\\
\textstyle
\left.-\sqrt{\frac{1}{35} \left(15-2 \sqrt{30}\right)},
\sqrt{\frac{1}{35} \left(15-2 \sqrt{30}\right)},-\sqrt{\frac{1}{35} \left(15+2 \sqrt{30}\right)},\sqrt{\frac{1}{35} 
\left(15+2 \sqrt{30}\right)}\right\}
\end{multlined} 
$$


\begin{remark}[\textsc{Time dependent problems}] 
Let us compare the nullspaces of $\PN[2]$ advection matrix $\mat{A}_{P_2}^{\bn}$: 
$$
\left[
\begin{array}{ccc}
 \frac{\sqrt{\frac{3}{5}} \left(n_z^2-1\right)}{1-2 n_y^2} & \frac{2 \sqrt{\frac{3}{5}} n_x n_z}{1-2 n_y^2} & \frac{\frac{3 n_z^2}{2 n_y^2-1}+1}{\sqrt{5}} \\
 0 & 0 & 0 \\
 0 & 0 & 0 \\
 0 & 0 & 0 \\
 \frac{n_x \left(n_z^2-2 n_y^2\right)}{n_y \left(2 n_y^2-1\right)} & -\frac{n_z-2 n_z^3}{n_y-2 n_y^3} & \frac{\sqrt{3} n_x n_z^2}{n_y-2 n_y^3} \\
 -\frac{n_z-n_z^3}{n_y-2 n_y^3} & \frac{n_x \left(-2 n_y^2-2 n_z^2+1\right)}{n_y \left(2 n_y^2-1\right)} & \frac{\sqrt{3} n_z \left(2 n_y^2+n_z^2-1\right)}{n_y \left(2 n_y^2-1\right)} \\
 0 & 0 & 1 \\
 0 & 1 & 0 \\
 1 & 0 & 0 \\
\end{array}
\right]
$$
and $\PN[3]$ advection matrix $\mat{A}_{P_3}^{\bn}$:
$$
\hspace*{-3.5cm}
\scriptsize
\left[
\begin{array}{cccc}
 0 & 0 & 0 & 0 \\
 \frac{\sqrt{\frac{15}{14}} n_x \left(n_z^2-1\right)}{n_y \left(4 n_y^2-3\right)} & \frac{\sqrt{\frac{5}{7}} n_z \left(2 n_y^2+3 n_z^2-3\right)}{n_y \left(4 n_y^2-3\right)} & \frac{n_x \left(-4 n_y^2-15 n_z^2+3\right)}{\sqrt{14} n_y \left(4 n_y^2-3\right)} & \frac{\sqrt{\frac{3}{7}} n_z \left(-4 n_y^2-5 n_z^2+3\right)}{n_y \left(4 n_y^2-3\right)} \\
 -\frac{\sqrt{\frac{30}{7}} n_x n_z \left(n_z^2-1\right)}{8 n_y^4-10 n_y^2+3} & -\frac{\sqrt{\frac{5}{7}} \left(2 n_z^2-1\right) \left(4 n_y^2+3 n_z^2-3\right)}{8 n_y^4-10 n_y^2+3} & \frac{5 \sqrt{\frac{2}{7}} n_x \left(n_y^2-3 n_x^2\right) n_z}{8 n_y^4-10 n_y^2+3} & \frac{\sqrt{\frac{3}{7}} \left(8 n_y^4-10 n_y^2+10 n_z^4+5 \left(4 n_y^2-3\right) n_z^2+3\right)}{8 n_y^4-10 n_y^2+3} \\
 -\frac{\sqrt{\frac{15}{14}} \left(2 n_y^2-2 n_z^2-1\right) \left(n_z^2-1\right)}{8 n_y^4-10 n_y^2+3} & \frac{2 \sqrt{\frac{5}{7}} n_x n_z \left(2 n_y^2-3 n_z^2\right)}{8 n_y^4-10 n_y^2+3} & \frac{8 n_y^4-10 \left(n_z^2+1\right) n_y^2-30 n_z^4+15 n_z^2+3}{\sqrt{14} \left(8 n_y^4-10 n_y^2+3\right)} & \frac{10 \sqrt{\frac{3}{7}} n_x n_z^3}{8 n_y^4-10 n_y^2+3} \\
 0 & 0 & 0 & 0 \\
 0 & 0 & 0 & 0 \\
 0 & 0 & 0 & 0 \\
 0 & 0 & 0 & 0 \\
 0 & 0 & 0 & 0 \\
 \frac{n_x \left(-8 n_y^4+\left(4 n_z^2+6\right) n_y^2-3 n_z^4+n_z^2-1\right)}{n_y \left(8 n_y^4-10 n_y^2+3\right)} & \frac{\sqrt{\frac{3}{2}} n_z \left(-6 n_z^4+5 n_z^2-1\right)}{n_y \left(8 n_y^4-10 n_y^2+3\right)} & \frac{\sqrt{15} n_x n_z^2 \left(3 n_z^2-1\right)}{n_y \left(8 n_y^4-10 n_y^2+3\right)} & \frac{\sqrt{\frac{5}{2}} n_z^3 \left(4 n_y^2+6 n_z^2-5\right)}{n_y \left(8 n_y^4-10 n_y^2+3\right)} \\
 \frac{\sqrt{\frac{3}{2}} n_z \left(-4 n_z^4+5 n_z^2-1\right)}{n_y \left(8 n_y^4-10 n_y^2+3\right)} & -\frac{n_x \left(2 n_y^2-3 n_z^2\right) \left(4 n_y^2+4 n_z^2-3\right)}{n_y \left(8 n_y^4-10 n_y^2+3\right)} & \frac{\sqrt{\frac{5}{2}} n_z \left(4 n_y^2+3 n_z^2-3\right) \left(4 n_z^2-1\right)}{n_y \left(8 n_y^4-10 n_y^2+3\right)} & \frac{\sqrt{15} n_x n_z^2 \left(-4 n_y^2-4 n_z^2+3\right)}{n_y \left(8 n_y^4-10 n_y^2+3\right)} \\
 \frac{\sqrt{15} n_x n_z^2 \left(n_z^2-1\right)}{n_y \left(8 n_y^4-10 n_y^2+3\right)} & \frac{\sqrt{\frac{5}{2}} n_z \left(2 n_z^2-1\right) \left(4 n_y^2+3 n_z^2-3\right)}{n_y \left(8 n_y^4-10 n_y^2+3\right)} & -\frac{n_x \left(8 n_y^4-10 n_y^2+15 n_z^4+5 \left(4 n_y^2-3\right) n_z^2+3\right)}{n_y \left(8 n_y^4-10 n_y^2+3\right)} & -\frac{\sqrt{\frac{3}{2}} n_z \left(16 n_y^4+20 \left(n_z^2-1\right) n_y^2+10 n_z^4-15 n_z^2+6\right)}{n_y \left(8 n_y^4-10 n_y^2+3\right)} \\
 0 & 0 & 0 & 1 \\
 0 & 0 & 1 & 0 \\
 0 & 1 & 0 & 0 \\
 1 & 0 & 0 & 0 \\
\end{array}
\right]
$$
Unlike the $\PN[3]$ approximation, we can see that the $\PN[2]$ approximation contains in its advection nullspace 
nonzero components of the 0-th moment of angular flux, which is proportional to the scalar flux (total spatial neutron 
density). Therefore, as a consequence of $\PN[2]$ approximation, not all scalar flux components are propagated by the
action $\mat{A}_{P_2}^{\bn} \Psi$. It turns out that this is true for any even-order $\PN$ approximation, which is
``probably the most salient argument why even-order expansions should be shunned for time dependent problems'' 
\cite[p. 20]{McClarren5}.
\end{remark}
  