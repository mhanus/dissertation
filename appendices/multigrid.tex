\chapter{On the origin of smoothed aggregations}\label{app:multigrid}

\section*{A short note dedicated to prof. Karel Segeth at the occasion of his 70th birthday.}

\subsubsection*{Authors:}

Pavla Fra\v{n}kov\'{a}, Milan Hanu\v{s},
Hana Kopincov\'{a}, Roman Ku\v{z}el,
Petr Van\v{e}k and Zbyn\v{e}k Vastl















\section{Introduction}
The smoothed aggregation method
\cite{vanek-accel,FMS,amg-theory,Vanek_Mandel_Brezina_1995} proved
to be a very efficient tool for solving various types
of elliptic problems and their singular perturbations.
In this short note, we turn
to the very roots of smoothed aggregation method and derive its
two-level variant on a systematic basis.

The multilevel method consists in combination of a coarse-grid correction
and smoothing.
The coarse-grid correction of a standard
two-level method is derived using the $A$-orthogonal projection of an
error to the range of the prolongator.
In other words, the coarse-grid correction
vector
is chosen to minimize the error {\em after coarse-grid correction procedure}.
This means, the standard two-level method
minimizes the error in an intermediate stage of the iteration, while
we are, naturally, interested in
minimizng  {\em the final error after accomplishing the entire iteration}.
In other words, we strive to minimize the error after
after coarse-grid correction and
subsequent smoothing. The two-level smoothed aggregation method
is obtained by solving this minimization problem. This, in the opinion
of the authors, explains its remarkable robustness.

We derive the two-level smoothed aggregation
method from the variational objective to minimize the
error after coarse-grid correction and subsequent post-smoothing.
Then, by a trivial argument, we extend our result
to the two-level method
with pre-smoothing, coarse-grid correction and post-smoothing.

The minimization of error after coarse-grid correction
and subsequent smoothing leads to a method with smoothed prolongator.
We can say that by smoothing the prolongator,
we adapt the coarse-space (the range of the
prolongator) to the post-smoother so that the resulting iteration is as
efficient as possible. Our short explanation applies to any two-level method
with smoothed prolongator. The particular case we have in mind is, however, a
method with smoothed {\em tentative} prolongator given by generalized
unknowns aggregations
\cite{amg-theory}. The discrete basis functions of the coarse-space
(the columns of the prolongator) given by
unknowns aggregations have no overlap; the natural
overlap of discrete
basis functions (like it is in the case of
finite element basis functions) is created by smoothing and, for
additive point-wise smoothers, leads to sparse
coarse-level
matrix.

Our argument is basically trivial. It, however, shows a fundamental
property of the method with smoothed prolongator, that is essential.
This argument is known to the authors for a long time,
but has never been published.

We conclude our paper by a numerical test. Namely, we demonstrate
experimentally that smoothed aggregation method with powerful smoother
and small coarse-space solves efficiently highly anisotropic problems
without the need to perform semi-coarsening (the coarsening that follows
only strong connections).


\section{Two-level method}
We solve a system of linear algebraic equations
\eq{A}
       A\vc{x} = \vc{f},
\qe
where $A$ is a symmetric positive definite matrix of order $n$ and
$\vc{f} \in \Re^n$. We assume that an injective linear {\em prolongator}
$p : \Re^m \rightarrow \Re^n,\; m < n$ is given.

The two-level method consists in the combination of
a {\em coarse-grid correction} and
{\em smoothing}. The smoothing means using point-wise iterative methods 
at the beginning and at the end of the iteration.
The coarse-grid correction is derived by correcting an error
$\vc{e}$ by a coarse-level vector $\vc{v}$ so that the resulting error
$\vc{e}-p\vc{v}$ is minimal in $A$-norm. In other words, we solve
the minimization problem
\eq{standard}
  \mbox{find} \; \vc{v} \in \Re^m \; \mbox{so that} \;
  \| \vc{e}-p\vc{v}\|_A \; \mbox{is minimal}.
\qe
It is well-known that such vector $p\vc{v}$ is an $A$-orthogonal projection
of the error $\vc{e}$ onto $\range(p)$, with the projection
operator given by
$$
      Q = p(p^TAp)^{-1}p^TA.
$$
Thus, the error propagation opeartor of the coarse-grid correction is given by
$I-Q=I-p(p^TAp)^{-1}p^TA$ and the error propagation operator of the two-level
method by
\eq{E}
      E_{TGM}=S_{post}[I-p(p^TAp)^{-1}p^TA]S_{pre},
\qe
where $S_{pre}$ and $S_{post}$ are error propagation operators of
pre- and post- smothing iterations, respectively.

Clearly, for the error $\vc{e}(\vc{x})\equiv \vc{x}-A^{-1}\vc{f}$ we have
$A\vc{e}(\vc{x})=A\vc{x}-\vc{f}$. Hence, the coarse-grid correction can be
algorithmized  as
$$
     \vc{x} \leftarrow \vc{x} - p(p^TAp)^{-1}p^T(A\vc{x}-\vc{f})
$$
and the variational two-level algorithm with post-smoothing step
proceeds as follows:
\begin{algorithm}
\label{TL}
\quad
\begin{enumerate}
\item Pre-smooth: $\vc{x} \leftarrow {\mathcal S}_{pre}(\vc{x},\vc{f})$,
\item evaluate the residual:
   $\vc{d}=A\vc{x}-\vc{f}$,
\item restrict the residual:
   $\vc{d}_2 = p^T\vc{d}$,
\item solve a coarse-level problem
   $A_2 \vc{v} = \vc{d}_2,\quad A_2 = p^TAp$,
\item correct the approximation
   $\vc{x} = \vc{x} - p\vc{v}$,
\item post-smooth
   $\vc{x} = {\mathcal S}_{post}(\vc{x},\vc{f})$.
\end{enumerate}
\end{algorithm}
Here, ${\mathcal S}_{pre}(.,.)$ and ${\mathcal S}_{post}(.,.)$, respectively,
represent one or more iterations of
point-wise iterative methods for solving \eqr{A}.


The coarse-grid correction vector $\vc{v}$ is chosen to minimize the error
after Step 5 of Algorithm \ref{TL}.
Thus, we conclude that in the case of a standard variational multigrid,
the coarse-grid correction procedure minimizes
the error in an intermediate stage of the iteration,
while we are in fact interested in minimizing the final
error after accomplishing the entire iteration. This means to minimize
the error after coarse-grid correction with subsequent smoothing.



\section{The smoothed aggregation two-level method}
In the smoothed aggregation method, we construct the coarse-grid
correction to minimize
the error {\em after coarse-grid correction with subsequent smoothing},
which means the final error on the exit of the iteration procedure.
The minimization of the error after pre-smoothing, coarse-grid correction
and post-smoothing then follows immediately by a trivial argument.

Let $S$ be the error propagation operator of the post-smoother
${\mathcal S}(.,.)={\mathcal S}_{post}(.,.)$.
Throughout this section we assume that $S$
is sparse. This is due to the fact that
the above minimization problem leads to smoothed prolongator
$P=Sp$
and we need a sparse coarse-level matrix $A_2=P^TAP$.
The additive point-wise smoothing methods have, in general, sparse
error propagation operator; this is the case of
Jacobi method or Richardson's iteration.

For a multilevel method with post-smoothing only, the
error after coarse-grid correction and subsequent smoothing is
given by
\eq{final-error}
       S(\vc{e}-p\vc{v}),
\qe
where $\vc{v}$ is a correction vector and $\vc{e}$ the error on the entry
of the iteration  procedure. We choose $\vc{v}$ so that
the error in \eqr{final-error} is minimal in $A$-norm,
that is, we solve the minimization
problem
\eq{var-object}
   \mbox{find}\;\vc{v} \in \Re^m\;\mbox{such that} \;
   \|S(\vc{e}-p\vc{v})\|_A \;\mbox{is minimal}.
\qe
Since $\|S(\vc{e}-p\vc{v})\|_A=\|\vc{e}-p\vc{v}\|_{S^TAS}$,
the minimum is attained for $\vc{v}$ satisfying
$$
 \langle S^TAS(\vc{e}-p\vc{v}),p\vc{w} \rangle = 0\;
 \forall \vc{w} \in \Re^m.
$$
We have $\langle S^TAS(\vc{e}-p\vc{v}),p\vc{w} \rangle =
 \langle p^TS^TAS(\vc{e}-p\vc{v}),\vc{w}\rangle$,
hence the above identity is equivalent to $p^TS^TASp\vc{v} = p^TS^TAS\vc{e}$
and setting $P=Sp$, it becomes
\eq{cgc}
  P^TAP \vc{v} =P^TAS\vc{e}.
\qe
Here, $\vc{e}$ is the error on the entry of the iteration procedure.
Assume for now that $P$ is injective.
Then by \eqr{cgc}, we have
$\vc{v}=(P^TAP)^{-1}P^TAS\vc{e}$ and the error after coarse-grid correction
and subsequent smoothing is given by
\eq{derive}
     S(\vc{e}-p\vc{v})=S\left[\vc{e} - p(P^TAP)^{-1}P^TAS\vc{e} \right] =
     \left[I - P(P^TAP)^{-1}P^TA \right] S\vc{e}.
\qe
By comparing the operator
\eq{E-result}
    E=\left[I - P(P^TAP)^{-1}P^TA \right] S
\qe
on the right-hand side
of \eqr{derive} with \eqr{E}, we identify $E$ as the error propagation
operator
of the variational multigrid with smoothed
prolongator $P=Sp$ and pre-smoothing
step given by $\vc{x} \leftarrow {\mathcal S}(\vc{x},\vc{f})$.
The algorithm is as follows:
\begin{algorithm}
\label{alg-SMA}
\quad
\begin{enumerate}
\item Pre-smooth: $\vc{x} \leftarrow {\mathcal S}(\vc{x},\vc{f})$,
\item evaluate the residual: $\vc{d}=A\vc{x}-\vc{f}$,
\item restrict the residual: $\vc{d}_2 = P^T\vc{d}$,
\item solve the coarse-level problem:
      $A_2 \vc{v} = \vc{d}_2,\quad A_2 = P^TAP$,
\item correct the approximation:
      $\vc{x}\leftarrow \vc{x} - P\vc{v}$.
\end{enumerate}
\end{algorithm}

\begin{note}
Note that in the process of the deriving the algorithm in \eqr{derive},
our post-smoother have become a pre-smoother. Nothing was lost in that
process; the algorithm minimizes the final error and takes into account
the pre-smoother.
\end{note}
\begin{note}
The smoothed prolongator $P=Sp$ is potentially non-injective, hence the
coarse-level matrix $A_2=P^TAP$ is potentially singular. In this case,
we need to replace the inverse of $P^TAP$ in \eqr{derive} by a
pseudo-inverse.
\end{note}

We summarize our considerations in the form of a theorem.
\begin{theorem}
The error propagation operator $E$ in \eqr{E-result} (the error
propagation operator of Algorithm~\ref{alg-SMA})
satisfies the identity
$$
      \|E\vc{e}\|_A = \inf_{\vc{v} \in \Re^m} \|S(\vc{e}-p\vc{v})\|_A
$$
for all $\vc{e} \in \Re^n$.
\end{theorem}
\begin{proof}
The proof follows directly from the fact that Algorithm~\ref{alg-SMA}
was derived from variational objective \eqr{var-object}.
\end{proof}

\begin{note}
\label{note-pre}
One may also start with the variational objective to minimize the final error
after performing the pre-smoothing, the coarse-grid correction and the
post-smoothing.
Such extension is trivial, the pre-smoother has no influence on the
coarse-grid correction operator $I - P(P^TAP)^{-1}P^TA$ and influences only
its argument.
Indeed, asuming the error propagation operator of the
pre-smoother is $S^*$ (the $A$-adjoint operator),
the final error is given by $S(S^*\vc{e}-p\vc{v})$
and we solve the minimization problem
\eq{objective1}
\mbox{for}\;\vc{e} \in \Re^n\;
\mbox{find} \; \vc{v} \in \Re^m\;:\; \|S(S^*\vc{e}-p\vc{v})\|_A \;
\mbox{is minimal}.
\qe
Fundamentally, this is the same minimization problem as \eqr{var-object};
to derive the corresponding algorithm, it is simply sufficient to follow
our manipulations from \eqr{var-object} to \eqr{derive}
with $\vc{e} \leftarrow S^*\vc{e}$. This way,
we end up with a two-level method that has the error propagation
operator
\eq{epp1}
   E=\left[I - P(P^TAP)^{-1}P^TA \right] SS^*,
\qe
(see \eqr{E}) that is, with the algorithm
\begin{algorithm}
\label{alg-SMA1-pre}
\quad
\begin{enumerate}
\item Pre-smooth: $\vc{x} \leftarrow {\mathcal S_t}(\vc{x},\vc{f})$,
where ${\mathcal S_t}$ is an iterative method with error propagation
operator $S^*$.
\item pre-smooth: $\vc{x} \leftarrow {\mathcal S}(\vc{x},\vc{f})$,
where ${\mathcal S}$ is an iterative method with error propagation
operator $S$.
\item evaluate the residual: $\vc{d}=A\vc{x}-\vc{f}$,
\item restrict the residual: $\vc{d}_2 = P^T\vc{d}$,
\item solve the coarse-level problem:
      $A_2 \vc{v} = \vc{d}_2,\quad A_2 = P^TAP$,
\item correct the approximation:
      $\vc{x}\leftarrow \vc{x} - P\vc{v}$.
\end{enumerate}
\end{algorithm}
\end{note}

We summarize the content of Remark~\ref{note-pre} as a theorem.
\begin{theorem}
The error propagation operator \eqr{epp1} of Algorithm~\ref{alg-SMA1-pre}
satisfies the identity
$$
      \|E\vc{e}\|_A = \inf_{\vc{v} \in \Re^m} \|S(S^*\vc{e}-p\vc{v})\|_A
$$
for all $\vc{e} \in \Re^n$.
\end{theorem}
\begin{proof}
The proof follows directly from the fact that Algorithm~\ref{alg-SMA1-pre}
was derived from variational objective \eqr{objective1}.
\end{proof}

\begin{note}
Our manipulations hold equally for a general pre-smoother with error
propagation operator $M \neq S^*$, simply by replacing $S^* \leftarrow M$.
The error propagation operator $M$ has no influence on the coarse-space
and thus it does not have to be sparse.
\end{note}

\section{Numerical example}
To demonstrate the robustness of smoothed aggregation method,
we consider the algorithm that is a modification of
the method proposed and analyzed in \cite{VBT}.
Its relationship to Algorithm~\ref{alg-SMA} is obvious.
This method uses the smoothing iterative method ${\mathcal S}(\cdot,\cdot)$
which is a sequence of Richardson's iterations with carefully chosen
iteration parameters. The error propagation
operator $S$ of the smoother ${\mathcal S}(\cdot,\cdot)$
is therefore a polynomial in the matrix $A$.

In this method, we use massive smoother $S$ and a small coarse-space
resulting in sparse coarse-level matrix.

Let ${\bar \lambda} \geq \varrho(A)$ and $d$ be the desired
degree of the smoothing polynomial $S$. We set
\eq{alpha_i}
     \alpha_i = \left[\frac{{\bar \lambda}}{2}
                     \left(1-\mbox{cos~}\frac{2i \pi}{2d+1} \right)
                \right]^{-1},
     \quad i=1, \ldots, d,
\qe
\eq{S}
   S=\left(I-\alpha_1 A\right) \ldots
     \left(I-\alpha_d A\right)
\qe
and
$$
   P=Sp.
$$
Here, $p$ is a {\em tentative prolongator} given by generalized
unknowns aggregation.
The simplest aggregation method is described in this section.

The smoother $S$ is chosen to minimize $\varrho(S^2A)$. The reason for
this comes from the fact that the convergence of the method of \cite{VBT}
is guided by the constant $C$ in the weak approximation condition
\eq{wac}
     \forall \vc{e} \in \Re^n \; \exists \vc{v} \in \Re^m \;:\;
     \|\vc{e} - p\vc{v}\| \leq \frac{C}{\sqrt{\varrho(S^2A)}}\|\vc{e}\|_A.
\qe
The smaller $\varrho(S^2A)$, the easier it becomes to satisfy \eqr{wac} with
a reasonable (sufficiently small) constant.
It holds that (\cite{VBT})
\eq{lambda-S^2A}
       {\bar \lambda}_{S^2A} \equiv \frac{{\bar \lambda}}{(1+2d)^2} \geq
       \varrho(S^2A).
\qe

The aggregates $\{\mathcal{A}_{j}\}$ are sets of fine-level degrees of freedom
that form a disjoint covering of the set of all fine-level degrees of freedom.
For example, we can choose aggregates to form a decomposition of the set of
degrees of freedom induced by a geometrically reasonable
partitioning of the computational domain.
For standard discretizations of scalar elliptic problems,
the tentative prolongator matrix $p$ is the $n\times m$ matrix
($m=$ the number of aggregates)
\eq{aggr-prol}
      p_{ij}=\left\{\begin{array}{ll}
                                  1 & \mbox{if}\;\; i \in \mathcal{A}_j \\
                                  0 & \mbox{otherwise}
                    \end{array} \right.
\qe
that is, the $j$-th column is created by restricting a vector of ones onto
the $j$-th aggregate, with zeroes elsewhere. Thus, the aggregation method
can be viewed as a piece-wise constant coarsening in a discrete sense.
The generalized aggregation
method, suitable for non-scalar elliptic problems
(like that of linear elasticity), is described in \cite{amg-theory}.

\begin{algorithm}
\label{alg-2L-final}
Given the degree $d$ of the smoothing polynomial $S=pol(A)$,
the smoothed prolongator $P=Sp$ where $p$ is the tentative prolongator and
the prolongator smoother $S$ is given by \eqr{S},
the upper bound
${\bar \lambda} \geq \varrho(A)$ and a parameter $\omega\in (0,1)$,
one iteration of the two-level
algorithm
$$
    \vc{x} \leftarrow TG(\vc{x},\vc{f})
$$
proceeds as follows:
\begin{enumerate}
\item perform
$$
         \vc{x} \leftarrow \vc{x} -
         \frac{\omega}{{\bar \lambda}_{S^2A}}S^2(A\vc{x}-\vc{f}),
$$
where $\bar{\lambda}_{S^2A}$ is given by \eqr{lambda-S^2A} and
$S$ by \eqr{S},
\item perform the iteration with symmetric error propagation operator $S$ 
given by \eqr{S}, that is, \\
for $i=1,\ldots,d$ do
$$
        \vc{x} \leftarrow \left(I-\alpha_i A\right)\vc{x} +
        \alpha_i \vc{f},
$$
\item evaluate the residual $\vc{d}=A\vc{x}-\vc{f}$,
\item restrict the residual $\vc{d}_2=P^T\vc{d}$,
\item solve the coarse-level problem $A_2\vc{v}=\vc{d}_2$, $A_2=P^TAP$,
\item correct the approximation $\vc{x} \leftarrow \vc{x}-P\vc{v}$,
\item for $i=1,\ldots,d$ do
$$
        \vc{x} \leftarrow \left(I-\alpha_i A\right)\vc{x} +
        \alpha_i \vc{f},
$$
\item perform
$$
         \vc{x} \leftarrow \vc{x} -
         \frac{\omega}{{\bar \lambda}_{S^2A}}S^2(A\vc{x}-\vc{f}).
$$
\end{enumerate}
\end{algorithm}
Thus, Algorithm~\ref{alg-2L-final} is a symmetrized version of
Algorithm~\ref{alg-SMA} with added smoothing in steps 1 and 8.


It is generally believed that in order to solve efficiently an anisotropic
problem, one has to perform coarsening only by following
{\em strong connections}. This technique is called {\em semi-coarsening}.
In our case, we form aggregates by coarsening by a factor of 10 in
all 3 spatial directions, which means, we do not perform semi-coarsening.
Despite of this fact, our method gives satisfactory results regardless
of the anisotropy coefficient $\varepsilon$.  In this experiment,
the symmetric Algorithm~\ref{alg-2L-final} is used as a
conjugate
gradient method preconditioner.

{\bf Test problem}
\begin{itemize}
\item
Problem:
\eq{prob-2}
  -\left(
     \frac{\partial^2}{\partial x^2} +
     \varepsilon \frac{\partial^2}{\partial y^2} +
     \frac{\partial^2}{\partial z^2}
   \right) u = f \; \mbox{on} \; \Omega = (0,1)^3, \; u=0 \;
   \mbox{on} \; \partial \Omega.
\qe
\item Mesh: $82 \times 82 \times 82$ regular square mesh, 512 000 unconstrained
      degrees of freedom.
\item Aggregates: cubic groups of $10 \times 10 \times 10$
      unconstrained vertices.
\item Coarse-space size: 512 degrees of freedom.
\item Degree of smoothing polynomial: 7.
\item Stopping criterion: relative residual $<10^{-9}$.
\end{itemize}	

The results are summed up in Table~\ref{table-anis}. Note that here, the
estimate of the rate of convergence after $N$ iterations is defined as
$$
  q_N=\left(\|A \vc{x}^{N} - \vc{f}\| / \|A \vc{x}^{0} - \vc{f}\|
    \right)^{\frac{1}{N}}.
$$
Here, $\vc{x}^{i}$ denotes the $i$-th iteration.

\begin{table}
\begin{center}
\begin{tabular}{|c|c|c|}
\hline
\multicolumn{3}{|c|}
{512 000 dofs, coarse space 512 dofs, $\mathrm{deg}(S)=7$, $H/h=9.$} \\
\hline
$\varepsilon $ &   rate of conv. $q_N$ & no. iter. $N$ \\ \hline \hline
1000           &   0.321  & 19          \\ \hline
100            &   0.241  & 15          \\ \hline
10             &   0.137  & 11          \\ \hline
1              &   0.131  & 11          \\ \hline
0.1            &   0.221  & 14          \\ \hline
0.01           &   0.317  & 19          \\ \hline
0.001          &   0.300  & 18          \\ \hline
\end{tabular}
\caption{3D anisotropic problem}
\label{table-anis}
\end{center}
\end{table}


\section*{Acknowledgements}
This work was sponsored by the
TA\v{C}R (Technologick\'{a} Agentura \v{C}esk\'{e} Republiky)
grant TA01020352, ITI (Institut Teoretick\'{e} Informatiky) grant 1R0545, Department of the Navy Grant N62909-11-1-7032.

% With BibTeX, uncomment following two lines:
% \bibliographystyle{am2013} % bibliographis style - name of *.bst file to use
% \bibliography{sample}    % bibliographic database - name of *.bib file

% Use these lines without BibTeX.
% Bibliography created by BibTeX can be loaded from sample.bbl file.
% \begin{thebibliography}{10}
% \bibitem{BPWX}
% {\sc J.~H. Bramble, J.~E. Pasciak, J.~Wang, and J.~Xu}, {\em Convergence
%   estimates for multigrid algorithms without regularity assumptions}, Math.
%   Comp., 57 (1991).
% \bibitem{XZ:identity}
% {\sc J.~Xu and L.~T. Zikatanov,}
% {\it The method of alternating projections and the method of subspace corrections in Hilbert space,}
% Journal of the American Mathematics Society {\bf 15}(2002).
% \bibitem{MLBFP}
% {\sc Panayot S. Vassilevski,} {\it Multilevel Block Factorization Preconditioners,}
% Matrix-based Analysis and Algorithms for Solving Finite Element Equations, Springer, New York, 2008.
% \bibitem{Achi}
% {\sc A. Brandt},
% {\em Algebraic Multigrid Theory: The symmetric Case},
% Appl. Math. Comput., 19(1986).
% \bibitem{ciarlet}
% {\sc P.~G.~Ciarlet},
% {\it The finite element method for elliptic problems,}
% Series ``Studies in Mathematics and its Applications'', North-Holland,
% Amsterdam,
% 1978.
% \bibitem{KV1}
% {\sc J. K\v{r}\'{\i}\v{z}kov\'a and P. Van\v{e}k},
% {\em Two-level preconditioner with small coarse grid appropriate
% for unstructured meshes},
% Numerical Linear Algebra with Applications, 3(1996), no 4.
% 
% \bibitem{V}
% {\sc P. Van\v{e}k},
% {\em Smoothed Prolongation Multigrid with Rapid Coarsening and Massive
% Smoothing},
% To appear in Appl. Math.
% \bibitem{VBT}
% {\sc P.~Van\v{e}k, M.~Brezina, and R.~Tezaur},
% {\em Two-grid method for linear elasticity on unstructured meshes},
% SIAM J. Sci Comput., 21(1999).
% \bibitem{BHMV}
% {\sc M. Brezina, C. Heberton, J. Mandel, P. Van\v{e}k},
% {\em An iterative method with convergence rate chosen a priori},
% UCD/CCR Report no. 140, 1999.
% \bibitem{Vanek_Mandel_Brezina_1995}
% {\sc P. Van\v{e}k, J. Mandel, M. Brezina},
% {\em Algebraic Multigrid by Smoothed Aggregation for Second and Fourth Order
% Elliptic Problems},
% Computing 56(1996).
% \bibitem{vanek-accel}
% {\sc P. Van\v{e}k},
% {\em Acceleration of Convergence of a Two-level Algorithm by Smoothing
% Transfer Operator},
% Appl. Math. 37(1992).
% \bibitem{FMS}
% {\sc P. Van\v{e}k},
% {\em Fast multigrid solver},
% Applications of Mathematics 40(1995), no. 1.
% \bibitem{amg-theory}
% {\sc P. Van\v{e}k, M. Brezina, J. Mandel},
% {\em Convergence of Algebraic Multigrid Based on Smoothed Aggregations},
% Numer. Math. 88(2001), no. 3.
% \end{thebibliography}
