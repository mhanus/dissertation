\chapter{Lack of rotational invariance of the $\SN$ advection-reaction operator}

\begin{theorem*}\label{thm:commut_NTE}
Let $L : V\to V$ be the advection-collision operator of the NTE satisfying
$RL = LR$ for all $R\in\mathrm{SO}(3)$. 
Then the corresponding $\SN$ approximation operator
\begin{equation}\label{eq:sn_op_app}
	\Projop[S_N]\op{L}\Projop[S_N]
\end{equation}
(with $\Projop[S_N]$ defined by \eqref{eq:proj_sn}, \eqref{eq:map_SN}, \eqref{eq:map_SN_inv} and particular
ordinates and quadrature sets \mbox{$\omega = \{\bomega_n\}_{\idxset{M}}$, $\mathcal{W} = \{w_n\}_{\idxset{M}}$})
satisfies $$
\op{R}\op{L}\Projop[S_N] = \op{L}\Projop[S_N]\op{R}\quad \forall R\in\mathrm{SO}(3)
$$
if and only if
\begin{equation}\label{eq:thm1_cond}
	\forall n\in \idxset{M}\ \exists m\in\idxset{M}: \mat{R}\bomega_n = \bomega_m \quad \forall \mat{R}\in\mathrm{SO}(3).
\end{equation} 
\end{theorem*}
\begin{remark*}
	Note that the conditions \eqref{eq:thm1_cond} can be satisfied only in the limit \mbox{$M\to\infty$}.
\end{remark*}
\begin{proof}
First, recall from \sref{sec:opsn} that for any $\SN$ approximate function \mbox{$f_{S_N}\in V_{S_N}$}, 
$$
	\Projop[SN]f_{S_N} = f_{S_N}.
$$
Also, for $\psi_{S_N} \in V_{S_N}$:
$$
	\bomega\cdot\nabla\psi_{S_N}(\br,\bomega) = \left.\pw{
		\bomega_n\cdot\nabla\psi(\br,\bomega_n)}{if $\bomega = \bomega_n$ for $\bomega_n\in\omega$}
		{0}{if $\bomega \not \in \omega$.}\right\} \in V_{S_N}
$$
where we recall the characterization of $V_{S_N}$ as a space of functions that, as functions of $\bomega$, vanish
everywhere on $\Sphere$ except points corresponding to the selected ordinates set. Therefore for $\psi_{S_N} \in
V_{S_N}$, 
$$
	L\psi_{S_N} \equiv \bomega\cdot\nabla\psi_{S_N} + \sigma_t\psi_{S_N} \in V_{S_N}.
$$
Since \mbox{$\Rng (\PiSN) = V_{S_N}$}, it follows that the $\SN$ operator \eqref{eq:sn_op_app} can be simplified as:
\begin{equation}\label{eq:sn_op2}
	\Projop[S_N]\op{L}\Projop[S_N] = \op{L} \Projop[SN].
\end{equation}
Because of the commutativity of $L$ and $R$, it therefore suffices to show that $\Projop[S_N]$ commutes
with $\op{R}$, that is (using def. \eqref{eq:proj_sn})
\begin{equation}\label{eq:thm1_point}
	R \PiSN \PihSN \psi = \PiSN \PihSN R \psi\quad  \forall \psi \in V, 
\end{equation}
if and only if the condition \eqref{eq:thm1_cond} holds.\\[.2em] 

As in \sref{sec:opsn}, we will suppress the spatial dependence as the $\SN$ approximation concerns only the angular
dependence. For any $\psi\in V$, we have
\begin{equation*}
	\PihSN R\psi(\bomega) = \colset{R\psi(\bomega_n)}{\idxset{M}} = \colset{\psi(\mat{R}^T\bomega_n)}{\idxset{M}}.
\end{equation*}
Applying $\PiSN$ thus yields a function $f_\psi = \PiSN\PihSN R\psi \in V_{S_N}$ such that
\begin{equation}\label{eq:thm1_f}
f_\psi(\bomega) = \pw{\psi(\mat{R}^T\bomega_n)}{if $\bomega = \bomega_n$ for $\bomega_n\in\omega$}{0}{if $\bomega \not
	\in \omega$.}
\end{equation}
On the other hand, let $u = \PiSN\PihSN\psi$. Then
$$
u(\bomega) = \pw{\psi(\bomega_n)}{if $\bomega = \bomega_n$ for $\bomega_n\in\omega$}{0}{if $\bomega \not
	\in \omega$}
$$ 
and the rotated function $g_\psi(\bomega) = R\PiSN\PihSN\psi = R u(\bomega)$ is given by
$$
g_\psi(\bomega) = u(\mat{R}^T\bomega) = \pw{\psi(\bomega_n)}{if $\mat{R}^T\bomega = \bomega_n$ for
$\bomega_n\in\omega$}{0}{if $\mat{R}^T\bomega \not \in \omega$}
$$
or, equivalently, by
\begin{equation}\label{eq:thm1_g}
g_\psi(\bomega) = \pw{\psi(\bomega_n)}{if $\bomega = \mat{R}\bomega_n$ for
$\bomega_n\in\omega$}{0}{if $\bomega \not \in \mat{R}\omega$}
\end{equation}
(where action of $\mat{R}$ on the set $\omega$ is understood element-wise). If the ordinate set $\omega$ contains for
any ordinate $\bomega_n$ also its rotated copy $\mat{R}\bomega_n$, i.e., condition \eqref{eq:thm1_cond} holds, then also
$$
	\bomega_n = \mat{R}^T\bomega_m
$$
and after simple substitution in \eqref{eq:thm1_g} and renumbering in \eqref{eq:thm1_f}, we obtain
$$
	f_\psi(\bomega) = g_\psi(\bomega).
$$
In order for this equality to hold true for any $\psi \in V$ (and hence for \eqref{eq:thm1_point} to hold true), the
condition \eqref{eq:thm1_f} is also necessary.
\end{proof}
\comment{
\begin{theorem}\label{thm:commut_NTE2}
Let $K : V\to V$ be the transfer operator of the NTE satisfying
$RK = KR$ for all $R\in\mathrm{SO}(3)$. 
Then the corresponding $\SN$ approximation operator
\begin{equation}\label{eq:sn_op_app}
	\Projop[S_N]\op{K_{S_N}}\Projop[S_N]
\end{equation}
where $\op{K}_{S_N}: V_{S_N} \to V$ is defined by
$$
	\op{K}_{S_N} f(\bomega) = \sum_{n=1}^{M} w_n\kappa(\bomega\cdot\bomega_n) f(\bomega_n), \quad \bomega_n\in \omega,\
	w_n\in \mathcal{W} $$
(and definitions of $\Projop[SN]$, $\omega$, $\mathcal{W}$ are as in previous theorem),  satisfies
$$
\op{R}\Projop[S_N] \op{K}_{S_N}\Projop[S_N] = \Projop[S_N] \op{K}_{S_N}\Projop[S_N]\op{R}\quad \forall
R\in\mathrm{SO}(3) $$
if and only if the condition \eqref{eq:thm1_cond} of Thm. \ref{thm:commut_NTE} holds.
\end{theorem}
\begin{proof}

\end{proof}
}
